\documentclass[11pt,a4paper]{article}
\usepackage[utf8]{inputenc}
\usepackage{latexsym, 
			amsfonts, 
			amssymb, 
			amsthm, 
			calc, 
%			enumerate, 
			fontawesome5,
			graphicx,
			longtable, 
            makecell,
            ngerman, % Deutsche Namen Inhaltsverzeichnis etc
			nicefrac, % for \nicefrac{1}{2}
			paralist,    % for compactenum
            pifont,
            rotating,
            tabularx,
            xcolor,
			}
\usepackage[hidelinks,linkbordercolor={1 1 1}]{hyperref}
\usepackage{pifont} 
\newcommand{\xmark}{\ding{55}}
\usepackage{draftwatermark}
\SetWatermarkText{Nur für die Vorab-Akkreditierung, Juni 2024}
\SetWatermarkScale{1}
%\SetWatermarkColor[rgb]{1,0.95,0.95}

%\usepackage{ae,array,calc,graphicx,ifthen,listofitems,lmodern,lscape,ngerman,pifont,verbatim,wasysym,tikz,tabularx}

\setlength{\topmargin}{-2.5cm}
\setlength{\oddsidemargin}{-1cm}
\setlength{\textwidth}{18cm}
\setlength{\textheight}{26.5cm}
\parindent0em
\parskip1ex




% ...
\begin{document}

\hrule\vskip1pt\hrule\medskip

\resizebox{\textwidth}{!}{Universität Freiburg -- Mathematisches Institut}

\medskip
\resizebox{\textwidth}{!}{Wintersemester 2024/25}

\bigskip
\resizebox{\textwidth}{!}{Aktuelle Ergänzungen der Modulhandbücher}

\medskip\hrule\vskip1pt\hrule

\bigskip
\bigskip

\setlength{\baselineskip}{20pt}
\hfill Version vom \today

\thispagestyle{empty}
\clearpage
\tableofcontents

\clearpage
\addcontentsline{toc}{section}{{Hinweise}}\addtocontents{toc}{\medskip\hrule\medskip}

%%%%%%%%%%%%%%%%%%%%%%%%%%%%%%%%%%%%%%%%%%%%%%%%%%%%%%%%%%%%%%%%%%%%%%%%%%%%%%%%
%% Über dieses Dokument
%%%%%%%%%%%%%%%%%%%%%%%%%%%%%%%%%%%%%%%%%%%%%%%%%%%%%%%%%%%%%%%%%%%%%%%%%%%%%%%%

\section*{Über dieses Dokument}

\addcontentsline{toc}{subsection}{Über dieses Dokument}
Dieses Dokument wurde zur Vorab-Akkreditierung des {\em MSc Mathematics in Data and Technology} erstellt. Die in diesem Dokument gegebenen Informationen stellen im prüfungs- und akkreditierungsrechtlichen Sinn eine Ergänzung der \href{https://www.math.uni-freiburg.de/lehre/pruefungsamt/modulhandbuecher.html}{Modulhandbücher} dar. Es zeigt, wie das konkrete Angebot an Lehrveranstaltungen in den Modulen {\em Basics in Applied Mathematics}, {\em Advanced Lecture in Numerics/Stochastics}, {\em Electives in Data}, {\em Elective} und {\em Seminar} des Studiengangs im Wintersemester 2024/25 aussieht. 

Wir geben unten eine Tabelle aller vom Mathematischen Instituts angebotenen Veranstaltungen, die im MSc Mathematics in Data and Technology in einem der genannten Modulen verwendbar sind. Die genauen Anforderungen an Studien- und Prüfungsleistungen in den entsprechenden Modulen sind beispielhaft auf den nächsten Seiten aufgeführt. In der von der Studienkommission Mathematik zu Beginn des Wintersemesters 2024/25 verabschiedeten Version dieses Dokuments werden die entsprechenden Daten für alle Veranstaltungen der Tabelle aufgeführt sein.

\bigskip 

\begin{longtable}{p{.6\textwidth}|c|c|c|c|c}
\parbox[b]{6cm}{Die {\bf fett} gedruckten Veranstaltungen sind auf den nächsten Seiten genauer dargestellt. \vspace{1cm}} & \rotatebox[origin=l]{90}{\parbox{3cm}{Basics in Applied Mathematics}} & \rotatebox[origin=l]{90}{\parbox{3.8cm}{Advanced Lecture in Numerics/Stochsatics}} & \rotatebox[origin=l]{90}{\parbox{3cm}{Electives in Data}} 
& \rotatebox[origin=l]{90}{\parbox{3cm}{Electives}} 
& \rotatebox[origin=l]{90}{\parbox{3cm}{Seminar}} 
\\[1ex] \hline & & & & & \\[.5ex]
\endhead
Algebra und Zahlentheorie (W. Soergel) & & &  & \xmark & \\[1ex]
% Analysis III (P. Dondl)  & & & & \xmark & \\[1ex]
% Erweiterung der Analysis (N. Große) & & &  \xmark & & \\[1ex]
{\bf Basics in Applied Mathematics} (P. Dondl, A. Rohde) & \xmark & & & & \\[1ex]
Algebraic Number Theory (A. Oswal) & & & & \xmark & \\[1ex]
Differentialgeometrie (S. Goette) & & & & \xmark & \\[1ex]
Einführung in partielle Differentialgleichungen (G. Wang) & & & \xmark & & \\[1ex]
Funktionentheorie (N. N.) & & & & \xmark & \\[1ex]
{\bf Introduction to Theory and Numerics of Partial Differential Equations} (D. Salimova) & &  \xmark &  \xmark & & \\[1ex]
Mathematical Statistics	(E. A. v. Hammerstein) & &  \xmark &  \xmark & & \\[1ex]
Numerical Optimal Control (M. Diehl) & &  \xmark &  \xmark & & \\[1ex]
Probability Theory II – Stochastic Processes (P. Pfaffelhuber) &  & \xmark &  \xmark & & \\[1ex]
Probability Theoriy III – Stochastic Integration und Financial Mathematics (T. Schmidt) &  &\xmark &  \xmark & & \\[1ex]
Semi-algebraische Geometrie	 (A. Huber-Klawitter, A. Martín Pizarro) & & & & \xmark & \\[1ex]
Set Theory – Independence Proofs (M. Levine) & & & & \xmark & \\[1ex]
Theory and Numerics of Partial Differential Equations – Non-linear Problems (S. Bartels) & &  \xmark &  \xmark & & \\[1ex]
Futures and Options (E. Lütkebohmert-Holtz) & & &  \xmark & & \\[1ex]
Lie-Gruppen und symmetrische Räume (M. Stegemeyer) & & & & \xmark & \\[1ex]
{\bf Measure Theory} (P. Pfaffelhuber) & & & &  \xmark & \\[1ex]
% Numerical Optimal Control (M. Diehl) & & & & & \\[1ex]
{\bf Praktische Übung zu {\em Introduction to Theory and Numerics of Partial Differential Equations}} (D. Salimova) & & & &  \xmark & \\[1ex]
Seminar: tba (E. A. v. Hammerstein) & & & & &  \xmark \\[1ex]
Seminar: Theorie der nicht-kommutativen Algebren (A. Huber-Klawitter) & & & & &  \xmark \\[1ex]
Seminar zur algebraischen Topologie (S. Goette) & & & & &  \xmark \\[1ex]
Seminar: Minimalflächen (G. Wang) & & & & &  \xmark \\[1ex]
Seminar: Machine Learning and Stochastic Analysis (T. Schmidt) & & & & &  \xmark \\[1ex]
{\bf Seminar: Medical Data Science} (H. Binder) & & & & &  \xmark \\[1ex]
\end{longtable}


%%%%%%%%%%%%%%%%%%%%%%%%%%%%%%%%%%%%%%%%%%%%%%%%%%%%%%%%%%%%%%%%%%%%%%%%%%%%%%%%
%%  BEGINN VERANSTALTUNGSTEIL
%%%%%%%%%%%%%%%%%%%%%%%%%%%%%%%%%%%%%%%%%%%%%%%%%%%%%%%%%%%%%%%%%%%%%%%%%%%%%%%%

\clearpage
\phantomsection
\thispagestyle{empty}
\vspace*{\fill}
\begin{center}
\Huge\bfseries Beispielveranstaltungen für den 
\\
MSc Mathematics in Data and Technology
\end{center}
\addcontentsline{toc}{section}{\textbf{Beispielveranstaltungen}}
\addtocontents{toc}{\medskip\hrule\medskip}\vspace*{\fill}\vspace*{\fill}\clearpage

\vfill

\thispagestyle{empty}
\clearpage


\clearpage\hrule\vskip1pt\hrule 
\section*{\Large Basics in Applied Mathematics}
\addcontentsline{toc}{subsection}{Basics in Applied Mathematics\ \textcolor{gray}{(\textit{Patrick Dondl, Angelika Rohde})}}
\vskip-2ex  
{\itshape Patrick Dondl, Angelika Rohde}, 
Assistenz: 
{\itshape N.N.}
\\
4-stündige Vorlesung mit 2-stündiger Übung und 2-stündiger Pratkischer Übung \\
Vorlesung: Di, Do 8-10, HS II, Albertstr. 23b\\ Praktische Übung: Termine werden noch festgelegt


\begin{tabularx}{\textwidth}{ p{.5\textwidth}
    X
    }
    & 
    \makecell[c]{\rotatebox[origin=l]{90}{\parbox{
    3
        cm}{\begin{flushleft}
        Basics in Applied Mathematics (MScData 2024)
    \end{flushleft} }}} 
    \\[2ex] \hline 
    \rule[0mm]{0cm}{.6cm}Erreichen von mindestens 50\% der Punkte, die insgesamt durch die Bearbeitung für die Übung ausgegebenen Übungsaufgaben erreicht werden können. (SL)\rule[-3mm]{0cm}{0cm}
    &
    \makecell[c]{\xmark}
    \\
    \rule[0mm]{0cm}{.6cm}Mindestens 1-maliges Vorrechnen von Übungsaufgaben im Tutorat. (SL)\rule[-3mm]{0cm}{0cm}
    &
    \makecell[c]{\xmark}
    \\
    \rule[0mm]{0cm}{.6cm}Klausur (ein- bis dreistündig). (SL)\rule[-3mm]{0cm}{0cm}
    &
    \makecell[c]{\xmark}
    \\
\end{tabularx}



\clearpage\hrule\vskip1pt\hrule 
\section*{\Large Introduction to Theory and Numerics of Partial Differential Equations}
\addcontentsline{toc}{subsection}{Introduction to Theory and Numerics of Partial Differential Equations\ \textcolor{gray}{(\textit{Diora Salimova})}}
\vskip-2ex  
{\itshape Diora Salimova}, 
Assistenz: 
{\itshape N.N.}
\\
4-stündige Vorlesung mit 2-stündiger Übung \\
Vorlesung: Di, Fr 12-14, SR 226, Hermann-Herder-Str. 10\\ Übung (2-stündig): Termin wird noch festgelegt


\begin{tabularx}{\textwidth}{ p{.5\textwidth}
    X
    X
    X
    X
    X
    X
    }
    & 
    \makecell[c]{\rotatebox[origin=l]{90}{\parbox{
    8
        cm}{\begin{flushleft}
        Modul Mathematik (MSc 2014)
    \end{flushleft} }}} 
    & 
    \makecell[c]{\rotatebox[origin=l]{90}{\parbox{
    8
        cm}{\begin{flushleft}
        Modul Angewandte Mathematik (MSc 2014)
    \end{flushleft} }}} 
    & 
    \makecell[c]{\rotatebox[origin=l]{90}{\parbox{
    8
        cm}{\begin{flushleft}
        Teile des Vertiefungsmoduls (MSc 2014)
    \end{flushleft} }}} 
    & 
    \makecell[c]{\rotatebox[origin=l]{90}{\parbox{
    8
        cm}{\begin{flushleft}
        Wahlpflichtmodul Mathematik (BSc 2021)
    \end{flushleft} }}} 
    & 
    \makecell[c]{\rotatebox[origin=l]{90}{\parbox{
    8
        cm}{\begin{flushleft}
        Advanced Lecture in Numerics/Stochastics (MScData 2024)
    \end{flushleft} }}} 
    & 
    \makecell[c]{\rotatebox[origin=l]{90}{\parbox{
    8
        cm}{\begin{flushleft}
        Electives in Data (MScData 2024)
    \end{flushleft} }}} 
    \\[2ex] \hline 
    \rule[0mm]{0cm}{.6cm}Mündliche Abschlussprüfung (Dauer ca. 30 Minuten) (PL)\rule[-3mm]{0cm}{0cm}
    &
    \makecell[c]{\xmark}
    &
    \makecell[c]{\xmark}
    &
    &
    &
    \makecell[c]{\xmark}
    &
    \\
    \rule[0mm]{0cm}{.6cm}Mündliche Abschlussprüfung über alle Teile des Moduls (Dauer ca. 45 Minuten) (PL)\rule[-3mm]{0cm}{0cm}
    &
    &
    &
    \makecell[c]{\xmark}
    &
    &
    &
    \\
    \rule[0mm]{0cm}{.6cm}Klausur (ein- bis dreistündig). (PL)\rule[-3mm]{0cm}{0cm}
    &
    &
    &
    &
    \makecell[c]{\xmark}
    &
    &
    \makecell[c]{\xmark}
    \\
    \rule[0mm]{0cm}{.6cm}Klausur (ein- bis dreistündig). (SL)\rule[-3mm]{0cm}{0cm}
    &
    \makecell[c]{\xmark}
    &
    \makecell[c]{\xmark}
    &
    \makecell[c]{\xmark}
    &
    &
    \makecell[c]{\xmark}
    &
    \\
    \rule[0mm]{0cm}{.6cm}Erreichen von mindestens 50\% der Punkte, die insgesamt durch die Bearbeitung für die Übung ausgegebenen Übungsaufgaben erreicht werden können. (SL)\rule[-3mm]{0cm}{0cm}
    &
    \makecell[c]{\xmark}
    &
    \makecell[c]{\xmark}
    &
    \makecell[c]{\xmark}
    &
    \makecell[c]{\xmark}
    &
    \makecell[c]{\xmark}
    &
    \makecell[c]{\xmark}
    \\
    \rule[0mm]{0cm}{.6cm}Mindestens 1-maliges Vorrechnen von Übungsaufgaben im Tutorat. (SL)\rule[-3mm]{0cm}{0cm}
    &
    \makecell[c]{\xmark}
    &
    \makecell[c]{\xmark}
    &
    \makecell[c]{\xmark}
    &
    \makecell[c]{\xmark}
    &
    \makecell[c]{\xmark}
    &
    \makecell[c]{\xmark}
    \\
\end{tabularx}



\clearpage\hrule\vskip1pt\hrule 
\section*{\Large Mathematical Statistics}
\addcontentsline{toc}{subsection}{Mathematical Statistics\ \textcolor{gray}{(\textit{E. A. von Hammerstein})}}
\vskip-2ex  
{\itshape E. A. von Hammerstein}, 
Assistenz: 
{\itshape N.N.}
\\
4-stündige Vorlesung mit 2-stündiger Übung \\
Vorlesung: Di, Do 14-16, HS Weismann-Haus, Albertstr. 21a\\ Übung (2-stündig): Termin wird noch festgelegt


\begin{tabularx}{\textwidth}{ p{.5\textwidth}
    X
    X
    X
    X
    X
    X
    }
    & 
    \makecell[c]{\rotatebox[origin=l]{90}{\parbox{
    8
        cm}{\begin{flushleft}
        Modul Mathematik (MSc 2014)
    \end{flushleft} }}} 
    & 
    \makecell[c]{\rotatebox[origin=l]{90}{\parbox{
    8
        cm}{\begin{flushleft}
        Modul Angewandte Mathematik (MSc 2014)
    \end{flushleft} }}} 
    & 
    \makecell[c]{\rotatebox[origin=l]{90}{\parbox{
    8
        cm}{\begin{flushleft}
        Teile des Vertiefungsmoduls (MSc 2014)
    \end{flushleft} }}} 
    & 
    \makecell[c]{\rotatebox[origin=l]{90}{\parbox{
    8
        cm}{\begin{flushleft}
        Wahlpflichtmodul Mathematik (BSc 2021)
    \end{flushleft} }}} 
    & 
    \makecell[c]{\rotatebox[origin=l]{90}{\parbox{
    8
        cm}{\begin{flushleft}
        Advanced Lecture in Numerics/Stochastics (MScData 2024)
    \end{flushleft} }}} 
    & 
    \makecell[c]{\rotatebox[origin=l]{90}{\parbox{
    8
        cm}{\begin{flushleft}
        Electives in Data (MScData 2024)
    \end{flushleft} }}} 
    \\[2ex] \hline 
    \rule[0mm]{0cm}{.6cm}Mündliche Abschlussprüfung (Dauer ca. 30 Minuten) (PL)\rule[-3mm]{0cm}{0cm}
    &
    \makecell[c]{\xmark}
    &
    \makecell[c]{\xmark}
    &
    &
    &
    \makecell[c]{\xmark}
    &
    \\
    \rule[0mm]{0cm}{.6cm}Mündliche Abschlussprüfung über alle Teile des Moduls (Dauer ca. 45 Minuten) (PL)\rule[-3mm]{0cm}{0cm}
    &
    &
    &
    \makecell[c]{\xmark}
    &
    &
    &
    \\
    \rule[0mm]{0cm}{.6cm}Erreichen von mindestens 50\% der Punkte, die insgesamt durch die Bearbeitung für die Übung ausgegebenen Übungsaufgaben erreicht werden können. (SL)\rule[-3mm]{0cm}{0cm}
    &
    \makecell[c]{\xmark}
    &
    \makecell[c]{\xmark}
    &
    \makecell[c]{\xmark}
    &
    \makecell[c]{\xmark}
    &
    \makecell[c]{\xmark}
    &
    \makecell[c]{\xmark}
    \\
    \rule[0mm]{0cm}{.6cm}Mindestens 1-maliges Vorrechnen von Übungsaufgaben im Tutorat. (SL)\rule[-3mm]{0cm}{0cm}
    &
    \makecell[c]{\xmark}
    &
    \makecell[c]{\xmark}
    &
    \makecell[c]{\xmark}
    &
    \makecell[c]{\xmark}
    &
    \makecell[c]{\xmark}
    &
    \makecell[c]{\xmark}
    \\
\end{tabularx}



\clearpage\hrule\vskip1pt\hrule 
\section*{\Large Measure Theory for Probabilists}
\addcontentsline{toc}{subsection}{Measure Theory for Probabilists\ \textcolor{gray}{(\textit{Peter Pfaffelhuber})}}
\vskip-2ex  
{\itshape Peter Pfaffelhuber}, 
Assistenz: 
{\itshape Samuel Adeosun}
\\
Online-Kurs mit 2-stündiger Übung \\
Vorlesung (2-stündig): asynchrone Videos\\ Übung (2-stündig): Termin wird noch festgelegt


\begin{tabularx}{\textwidth}{ p{.5\textwidth}
    X
    }
    & 
    \makecell[c]{\rotatebox[origin=l]{90}{\parbox{
    3
        cm}{\begin{flushleft}
        Electives (MScData 2024)
    \end{flushleft} }}} 
    \\[2ex] \hline 
    \rule[0mm]{0cm}{.6cm}Erreichen von mindestens 50\% der Punkte, die insgesamt durch die Bearbeitung für die Übung ausgegebenen Übungsaufgaben erreicht werden können. (SL)\rule[-3mm]{0cm}{0cm}
    &
    \makecell[c]{\xmark}
    \\
    \rule[0mm]{0cm}{.6cm}Mindestens 1-maliges Vorrechnen von Übungsaufgaben im Tutorat. (SL)\rule[-3mm]{0cm}{0cm}
    &
    \makecell[c]{\xmark}
    \\
\end{tabularx}



\clearpage\hrule\vskip1pt\hrule 
\section*{\Large Praktische Übung zu 'Introduction to Theory and Numerics of Partial Differential Equations'}
\addcontentsline{toc}{subsection}{Praktische Übung zu 'Introduction to Theory and Numerics of Partial Differential Equations'\ \textcolor{gray}{(\textit{Diora Salimova})}}
\vskip-2ex  
{\itshape Diora Salimova}, 
Assistenz: 
{\itshape N.N.}
\\
2-stündige Praktische Übung \\
2-stündig: Termin wird noch festgelegt


\begin{tabularx}{\textwidth}{ p{.5\textwidth}
    X
    X
    X
    }
    & 
    \makecell[c]{\rotatebox[origin=l]{90}{\parbox{
    3
        cm}{\begin{flushleft}
        Teil des Wahlmoduls (MSc 2014)
    \end{flushleft} }}} 
    & 
    \makecell[c]{\rotatebox[origin=l]{90}{\parbox{
    3
        cm}{\begin{flushleft}
        Teil des Wahlmoduls (BSc 2021)
    \end{flushleft} }}} 
    & 
    \makecell[c]{\rotatebox[origin=l]{90}{\parbox{
    3
        cm}{\begin{flushleft}
        Electives (MScData 2024)
    \end{flushleft} }}} 
    \\[2ex] \hline 
    \rule[0mm]{0cm}{.6cm}Regelmäßige Teilnahme an der Veranstaltung (wie in der Prüfungsordnung definiert). (SL)\rule[-3mm]{0cm}{0cm}
    &
    \makecell[c]{\xmark}
    &
    \makecell[c]{\xmark}
    &
    \makecell[c]{\xmark}
    \\
    \rule[0mm]{0cm}{.6cm}Erreichen von mindestens 50\% der Punkte, die insgesamt durch die Bearbeitung der für die Praktische Übung ausgegebenen Programmieraufgaben erreicht werden können. (SL)\rule[-3mm]{0cm}{0cm}
    &
    \makecell[c]{\xmark}
    &
    \makecell[c]{\xmark}
    &
    \makecell[c]{\xmark}
    \\
\end{tabularx}



\clearpage\hrule\vskip1pt\hrule 
\section*{\Large Medical Data Science}
\addcontentsline{toc}{subsection}{Medical Data Science\ \textcolor{gray}{(\textit{Harald Binder})}}
\vskip-2ex  
{\itshape Harald Binder}, 
Assistenz: 
{\itshape N.N.}
\\
Seminar \\
Mi 10-11:30, HS Medizinische Biometrie, Stefan-Meier-Str. 26


\begin{tabularx}{\textwidth}{ p{.5\textwidth}
    X
    X
    X
    }
    & 
    \makecell[c]{\rotatebox[origin=l]{90}{\parbox{
    3
        cm}{\begin{flushleft}
        Mathematisches Seminar (BSc 2021)
    \end{flushleft} }}} 
    & 
    \makecell[c]{\rotatebox[origin=l]{90}{\parbox{
    3
        cm}{\begin{flushleft}
        Seminar A/B (MSc 2014)
    \end{flushleft} }}} 
    & 
    \makecell[c]{\rotatebox[origin=l]{90}{\parbox{
    3
        cm}{\begin{flushleft}
        Seminar (MScData 2024)
    \end{flushleft} }}} 
    \\[2ex] \hline 
    \rule[0mm]{0cm}{.6cm}Vortrag (Dauer 45 bis 90 Minuten) (PL)\rule[-3mm]{0cm}{0cm}
    &
    \makecell[c]{\xmark}
    &
    \makecell[c]{\xmark}
    &
    \makecell[c]{\xmark}
    \\
    \rule[0mm]{0cm}{.6cm}Regelmäßige Teilnahme an der Veranstaltung (wie in der Prüfungsordnung definiert). (SL)\rule[-3mm]{0cm}{0cm}
    &
    \makecell[c]{\xmark}
    &
    \makecell[c]{\xmark}
    &
    \makecell[c]{\xmark}
    \\
\end{tabularx}





\newpage\section*{Abkürzungen}
\addcontentsline{toc}{section}{Abkürzungen}

\begin{tabular}{p{.2\textwidth}p{.8\textwidth}}
MScData (2014) & Master of Science, Prüfungsordnung 2014 \\
BSc (2021) & Bachelor of Science Mathematik, Prüfungsordnung 2021 \\
MScData (2024) & Master of Science Mathematics in Data and Technology, Prüfungsordnung 2024 \\
PL & Prüfungsleistung \\
SL & Studienleistung \\
\end{tabular}




\end{document}

