
\clearpage
\phantomsection
\thispagestyle{empty}
\vspace*{\fill}
\begin{center}
\Huge\bfseries 1a. Einführende Vorlesungen und Pflichtvorlesungen der verschiedenen Studiengänge
\end{center}
\addcontentsline{toc}{section}{\textbf{1a. Einführende Vorlesungen und Pflichtvorlesungen der verschiedenen Studiengänge}}
\addtocontents{toc}{\medskip\hrule\medskip}\vspace*{\fill}\vspace*{\fill}\clearpage
\vfill
\thispagestyle{empty}
\clearpage

\clearpage\hrule\vskip1pt\hrule 
\section*{\Large Lineare Algebra II}
\addcontentsline{toc}{subsection}{Lineare Algebra II\ \textcolor{gray}{(\em Angelika Rohde)}}
\vskip-2ex  
Angelika Rohde, Assistenz: Johannes Brutsche\\
Vorlesung: Di, Do, 8--10, HS Rundbau, \href{https://www.openstreetmap.org/?mlat=48.00156\&mlon=7.84931\#map=19/48.00156/7.84931}{Albertstr. 21}\\
\subsubsection*{\Large Inhalt:}
Magnam amet quaerat voluptatem. Quaerat quiquia quiquia porro porro. Porro non quisquam aliquam tempora porro neque. Non velit dolorem sed aliquam magnam. Ipsum aliquam quaerat sed dolorem tempora quisquam.
\subsubsection*{\Large Literatur:}
Est ipsum adipisci est.
\subsubsection*{\Large Vorkenntnisse:}
Man erhält so leicht, dass $x_{1/2} = \frac{-b \pm \sqrt{b^2 - 4ac}}{2a}$
\subsubsection*{\Large Verwendbarkeit, Studien- und Prüfungsleistungen:}
\begin{tabularx}{\textwidth}{ p{.5\textwidth}
    X
    X
    X
    }
    & 
    \makecell[c]{\rotatebox[origin=l]{90}{\parbox{
    4
        cm}{\begin{flushleft}
        Lineare Algebra II (BSc, 2HfB, BScPhysik2019)
    \end{flushleft} }}} 
    & 
    \makecell[c]{\rotatebox[origin=l]{90}{\parbox{
    4
        cm}{\begin{flushleft}
        Teil des Moduls "Lineare Algebra" (BSc21, 2HfB21, MEH21, MEB21, GymPO)
    \end{flushleft} }}} 
    & 
    \makecell[c]{\rotatebox[origin=l]{90}{\parbox{
    4
        cm}{\begin{flushleft}
        Lineare Algebra II (als fachfremdes Wahlmodul) (BScInfo)
    \end{flushleft} }}} 
    \\[2ex] \hline 
    \rule[0mm]{0cm}{.6cm}Regelmäßige Teilnahme (wie in der Prüfungsordnung definiert) an einem der Tutorate zur Vorlesung. (SL) \rule[-3mm]{0cm}{0cm}
    &
    \makecell[c]{\xmark}
    &
    \makecell[c]{\xmark}
    &
    \makecell[c]{\xmark}
    \\
    \rule[0mm]{0cm}{.6cm}Erreichen von mindestens 50% der Punkte, die insgesamt durch die Bearbeitung der für die Übung ausgegebenen Übungsaufgaben erreicht werden können. (SL) \rule[-3mm]{0cm}{0cm}
    &
    \makecell[c]{\xmark}
    &
    \makecell[c]{\xmark}
    &
    \makecell[c]{\xmark}
    \\
    \rule[0mm]{0cm}{.6cm}Mündliche Prüfung (Dauer ca. 30 Minuten) über Lineare Algebra I und II am Ende des Moduls. (Die bestandene Klausur zu Lineare Algebra I und die bestandene Übung zu Lineare Algebra II sind Zulassungsvoraussetzungen). (PL) \rule[-3mm]{0cm}{0cm}
    &
    \makecell[c]{\xmark}
    &
    \makecell[c]{\xmark}
    &
    \\
    \rule[0mm]{0cm}{.6cm}Bestehen eines mündlichen Abschlusstests. (SL) \rule[-3mm]{0cm}{0cm}
    &
    &
    &
    \makecell[c]{\xmark}
    \\
\end{tabularx}

\clearpage\hrule\vskip1pt\hrule 
\section*{\Large Analysis II}
\addcontentsline{toc}{subsection}{Analysis II\ \textcolor{gray}{(\em Patrick Dondl)}}
\vskip-2ex  
Patrick Dondl, Assistenz: Coffi Aristide Hounkpe\\
Vorlesung: Mo, Mi, 8--10, HS Rundbau, \href{https://www.openstreetmap.org/?mlat=48.00156\&mlon=7.84931\#map=19/48.00156/7.84931}{Albertstr. 21}\\
\subsubsection*{\Large Inhalt:}
Non ipsum porro quaerat modi. Voluptatem non dolorem velit adipisci. Ut ut amet dolor modi. Dolore dolor sed non consectetur. Porro labore quisquam sit sit non etincidunt.
\subsubsection*{\Large Literatur:}
Neque quaerat sit numquam.
\subsubsection*{\Large Vorkenntnisse:}
Man erhält so leicht, dass $x_{1/2} = \frac{-b \pm \sqrt{b^2 - 4ac}}{2a}$
\subsubsection*{\Large Verwendbarkeit, Studien- und Prüfungsleistungen:}
\begin{tabularx}{\textwidth}{ p{.5\textwidth}
    X
    X
    X
    }
    & 
    \makecell[c]{\rotatebox[origin=l]{90}{\parbox{
    4
        cm}{\begin{flushleft}
        Analysis II (BSc, 2HfB)
    \end{flushleft} }}} 
    & 
    \makecell[c]{\rotatebox[origin=l]{90}{\parbox{
    4
        cm}{\begin{flushleft}
        Teil des Moduls "Analysis" (BSc21, 2HfB21, MEH21, MEB21, GymPO)
    \end{flushleft} }}} 
    & 
    \makecell[c]{\rotatebox[origin=l]{90}{\parbox{
    4
        cm}{\begin{flushleft}
        Analysis II (als fachfremdes Wahlmodul) (BScInfo)
    \end{flushleft} }}} 
    \\[2ex] \hline 
    \rule[0mm]{0cm}{.6cm}Bei wenigen Teilnehmern wird die Klausur durch eine mündliche Prüfung ersetzt. (Kommentar) \rule[-3mm]{0cm}{0cm}
    &
    \makecell[c]{\xmark}
    &
    &
    \makecell[c]{\xmark}
    \\
    \rule[0mm]{0cm}{.6cm}Mindestens zweimaliges Vorrechnen von Übungsaufgaben  im Tutorat. (SL) \rule[-3mm]{0cm}{0cm}
    &
    \makecell[c]{\xmark}
    &
    \makecell[c]{\xmark}
    &
    \makecell[c]{\xmark}
    \\
    \rule[0mm]{0cm}{.6cm}Bestehen der Abschlussklausur (Dauer 1 bis 3 Stunden). (SL) \rule[-3mm]{0cm}{0cm}
    &
    \makecell[c]{\xmark}
    &
    &
    \makecell[c]{\xmark}
    \\
    \rule[0mm]{0cm}{.6cm}Abgegebene Übungsaufgaben müssen auf Aufforderung durch den Tutor/die Tutorin hin im Tutorat präsentiert werden können. (SL) \rule[-3mm]{0cm}{0cm}
    &
    \makecell[c]{\xmark}
    &
    \makecell[c]{\xmark}
    &
    \makecell[c]{\xmark}
    \\
    \rule[0mm]{0cm}{.6cm}Regelmäßige Teilnahme (wie in der Prüfungsordnung definiert) an einem der Tutorate zur Vorlesung. (SL) \rule[-3mm]{0cm}{0cm}
    &
    \makecell[c]{\xmark}
    &
    \makecell[c]{\xmark}
    &
    \makecell[c]{\xmark}
    \\
    \rule[0mm]{0cm}{.6cm}Erreichen von mindestens 50% der Punkte, die insgesamt durch die Bearbeitung der für die Übung ausgegebenen Übungsaufgaben erreicht werden können. (SL) \rule[-3mm]{0cm}{0cm}
    &
    \makecell[c]{\xmark}
    &
    \makecell[c]{\xmark}
    &
    \makecell[c]{\xmark}
    \\
    \rule[0mm]{0cm}{.6cm}Mündliche Prüfung (Dauer ca. 30 Minuten) über Analysis I und II am Ende des Moduls. (Die bestandene Klausur zu Analysis I und die bestandene Übung zu Analysis II sind Zulassungsvoraussetzungen). (PL) \rule[-3mm]{0cm}{0cm}
    &
    &
    \makecell[c]{\xmark}
    &
    \\
    \rule[0mm]{0cm}{.6cm}Übergangsregelung: Für Studierende in PO 2021, die die Klausur in Analysis II bereits erfolgreich absolviert haben, gelten für die Analysis-Module die Regeln der PO 2012. (Kommentar) \rule[-3mm]{0cm}{0cm}
    &
    &
    \makecell[c]{\xmark}
    &
    \\
\end{tabularx}

\clearpage\hrule\vskip1pt\hrule 
\section*{\Large Brückenkurs Lineare Algebra}
\addcontentsline{toc}{subsection}{Brückenkurs Lineare Algebra\ \textcolor{gray}{(\em Susanne Knies)}}
\vskip-2ex  
Susanne Knies\\
\subsubsection*{\Large Inhalt:}
Modi dolor voluptatem consectetur dolorem. Ipsum ipsum quisquam porro voluptatem quisquam amet quisquam. Modi dolorem sit ut numquam voluptatem. Ut eius neque eius aliquam. Sit labore ut ut non numquam labore. Modi adipisci aliquam quaerat labore quiquia. Voluptatem dolore consectetur eius. Neque numquam etincidunt neque dolorem dolor. Etincidunt voluptatem etincidunt etincidunt.
\subsubsection*{\Large Literatur:}
Adipisci ut tempora ut adipisci sed ipsum adipisci.
\subsubsection*{\Large Vorkenntnisse:}
Man erhält so leicht, dass $x_{1/2} = \frac{-b \pm \sqrt{b^2 - 4ac}}{2a}$
\subsubsection*{\Large Verwendbarkeit, Studien- und Prüfungsleistungen:}
\begin{tabularx}{\textwidth}{ p{.5\textwidth}
    }
    \\[2ex] \hline 
\end{tabularx}

\clearpage\hrule\vskip1pt\hrule 
\section*{\Large Numerik II}
\addcontentsline{toc}{subsection}{Numerik II\ \textcolor{gray}{(\em Alexei Gazca)}}
\vskip-2ex  
Alexei Gazca\\
Vorlesung: Mi, 14--16, HS Rundbau, \href{https://www.openstreetmap.org/?mlat=48.00156\&mlon=7.84931\#map=19/48.00156/7.84931}{Albertstr. 21}\\
\subsubsection*{\Large Inhalt:}
Aliquam voluptatem tempora eius. Sed sed magnam voluptatem. Dolore quisquam numquam velit. Est magnam magnam porro amet labore. Adipisci velit quiquia quiquia. Neque quiquia eius aliquam voluptatem est.
\subsubsection*{\Large Literatur:}
Numquam quaerat labore modi eius ipsum tempora quaerat.
\subsubsection*{\Large Vorkenntnisse:}
Man erhält so leicht, dass $x_{1/2} = \frac{-b \pm \sqrt{b^2 - 4ac}}{2a}$
\subsubsection*{\Large Verwendbarkeit, Studien- und Prüfungsleistungen:}
\begin{tabularx}{\textwidth}{ p{.5\textwidth}
    X
    }
    & 
    \makecell[c]{\rotatebox[origin=l]{90}{\parbox{
    4
        cm}{\begin{flushleft}
        Teil des Moduls "Numerik" (BSc, BSc21, 2HfB21, MEH21)
    \end{flushleft} }}} 
    \\[2ex] \hline 
    \rule[0mm]{0cm}{.6cm}Vorrechnen von mindestens einer Übungsaufgaben im Tutorat. (SL) \rule[-3mm]{0cm}{0cm}
    &
    \makecell[c]{\xmark}
    \\
    \rule[0mm]{0cm}{.6cm}Klausur über Numerik I und II (Dauer: 1 bis 3 Stunden). (PL) \rule[-3mm]{0cm}{0cm}
    &
    \makecell[c]{\xmark}
    \\
    \rule[0mm]{0cm}{.6cm}Die Anforderungen an die Studienleistungen gelten separat für beide Semester des Moduls! (Kommentar) \rule[-3mm]{0cm}{0cm}
    &
    \makecell[c]{\xmark}
    \\
    \rule[0mm]{0cm}{.6cm}Abgegebene Übungsaufgaben müssen auf Aufforderung durch den Tutor/die Tutorin hin im Tutorat präsentiert werden können. (SL) \rule[-3mm]{0cm}{0cm}
    &
    \makecell[c]{\xmark}
    \\
    \rule[0mm]{0cm}{.6cm}Erreichen von mindestens 50% der Punkte, die insgesamt durch die Bearbeitung der für die Übung ausgegebenen Übungsaufgaben erreicht werden können. (SL) \rule[-3mm]{0cm}{0cm}
    &
    \makecell[c]{\xmark}
    \\
    \rule[0mm]{0cm}{.6cm}Regelmäßige Teilnahme (wie in der Prüfungsordnung definiert) an einem der Tutorate zur Vorlesung. (SL) \rule[-3mm]{0cm}{0cm}
    &
    \makecell[c]{\xmark}
    \\
\end{tabularx}

\clearpage\hrule\vskip1pt\hrule 
\section*{\Large Elementargeometrie}
\addcontentsline{toc}{subsection}{Elementargeometrie\ \textcolor{gray}{(\em Nadine Große)}}
\vskip-2ex  
Nadine Große, Assistenz: Marius Amann\\
Vorlesung: Mi, 10--12, HS Weismann-Haus, \href{https://www.openstreetmap.org/?mlat=48.00170\&mlon=7.84946\#map=19/48.00170/7.84946}{Albertstr. 21a}\\
\subsubsection*{\Large Inhalt:}
Sed aliquam dolor ipsum ipsum. Porro voluptatem dolore aliquam. Etincidunt labore est labore. Dolore dolore porro sed velit sit quisquam. Adipisci etincidunt adipisci eius dolore sed. Adipisci sed consectetur quiquia quiquia sit sed. Adipisci est etincidunt aliquam. Ipsum ut modi labore ipsum aliquam consectetur voluptatem.
\subsubsection*{\Large Literatur:}
Ipsum ipsum tempora voluptatem sit eius quiquia.
\subsubsection*{\Large Vorkenntnisse:}
Man erhält so leicht, dass $x_{1/2} = \frac{-b \pm \sqrt{b^2 - 4ac}}{2a}$
\subsubsection*{\Large Verwendbarkeit, Studien- und Prüfungsleistungen:}
\begin{tabularx}{\textwidth}{ p{.5\textwidth}
    X
    X
    }
    & 
    \makecell[c]{\rotatebox[origin=l]{90}{\parbox{
    4
        cm}{\begin{flushleft}
        Modul im Wahlpflichtbereich Mathematik (BSc, BSc21)
    \end{flushleft} }}} 
    & 
    \makecell[c]{\rotatebox[origin=l]{90}{\parbox{
    4
        cm}{\begin{flushleft}
        Elementargeometrie (2HfB21, MEH21, MEB21, 2HfB)
    \end{flushleft} }}} 
    \\[2ex] \hline 
    \rule[0mm]{0cm}{.6cm}Vorrechnen von mindestens einer Übungsaufgaben im Tutorat. (SL) \rule[-3mm]{0cm}{0cm}
    &
    \makecell[c]{\xmark}
    &
    \makecell[c]{\xmark}
    \\
    \rule[0mm]{0cm}{.6cm}Klausur (Dauer: 1 bis 3 Stunden). (PL) \rule[-3mm]{0cm}{0cm}
    &
    \makecell[c]{\xmark}
    &
    \makecell[c]{\xmark}
    \\
    \rule[0mm]{0cm}{.6cm}Erreichen von mindestens 50% der Punkte, die insgesamt durch die Bearbeitung der für die Übung ausgegebenen Übungsaufgaben und Kurztests erreicht werden können. (SL) \rule[-3mm]{0cm}{0cm}
    &
    \makecell[c]{\xmark}
    &
    \makecell[c]{\xmark}
    \\
    \rule[0mm]{0cm}{.6cm}Abgegebene Übungsaufgaben müssen auf Aufforderung durch den Tutor/die Tutorin hin im Tutorat präsentiert werden können. (SL) \rule[-3mm]{0cm}{0cm}
    &
    \makecell[c]{\xmark}
    &
    \makecell[c]{\xmark}
    \\
    \rule[0mm]{0cm}{.6cm}Für das absolvierte Modul (oder ggf. den Teil des Moduls) gibt es 6 ECTS-Punkte. (Kommentar) \rule[-3mm]{0cm}{0cm}
    &
    \makecell[c]{\xmark}
    &
    \\
\end{tabularx}

\clearpage\hrule\vskip1pt\hrule 
\section*{\Large Stochastik II}
\addcontentsline{toc}{subsection}{Stochastik II\ \textcolor{gray}{(\em Ernst August v. Hammerstein)}}
\vskip-2ex  
Ernst August v. Hammerstein, Assistenz: Timo Enger\\
Vorlesung: Fr, 10--12, HS Weismann-Haus, \href{https://www.openstreetmap.org/?mlat=48.00170\&mlon=7.84946\#map=19/48.00170/7.84946}{Albertstr. 21a}\\
\subsubsection*{\Large Inhalt:}
Quisquam dolore est dolorem. Eius numquam magnam numquam. Voluptatem consectetur consectetur magnam numquam ut modi. Labore eius dolor consectetur. Dolore quisquam quiquia quaerat quaerat numquam quiquia dolorem. Ut dolorem tempora velit non.
\subsubsection*{\Large Literatur:}
Modi quiquia est numquam velit dolore est magnam.
\subsubsection*{\Large Vorkenntnisse:}
Man erhält so leicht, dass $x_{1/2} = \frac{-b \pm \sqrt{b^2 - 4ac}}{2a}$
\subsubsection*{\Large Verwendbarkeit, Studien- und Prüfungsleistungen:}
\begin{tabularx}{\textwidth}{ p{.5\textwidth}
    X
    X
    }
    & 
    \makecell[c]{\rotatebox[origin=l]{90}{\parbox{
    4
        cm}{\begin{flushleft}
        Modul im Wahlpflichtbereich Mathematik (BSc, BSc21)
    \end{flushleft} }}} 
    & 
    \makecell[c]{\rotatebox[origin=l]{90}{\parbox{
    4
        cm}{\begin{flushleft}
        Teil des Moduls "Stochastik" (BSc, 2HfB21, MEH21)
    \end{flushleft} }}} 
    \\[2ex] \hline 
    \rule[0mm]{0cm}{.6cm}Vorrechnen von mindestens einer Übungsaufgaben im Tutorat. (SL) \rule[-3mm]{0cm}{0cm}
    &
    \makecell[c]{\xmark}
    &
    \makecell[c]{\xmark}
    \\
    \rule[0mm]{0cm}{.6cm}Klausur (Dauer: 1 bis 3 Stunden). (PL) \rule[-3mm]{0cm}{0cm}
    &
    \makecell[c]{\xmark}
    &
    \\
    \rule[0mm]{0cm}{.6cm}Abgegebene Übungsaufgaben müssen auf Aufforderung durch den Tutor/die Tutorin hin im Tutorat präsentiert werden können. (SL) \rule[-3mm]{0cm}{0cm}
    &
    \makecell[c]{\xmark}
    &
    \makecell[c]{\xmark}
    \\
    \rule[0mm]{0cm}{.6cm}Für das absolvierte Modul (oder ggf. den Teil des Moduls) gibt es 5 ECTS-Punkte. (Kommentar) \rule[-3mm]{0cm}{0cm}
    &
    \makecell[c]{\xmark}
    &
    \\
    \rule[0mm]{0cm}{.6cm}Regelmäßige Teilnahme (wie in der Prüfungsordnung definiert) an einem der Tutorate zur Vorlesung. (SL) \rule[-3mm]{0cm}{0cm}
    &
    \makecell[c]{\xmark}
    &
    \makecell[c]{\xmark}
    \\
    \rule[0mm]{0cm}{.6cm}Erreichen von mindestens 50% der Punkte, die insgesamt durch die Bearbeitung der für die Übung ausgegebenen Übungsaufgaben erreicht werden können. (SL) \rule[-3mm]{0cm}{0cm}
    &
    \makecell[c]{\xmark}
    &
    \makecell[c]{\xmark}
    \\
    \rule[0mm]{0cm}{.6cm}Klausur über Stochastik I und II (Dauer: 2 bis 4 Stunden). (PL) \rule[-3mm]{0cm}{0cm}
    &
    &
    \makecell[c]{\xmark}
    \\
    \rule[0mm]{0cm}{.6cm}Die Anforderungen an die Studienleistungen gelten separat für beide Semester des Moduls! (Kommentar) \rule[-3mm]{0cm}{0cm}
    &
    &
    \makecell[c]{\xmark}
    \\
\end{tabularx}

\clearpage
\phantomsection
\thispagestyle{empty}
\vspace*{\fill}
\begin{center}
\Huge\bfseries 1b. Weiterführende vierstündige Vorlesungen
\end{center}
\addcontentsline{toc}{section}{\textbf{1b. Weiterführende vierstündige Vorlesungen}}
\addtocontents{toc}{\medskip\hrule\medskip}\vspace*{\fill}\vspace*{\fill}\clearpage
\vfill
\thispagestyle{empty}
\clearpage

\clearpage\hrule\vskip1pt\hrule 
\section*{\Large Kommutative Algebra und Einführung in die algebraische Geometrie}
\addcontentsline{toc}{subsection}{Kommutative Algebra und Einführung in die algebraische Geometrie\ \textcolor{gray}{(\em Annette Huber-Klawitter)}}
\vskip-2ex  
Annette Huber-Klawitter, Assistenz: Christoph Brackenhofer\\
Vorlesung: Di, Do, 8--10, HS II, \href{https://www.openstreetmap.org/?mlat=48.00233\&mlon=7.84788\#map=19/48.00233/7.84788}{Albertstr. 23b}\\
\subsubsection*{\Large Inhalt:}
Ut non est dolorem etincidunt eius quiquia. Quiquia ipsum sit modi ipsum. Amet modi magnam labore voluptatem eius neque etincidunt. Numquam sed non numquam dolorem amet dolorem aliquam. Dolor consectetur tempora non. Adipisci eius ipsum labore aliquam ipsum.
\subsubsection*{\Large Literatur:}
Ipsum quiquia amet aliquam etincidunt.
\subsubsection*{\Large Vorkenntnisse:}
Man erhält so leicht, dass $x_{1/2} = \frac{-b \pm \sqrt{b^2 - 4ac}}{2a}$
\subsubsection*{\Large Verwendbarkeit, Studien- und Prüfungsleistungen:}
\begin{tabularx}{\textwidth}{ p{.5\textwidth}
    X
    X
    X
    X
    }
    & 
    \makecell[c]{\rotatebox[origin=l]{90}{\parbox{
    8
        cm}{\begin{flushleft}
        Mathematische Vertiefung (MEd, MEH21)
    \end{flushleft} }}} 
    & 
    \makecell[c]{\rotatebox[origin=l]{90}{\parbox{
    8
        cm}{\begin{flushleft}
        Wahlmodul (BSc, MSc, BSc21, 2HfB21, 2HfB)
    \end{flushleft} }}} 
    & 
    \makecell[c]{\rotatebox[origin=l]{90}{\parbox{
    8
        cm}{\begin{flushleft}
        Modul im Wahlpflichtbereich Mathematik (BSc, BSc21)
    \end{flushleft} }}} 
    & 
    \makecell[c]{\rotatebox[origin=l]{90}{\parbox{
    8
        cm}{\begin{flushleft}
        Reine Mathematik, Mathematik oder Teil des Vertiefungsmoduls (MSc)
    \end{flushleft} }}} 
    \\[2ex] \hline 
    \rule[0mm]{0cm}{.6cm}Vorrechnen von mindestens einer Übungsaufgaben im Tutorat. (SL) \rule[-3mm]{0cm}{0cm}
    &
    \makecell[c]{\xmark}
    &
    \makecell[c]{\xmark}
    &
    \makecell[c]{\xmark}
    &
    \makecell[c]{\xmark}
    \\
    \rule[0mm]{0cm}{.6cm}Mündliche Prüfung (Dauer: ca. 30 Minuten). (PL) \rule[-3mm]{0cm}{0cm}
    &
    \makecell[c]{\xmark}
    &
    &
    &
    \\
    \rule[0mm]{0cm}{.6cm}Abgegebene Übungsaufgaben müssen auf Aufforderung durch den Tutor/die Tutorin hin im Tutorat präsentiert werden können. (SL) \rule[-3mm]{0cm}{0cm}
    &
    \makecell[c]{\xmark}
    &
    \makecell[c]{\xmark}
    &
    \makecell[c]{\xmark}
    &
    \makecell[c]{\xmark}
    \\
    \rule[0mm]{0cm}{.6cm}Erreichen von mindestens 50% der Punkte, die insgesamt durch die Bearbeitung der für die Übung ausgegebenen Übungsaufgaben erreicht werden können. (SL) \rule[-3mm]{0cm}{0cm}
    &
    \makecell[c]{\xmark}
    &
    \makecell[c]{\xmark}
    &
    \makecell[c]{\xmark}
    &
    \makecell[c]{\xmark}
    \\
    \rule[0mm]{0cm}{.6cm}Bestehen der Abschlussklausur (Dauer 1 bis 3 Stunden). (SL) \rule[-3mm]{0cm}{0cm}
    &
    &
    \makecell[c]{\xmark}
    &
    &
    \\
    \rule[0mm]{0cm}{.6cm}Verwendbar für die Option "Individuelle Schwerpunktgestaltung". (Kommentar) \rule[-3mm]{0cm}{0cm}
    &
    &
    \makecell[c]{\xmark}
    &
    &
    \\
    \rule[0mm]{0cm}{.6cm}Für das absolvierte Modul (oder ggf. den Teil des Moduls) gibt es 9 ECTS-Punkte. (Kommentar) \rule[-3mm]{0cm}{0cm}
    &
    &
    \makecell[c]{\xmark}
    &
    \makecell[c]{\xmark}
    &
    \\
    \rule[0mm]{0cm}{.6cm}Zählt bei Bedarf als eines der vier Module "Vorlesung mit Übung A" bis "Vorlesung mit Übung D" und deckt die Bedingung ab, dass mindestens eines davon zur Reinen Mathematik gehören muss. (Kommentar) \rule[-3mm]{0cm}{0cm}
    &
    &
    &
    \makecell[c]{\xmark}
    &
    \\
    \rule[0mm]{0cm}{.6cm}Klausur (Dauer: 1 bis 3 Stunden). (PL) \rule[-3mm]{0cm}{0cm}
    &
    &
    &
    \makecell[c]{\xmark}
    &
    \\
    \rule[0mm]{0cm}{.6cm}Die Zusammensetzung des Vertiefungsmoduls muss mit dem Prüfer/der Prüferin zuvor abgesprochen sein. Nicht alle Kombinationen sind zulässig. Die absolvierte Studienleistung in dieser Veranstaltung zählt mit 9 ECTS-Punkten in das Vertiefungsmodul. (Kommentar) \rule[-3mm]{0cm}{0cm}
    &
    &
    \makecell[c]{\xmark}
    &
    &
    \\
    \rule[0mm]{0cm}{.6cm}Mündliche Prüfung über alle Teile des Moduls (Dauer: ca. 30 Minuten, im Vertiefungsmodul ca. 45 Minuten) (PL) \rule[-3mm]{0cm}{0cm}
    &
    &
    &
    &
    \makecell[c]{\xmark}
    \\
    \rule[0mm]{0cm}{.6cm}Zählt bei Bedarf als eines der drei Module "Vorlesung mit Übung A" bis "Vorlesung mit Übung C" und deckt die Bedingung ab, dass mindestens eines davon zur Reinen Mathematik gehören muss. (Kommentar) \rule[-3mm]{0cm}{0cm}
    &
    &
    &
    \makecell[c]{\xmark}
    &
    \\
\end{tabularx}

\clearpage\hrule\vskip1pt\hrule 
\section*{\Large Funktionalanalysis}
\addcontentsline{toc}{subsection}{Funktionalanalysis\ \textcolor{gray}{(\em Ernst Kuwert)}}
\vskip-2ex  
Ernst Kuwert, Assistenz: Florian Johne\\
Vorlesung: Mo, Mi, 14--16, HS II, \href{https://www.openstreetmap.org/?mlat=48.00233\&mlon=7.84788\#map=19/48.00233/7.84788}{Albertstr. 23b}\\
\subsubsection*{\Large Inhalt:}
Velit eius eius porro. Consectetur modi quaerat porro sit dolor sit numquam. Porro ut amet consectetur quaerat. Porro sed porro sed etincidunt ut ut. Numquam aliquam velit dolore eius ut sed. Labore dolorem amet quiquia modi quisquam voluptatem.
\subsubsection*{\Large Literatur:}
Porro numquam quaerat tempora adipisci sit est.
\subsubsection*{\Large Vorkenntnisse:}
Man erhält so leicht, dass $x_{1/2} = \frac{-b \pm \sqrt{b^2 - 4ac}}{2a}$
\subsubsection*{\Large Verwendbarkeit, Studien- und Prüfungsleistungen:}
\begin{tabularx}{\textwidth}{ p{.5\textwidth}
    X
    X
    X
    X
    X
    }
    & 
    \makecell[c]{\rotatebox[origin=l]{90}{\parbox{
    8
        cm}{\begin{flushleft}
        Mathematische Vertiefung (MEd, MEH21)
    \end{flushleft} }}} 
    & 
    \makecell[c]{\rotatebox[origin=l]{90}{\parbox{
    8
        cm}{\begin{flushleft}
        Modul im Wahlpflichtbereich Mathematik (BSc, BSc21)
    \end{flushleft} }}} 
    & 
    \makecell[c]{\rotatebox[origin=l]{90}{\parbox{
    8
        cm}{\begin{flushleft}
        Wahlmodul (BSc, MSc, BSc21, 2HfB21, 2HfB)
    \end{flushleft} }}} 
    & 
    \makecell[c]{\rotatebox[origin=l]{90}{\parbox{
    8
        cm}{\begin{flushleft}
        Reine Mathematik (MSc)
    \end{flushleft} }}} 
    & 
    \makecell[c]{\rotatebox[origin=l]{90}{\parbox{
    8
        cm}{\begin{flushleft}
        Angewandte Mathematik (MSc)
    \end{flushleft} }}} 
    \\[2ex] \hline 
    \rule[0mm]{0cm}{.6cm}Mindestens zweimaliges Vorrechnen von Übungsaufgaben  im Tutorat. (SL) \rule[-3mm]{0cm}{0cm}
    &
    \makecell[c]{\xmark}
    &
    \makecell[c]{\xmark}
    &
    \makecell[c]{\xmark}
    &
    \makecell[c]{\xmark}
    &
    \makecell[c]{\xmark}
    \\
    \rule[0mm]{0cm}{.6cm}Regelmäßige Teilnahme (wie in der Prüfungsordnung definiert) an einem der Tutorate zur Vorlesung. (SL) \rule[-3mm]{0cm}{0cm}
    &
    \makecell[c]{\xmark}
    &
    \makecell[c]{\xmark}
    &
    \makecell[c]{\xmark}
    &
    \makecell[c]{\xmark}
    &
    \makecell[c]{\xmark}
    \\
    \rule[0mm]{0cm}{.6cm}Erreichen von mindestens 50% der Punkte, die insgesamt durch die Bearbeitung der für die Übung ausgegebenen Übungsaufgaben erreicht werden können. (SL) \rule[-3mm]{0cm}{0cm}
    &
    \makecell[c]{\xmark}
    &
    \makecell[c]{\xmark}
    &
    \makecell[c]{\xmark}
    &
    \makecell[c]{\xmark}
    &
    \makecell[c]{\xmark}
    \\
    \rule[0mm]{0cm}{.6cm}Mündliche Prüfung (Dauer: ca. 30 Minuten). (PL) \rule[-3mm]{0cm}{0cm}
    &
    \makecell[c]{\xmark}
    &
    &
    &
    &
    \\
    \rule[0mm]{0cm}{.6cm}Abgegebene Übungsaufgaben müssen auf Aufforderung durch den Tutor/die Tutorin hin im Tutorat präsentiert werden können. (SL) \rule[-3mm]{0cm}{0cm}
    &
    \makecell[c]{\xmark}
    &
    \makecell[c]{\xmark}
    &
    \makecell[c]{\xmark}
    &
    \makecell[c]{\xmark}
    &
    \makecell[c]{\xmark}
    \\
    \rule[0mm]{0cm}{.6cm}Klausur (Dauer: 1 bis 3 Stunden). (PL) \rule[-3mm]{0cm}{0cm}
    &
    &
    \makecell[c]{\xmark}
    &
    &
    &
    \\
    \rule[0mm]{0cm}{.6cm}Zählt bei Bedarf als eines der drei Module "Vorlesung mit Übung A" bis "Vorlesung mit Übung C" und deckt die Bedingung ab, dass mindestens eines davon zur Reinen Mathematik gehören muss. (Kommentar) \rule[-3mm]{0cm}{0cm}
    &
    &
    \makecell[c]{\xmark}
    &
    &
    &
    \\
    \rule[0mm]{0cm}{.6cm}Für das absolvierte Modul (oder ggf. den Teil des Moduls) gibt es 9 ECTS-Punkte. (Kommentar) \rule[-3mm]{0cm}{0cm}
    &
    &
    \makecell[c]{\xmark}
    &
    \makecell[c]{\xmark}
    &
    &
    \\
    \rule[0mm]{0cm}{.6cm}Zählt bei Bedarf als eines der vier Module "Vorlesung mit Übung A" bis "Vorlesung mit Übung D" und deckt die Bedingung ab, dass mindestens eines davon zur Reinen Mathematik gehören muss. (Kommentar) \rule[-3mm]{0cm}{0cm}
    &
    &
    \makecell[c]{\xmark}
    &
    &
    &
    \\
    \rule[0mm]{0cm}{.6cm}Bestehen der Abschlussklausur (Dauer 1 bis 3 Stunden). (SL) \rule[-3mm]{0cm}{0cm}
    &
    &
    &
    \makecell[c]{\xmark}
    &
    \makecell[c]{\xmark}
    &
    \makecell[c]{\xmark}
    \\
    \rule[0mm]{0cm}{.6cm}Verwendbar für die Option "Individuelle Schwerpunktgestaltung". (Kommentar) \rule[-3mm]{0cm}{0cm}
    &
    &
    &
    \makecell[c]{\xmark}
    &
    &
    \\
    \rule[0mm]{0cm}{.6cm}Mündliche Prüfung über alle Teile des Moduls (Dauer: ca. 30 Minuten, im Vertiefungsmodul ca. 45 Minuten) (PL) \rule[-3mm]{0cm}{0cm}
    &
    &
    &
    &
    \makecell[c]{\xmark}
    &
    \makecell[c]{\xmark}
    \\
\end{tabularx}

\clearpage\hrule\vskip1pt\hrule 
\section*{\Large Mathematische Logik}
\addcontentsline{toc}{subsection}{Mathematische Logik\ \textcolor{gray}{(\em Heike Mildenberger)}}
\vskip-2ex  
Heike Mildenberger, Assistenz: Hannes Jakob\\
Vorlesung: Di, Do, 10--12, HS II, \href{https://www.openstreetmap.org/?mlat=48.00233\&mlon=7.84788\#map=19/48.00233/7.84788}{Albertstr. 23b}\\
\subsubsection*{\Large Inhalt:}
Porro velit ipsum velit. Quisquam voluptatem etincidunt porro. Aliquam voluptatem neque tempora tempora est. Porro quisquam ipsum etincidunt amet. Ipsum ipsum voluptatem neque. Porro adipisci ipsum quisquam voluptatem velit dolore. Labore sit labore aliquam voluptatem. Dolorem labore consectetur dolorem ipsum tempora sit.
\subsubsection*{\Large Literatur:}
Adipisci numquam dolore etincidunt etincidunt non.
\subsubsection*{\Large Vorkenntnisse:}
Man erhält so leicht, dass $x_{1/2} = \frac{-b \pm \sqrt{b^2 - 4ac}}{2a}$
\subsubsection*{\Large Verwendbarkeit, Studien- und Prüfungsleistungen:}
\begin{tabularx}{\textwidth}{ p{.5\textwidth}
    X
    X
    X
    X
    }
    & 
    \makecell[c]{\rotatebox[origin=l]{90}{\parbox{
    8
        cm}{\begin{flushleft}
        Mathematische Vertiefung (MEd, MEH21)
    \end{flushleft} }}} 
    & 
    \makecell[c]{\rotatebox[origin=l]{90}{\parbox{
    8
        cm}{\begin{flushleft}
        Wahlmodul (BSc, MSc, BSc21, 2HfB21, 2HfB)
    \end{flushleft} }}} 
    & 
    \makecell[c]{\rotatebox[origin=l]{90}{\parbox{
    8
        cm}{\begin{flushleft}
        Reine Mathematik (MSc)
    \end{flushleft} }}} 
    & 
    \makecell[c]{\rotatebox[origin=l]{90}{\parbox{
    8
        cm}{\begin{flushleft}
        Modul im Wahlpflichtbereich Mathematik (BSc, BSc21)
    \end{flushleft} }}} 
    \\[2ex] \hline 
    \rule[0mm]{0cm}{.6cm}Regelmäßige Teilnahme (wie in der Prüfungsordnung definiert) an einem der Tutorate zur Vorlesung. (SL) \rule[-3mm]{0cm}{0cm}
    &
    \makecell[c]{\xmark}
    &
    \makecell[c]{\xmark}
    &
    \makecell[c]{\xmark}
    &
    \makecell[c]{\xmark}
    \\
    \rule[0mm]{0cm}{.6cm}Erreichen von mindestens 30% der erreichbaren Punkte in der Heimklausur. (SL) \rule[-3mm]{0cm}{0cm}
    &
    \makecell[c]{\xmark}
    &
    \makecell[c]{\xmark}
    &
    \makecell[c]{\xmark}
    &
    \makecell[c]{\xmark}
    \\
    \rule[0mm]{0cm}{.6cm}Erreichen von mindestens 50% der Punkte, die insgesamt durch die Bearbeitung der für die Übung ausgegebenen Übungsaufgaben erreicht werden können. (SL) \rule[-3mm]{0cm}{0cm}
    &
    \makecell[c]{\xmark}
    &
    \makecell[c]{\xmark}
    &
    \makecell[c]{\xmark}
    &
    \makecell[c]{\xmark}
    \\
    \rule[0mm]{0cm}{.6cm}Mündliche Prüfung (Dauer: ca. 30 Minuten). (PL) \rule[-3mm]{0cm}{0cm}
    &
    \makecell[c]{\xmark}
    &
    &
    &
    \\
    \rule[0mm]{0cm}{.6cm}Abgegebene Übungsaufgaben müssen auf Aufforderung durch den Tutor/die Tutorin hin im Tutorat präsentiert werden können. (SL) \rule[-3mm]{0cm}{0cm}
    &
    \makecell[c]{\xmark}
    &
    \makecell[c]{\xmark}
    &
    \makecell[c]{\xmark}
    &
    \makecell[c]{\xmark}
    \\
    \rule[0mm]{0cm}{.6cm}Bestehen der Abschlussklausur (Dauer 1 bis 3 Stunden). (SL) \rule[-3mm]{0cm}{0cm}
    &
    &
    \makecell[c]{\xmark}
    &
    \makecell[c]{\xmark}
    &
    \\
    \rule[0mm]{0cm}{.6cm}Für das absolvierte Modul (oder ggf. den Teil des Moduls) gibt es 9 ECTS-Punkte. (Kommentar) \rule[-3mm]{0cm}{0cm}
    &
    &
    \makecell[c]{\xmark}
    &
    &
    \makecell[c]{\xmark}
    \\
    \rule[0mm]{0cm}{.6cm}Mündliche Prüfung über alle Teile des Moduls (Dauer: ca. 30 Minuten, im Vertiefungsmodul ca. 45 Minuten) (PL) \rule[-3mm]{0cm}{0cm}
    &
    &
    &
    \makecell[c]{\xmark}
    &
    \\
    \rule[0mm]{0cm}{.6cm}Klausur (Dauer: 1 bis 3 Stunden). (PL) \rule[-3mm]{0cm}{0cm}
    &
    &
    &
    &
    \makecell[c]{\xmark}
    \\
    \rule[0mm]{0cm}{.6cm}Zählt bei Bedarf als eines der drei Module "Vorlesung mit Übung A" bis "Vorlesung mit Übung C" und deckt die Bedingung ab, dass mindestens eines davon zur Reinen Mathematik gehören muss. (Kommentar) \rule[-3mm]{0cm}{0cm}
    &
    &
    &
    &
    \makecell[c]{\xmark}
    \\
    \rule[0mm]{0cm}{.6cm}Verwendbar für die Option "Individuelle Schwerpunktgestaltung". (Kommentar) \rule[-3mm]{0cm}{0cm}
    &
    &
    \makecell[c]{\xmark}
    &
    &
    \\
    \rule[0mm]{0cm}{.6cm}Zählt bei Bedarf als eines der vier Module "Vorlesung mit Übung A" bis "Vorlesung mit Übung D" und deckt die Bedingung ab, dass mindestens eines davon zur Reinen Mathematik gehören muss. (Kommentar) \rule[-3mm]{0cm}{0cm}
    &
    &
    &
    &
    \makecell[c]{\xmark}
    \\
\end{tabularx}

\clearpage\hrule\vskip1pt\hrule 
\section*{\Large Topology}
\addcontentsline{toc}{subsection}{Topology\ \textcolor{gray}{(\em Amador Martín Pizarro)}}
\vskip-2ex  
Amador Martín Pizarro, Assistenz: Xier Ren\\
Vorlesung: Di, Do, 12--14, HS II, \href{https://www.openstreetmap.org/?mlat=48.00233\&mlon=7.84788\#map=19/48.00233/7.84788}{Albertstr. 23b}\\
\subsubsection*{\Large Inhalt:}
Dolorem ut voluptatem sed. Labore est modi non porro dolore. Voluptatem quisquam ut amet. Ipsum non modi voluptatem quisquam. Sed est consectetur consectetur modi magnam tempora. Quisquam porro ut modi porro tempora adipisci voluptatem. Adipisci numquam consectetur non. Sed numquam neque modi sed dolor.
\subsubsection*{\Large Literatur:}
Non consectetur non velit ut dolor.
\subsubsection*{\Large Vorkenntnisse:}
Man erhält so leicht, dass $x_{1/2} = \frac{-b \pm \sqrt{b^2 - 4ac}}{2a}$
\subsubsection*{\Large Verwendbarkeit, Studien- und Prüfungsleistungen:}
\begin{tabularx}{\textwidth}{ p{.5\textwidth}
    X
    X
    X
    X
    }
    & 
    \makecell[c]{\rotatebox[origin=l]{90}{\parbox{
    8
        cm}{\begin{flushleft}
        Wahlmodul (BSc, MSc, BSc21, 2HfB21, 2HfB)
    \end{flushleft} }}} 
    & 
    \makecell[c]{\rotatebox[origin=l]{90}{\parbox{
    8
        cm}{\begin{flushleft}
        Reine Mathematik (MSc)
    \end{flushleft} }}} 
    & 
    \makecell[c]{\rotatebox[origin=l]{90}{\parbox{
    8
        cm}{\begin{flushleft}
        Mathematische Vertiefung (MEd, MEH21)
    \end{flushleft} }}} 
    & 
    \makecell[c]{\rotatebox[origin=l]{90}{\parbox{
    8
        cm}{\begin{flushleft}
        Modul im Wahlpflichtbereich Mathematik (BSc, BSc21)
    \end{flushleft} }}} 
    \\[2ex] \hline 
    \rule[0mm]{0cm}{.6cm}Regelmäßige Teilnahme (wie in der Prüfungsordnung definiert) an einem der Tutorate zur Vorlesung. (SL) \rule[-3mm]{0cm}{0cm}
    &
    \makecell[c]{\xmark}
    &
    \makecell[c]{\xmark}
    &
    \makecell[c]{\xmark}
    &
    \makecell[c]{\xmark}
    \\
    \rule[0mm]{0cm}{.6cm}Mindestens zweimaliges Vorrechnen von Übungsaufgaben  im Tutorat. (SL) \rule[-3mm]{0cm}{0cm}
    &
    \makecell[c]{\xmark}
    &
    \makecell[c]{\xmark}
    &
    \makecell[c]{\xmark}
    &
    \makecell[c]{\xmark}
    \\
    \rule[0mm]{0cm}{.6cm}Abgegebene Übungsaufgaben müssen auf Aufforderung durch den Tutor/die Tutorin hin im Tutorat präsentiert werden können. (SL) \rule[-3mm]{0cm}{0cm}
    &
    \makecell[c]{\xmark}
    &
    \makecell[c]{\xmark}
    &
    \makecell[c]{\xmark}
    &
    \makecell[c]{\xmark}
    \\
    \rule[0mm]{0cm}{.6cm}Erreichen von mindestens 50% der Punkte, die insgesamt durch die Bearbeitung der für die Übung ausgegebenen Übungsaufgaben erreicht werden können. (SL) \rule[-3mm]{0cm}{0cm}
    &
    \makecell[c]{\xmark}
    &
    \makecell[c]{\xmark}
    &
    \makecell[c]{\xmark}
    &
    \makecell[c]{\xmark}
    \\
    \rule[0mm]{0cm}{.6cm}Bestehen der Abschlussklausur (Dauer 1 bis 3 Stunden). (SL) \rule[-3mm]{0cm}{0cm}
    &
    \makecell[c]{\xmark}
    &
    \makecell[c]{\xmark}
    &
    &
    \\
    \rule[0mm]{0cm}{.6cm}Für das absolvierte Modul (oder ggf. den Teil des Moduls) gibt es 9 ECTS-Punkte. (Kommentar) \rule[-3mm]{0cm}{0cm}
    &
    \makecell[c]{\xmark}
    &
    &
    &
    \makecell[c]{\xmark}
    \\
    \rule[0mm]{0cm}{.6cm}Verwendbar für die Option "Individuelle Schwerpunktgestaltung". (Kommentar) \rule[-3mm]{0cm}{0cm}
    &
    \makecell[c]{\xmark}
    &
    &
    &
    \\
    \rule[0mm]{0cm}{.6cm}Mündliche Prüfung über alle Teile des Moduls (Dauer: ca. 30 Minuten, im Vertiefungsmodul ca. 45 Minuten) (PL) \rule[-3mm]{0cm}{0cm}
    &
    &
    \makecell[c]{\xmark}
    &
    &
    \\
    \rule[0mm]{0cm}{.6cm}Mündliche Prüfung (Dauer: ca. 30 Minuten). (PL) \rule[-3mm]{0cm}{0cm}
    &
    &
    &
    \makecell[c]{\xmark}
    &
    \\
    \rule[0mm]{0cm}{.6cm}Klausur (Dauer: 1 bis 3 Stunden). (PL) \rule[-3mm]{0cm}{0cm}
    &
    &
    &
    &
    \makecell[c]{\xmark}
    \\
    \rule[0mm]{0cm}{.6cm}Zählt bei Bedarf als eines der drei Module "Vorlesung mit Übung A" bis "Vorlesung mit Übung C" und deckt die Bedingung ab, dass mindestens eines davon zur Reinen Mathematik gehören muss. (Kommentar) \rule[-3mm]{0cm}{0cm}
    &
    &
    &
    &
    \makecell[c]{\xmark}
    \\
    \rule[0mm]{0cm}{.6cm}Zählt bei Bedarf als eines der vier Module "Vorlesung mit Übung A" bis "Vorlesung mit Übung D" und deckt die Bedingung ab, dass mindestens eines davon zur Reinen Mathematik gehören muss. (Kommentar) \rule[-3mm]{0cm}{0cm}
    &
    &
    &
    &
    \makecell[c]{\xmark}
    \\
\end{tabularx}

\clearpage\hrule\vskip1pt\hrule 
\section*{\Large Wahrscheinlichkeitstheorie}
\addcontentsline{toc}{subsection}{Wahrscheinlichkeitstheorie\ \textcolor{gray}{(\em Peter Pfaffelhuber)}}
\vskip-2ex  
Peter Pfaffelhuber, Assistenz: Samuel Adeosun\\
Vorlesung: Mo, Mi, 12--14, HS II, \href{https://www.openstreetmap.org/?mlat=48.00233\&mlon=7.84788\#map=19/48.00233/7.84788}{Albertstr. 23b}\\
\subsubsection*{\Large Inhalt:}
Tempora etincidunt voluptatem dolorem sit labore numquam. Sit dolor dolor sit porro dolorem. Dolore sed magnam neque aliquam etincidunt est tempora. Dolore ipsum non est magnam magnam quisquam. Non dolorem etincidunt tempora. Quaerat sed quisquam voluptatem modi etincidunt neque.
\subsubsection*{\Large Literatur:}
Neque voluptatem est dolor modi quiquia velit.
\subsubsection*{\Large Vorkenntnisse:}
Man erhält so leicht, dass $x_{1/2} = \frac{-b \pm \sqrt{b^2 - 4ac}}{2a}$
\subsubsection*{\Large Verwendbarkeit, Studien- und Prüfungsleistungen:}
\begin{tabularx}{\textwidth}{ p{.5\textwidth}
    X
    X
    X
    X
    }
    & 
    \makecell[c]{\rotatebox[origin=l]{90}{\parbox{
    8
        cm}{\begin{flushleft}
        Modul im Wahlpflichtbereich Mathematik (BSc, BSc21)
    \end{flushleft} }}} 
    & 
    \makecell[c]{\rotatebox[origin=l]{90}{\parbox{
    8
        cm}{\begin{flushleft}
        Mathematische Vertiefung (MEd, MEH21)
    \end{flushleft} }}} 
    & 
    \makecell[c]{\rotatebox[origin=l]{90}{\parbox{
    8
        cm}{\begin{flushleft}
        Wahlmodul (BSc, MSc, BSc21, 2HfB21, 2HfB)
    \end{flushleft} }}} 
    & 
    \makecell[c]{\rotatebox[origin=l]{90}{\parbox{
    8
        cm}{\begin{flushleft}
        Angewandte Mathematik (MSc)
    \end{flushleft} }}} 
    \\[2ex] \hline 
    \rule[0mm]{0cm}{.6cm}Klausur (Dauer: 1 bis 3 Stunden). (PL) \rule[-3mm]{0cm}{0cm}
    &
    \makecell[c]{\xmark}
    &
    &
    &
    \\
    \rule[0mm]{0cm}{.6cm}Erreichen von mindestens 50% der Punkte, die insgesamt durch die Bearbeitung der für die Übung ausgegebenen Übungsaufgaben erreicht werden können. (SL) \rule[-3mm]{0cm}{0cm}
    &
    \makecell[c]{\xmark}
    &
    \makecell[c]{\xmark}
    &
    \makecell[c]{\xmark}
    &
    \makecell[c]{\xmark}
    \\
    \rule[0mm]{0cm}{.6cm}Abgegebene Übungsaufgaben müssen auf Aufforderung durch den Tutor/die Tutorin hin im Tutorat präsentiert werden können. (SL) \rule[-3mm]{0cm}{0cm}
    &
    \makecell[c]{\xmark}
    &
    \makecell[c]{\xmark}
    &
    \makecell[c]{\xmark}
    &
    \makecell[c]{\xmark}
    \\
    \rule[0mm]{0cm}{.6cm}Zählt bei Bedarf als eines der drei Module "Vorlesung mit Übung A" bis "Vorlesung mit Übung c". (Kommentar) \rule[-3mm]{0cm}{0cm}
    &
    \makecell[c]{\xmark}
    &
    &
    &
    \\
    \rule[0mm]{0cm}{.6cm}Für das absolvierte Modul (oder ggf. den Teil des Moduls) gibt es 9 ECTS-Punkte. (Kommentar) \rule[-3mm]{0cm}{0cm}
    &
    \makecell[c]{\xmark}
    &
    &
    \makecell[c]{\xmark}
    &
    \\
    \rule[0mm]{0cm}{.6cm}Mindestens zweimaliges Vorrechnen von Übungsaufgaben  im Tutorat. (SL) \rule[-3mm]{0cm}{0cm}
    &
    \makecell[c]{\xmark}
    &
    \makecell[c]{\xmark}
    &
    \makecell[c]{\xmark}
    &
    \makecell[c]{\xmark}
    \\
    \rule[0mm]{0cm}{.6cm}Regelmäßige Teilnahme (wie in der Prüfungsordnung definiert) an einem der Tutorate zur Vorlesung. (SL) \rule[-3mm]{0cm}{0cm}
    &
    \makecell[c]{\xmark}
    &
    \makecell[c]{\xmark}
    &
    \makecell[c]{\xmark}
    &
    \makecell[c]{\xmark}
    \\
    \rule[0mm]{0cm}{.6cm}Mündliche Prüfung (Dauer: ca. 30 Minuten). (PL) \rule[-3mm]{0cm}{0cm}
    &
    &
    \makecell[c]{\xmark}
    &
    &
    \\
    \rule[0mm]{0cm}{.6cm}Verwendbar für die Option "Individuelle Schwerpunktgestaltung". (Kommentar) \rule[-3mm]{0cm}{0cm}
    &
    &
    &
    \makecell[c]{\xmark}
    &
    \\
    \rule[0mm]{0cm}{.6cm}Bestehen der Abschlussklausur (Dauer 1 bis 3 Stunden). (SL) \rule[-3mm]{0cm}{0cm}
    &
    &
    &
    \makecell[c]{\xmark}
    &
    \makecell[c]{\xmark}
    \\
    \rule[0mm]{0cm}{.6cm}Zählt bei Bedarf als eines der vier Module "Vorlesung mit Übung A" bis "Vorlesung mit Übung D". (Kommentar) \rule[-3mm]{0cm}{0cm}
    &
    \makecell[c]{\xmark}
    &
    &
    &
    \\
    \rule[0mm]{0cm}{.6cm}Mündliche Prüfung über alle Teile des Moduls (Dauer: ca. 30 Minuten, im Vertiefungsmodul ca. 45 Minuten) (PL) \rule[-3mm]{0cm}{0cm}
    &
    &
    &
    &
    \makecell[c]{\xmark}
    \\
\end{tabularx}

\clearpage\hrule\vskip1pt\hrule 
\section*{\Large Kurven und Flächen}
\addcontentsline{toc}{subsection}{Kurven und Flächen\ \textcolor{gray}{(\em Christian Ketterer)}}
\vskip-2ex  
Christian Ketterer, Assistenz: Thorsten Schmidt\\
Vorlesung: Mo, Mi, 10--12, HS II, \href{https://www.openstreetmap.org/?mlat=48.00233\&mlon=7.84788\#map=19/48.00233/7.84788}{Albertstr. 23b}\\
\subsubsection*{\Large Inhalt:}
Dolor velit est numquam porro. Numquam sit ipsum neque. Est est voluptatem ipsum eius quiquia dolorem. Ut ut numquam voluptatem sed porro eius neque. Tempora ut adipisci eius. Neque aliquam adipisci adipisci neque neque ut.
\subsubsection*{\Large Literatur:}
Amet amet consectetur eius labore.
\subsubsection*{\Large Vorkenntnisse:}
Man erhält so leicht, dass $x_{1/2} = \frac{-b \pm \sqrt{b^2 - 4ac}}{2a}$
\subsubsection*{\Large Verwendbarkeit, Studien- und Prüfungsleistungen:}
\begin{tabularx}{\textwidth}{ p{.5\textwidth}
    X
    X
    X
    X
    }
    & 
    \makecell[c]{\rotatebox[origin=l]{90}{\parbox{
    8
        cm}{\begin{flushleft}
        Mathematische Vertiefung (MEd, MEH21)
    \end{flushleft} }}} 
    & 
    \makecell[c]{\rotatebox[origin=l]{90}{\parbox{
    8
        cm}{\begin{flushleft}
        Wahlmodul (BSc, MSc, BSc21, 2HfB21, 2HfB)
    \end{flushleft} }}} 
    & 
    \makecell[c]{\rotatebox[origin=l]{90}{\parbox{
    8
        cm}{\begin{flushleft}
        Reine Mathematik (MSc)
    \end{flushleft} }}} 
    & 
    \makecell[c]{\rotatebox[origin=l]{90}{\parbox{
    8
        cm}{\begin{flushleft}
        Modul im Wahlpflichtbereich Mathematik (BSc, BSc21)
    \end{flushleft} }}} 
    \\[2ex] \hline 
    \rule[0mm]{0cm}{.6cm}Erreichen von mindestens 50% der Punkte, die insgesamt durch die Bearbeitung der für die Übung ausgegebenen Übungsaufgaben erreicht werden können. (SL) \rule[-3mm]{0cm}{0cm}
    &
    \makecell[c]{\xmark}
    &
    \makecell[c]{\xmark}
    &
    \makecell[c]{\xmark}
    &
    \makecell[c]{\xmark}
    \\
    \rule[0mm]{0cm}{.6cm}Abgegebene Übungsaufgaben müssen auf Aufforderung durch den Tutor/die Tutorin hin im Tutorat präsentiert werden können. (SL) \rule[-3mm]{0cm}{0cm}
    &
    \makecell[c]{\xmark}
    &
    \makecell[c]{\xmark}
    &
    \makecell[c]{\xmark}
    &
    \makecell[c]{\xmark}
    \\
    \rule[0mm]{0cm}{.6cm}Mindestens zweimaliges Vorrechnen von Übungsaufgaben  im Tutorat. (SL) \rule[-3mm]{0cm}{0cm}
    &
    \makecell[c]{\xmark}
    &
    \makecell[c]{\xmark}
    &
    \makecell[c]{\xmark}
    &
    \makecell[c]{\xmark}
    \\
    \rule[0mm]{0cm}{.6cm}Regelmäßige Teilnahme (wie in der Prüfungsordnung definiert) an einem der Tutorate zur Vorlesung. (SL) \rule[-3mm]{0cm}{0cm}
    &
    \makecell[c]{\xmark}
    &
    \makecell[c]{\xmark}
    &
    \makecell[c]{\xmark}
    &
    \makecell[c]{\xmark}
    \\
    \rule[0mm]{0cm}{.6cm}Mündliche Prüfung (Dauer: ca. 30 Minuten). (PL) \rule[-3mm]{0cm}{0cm}
    &
    \makecell[c]{\xmark}
    &
    &
    &
    \\
    \rule[0mm]{0cm}{.6cm}Verwendbar für die Option "Individuelle Schwerpunktgestaltung". (Kommentar) \rule[-3mm]{0cm}{0cm}
    &
    &
    \makecell[c]{\xmark}
    &
    &
    \\
    \rule[0mm]{0cm}{.6cm}Bestehen der Abschlussklausur (Dauer 1 bis 3 Stunden). (SL) \rule[-3mm]{0cm}{0cm}
    &
    &
    \makecell[c]{\xmark}
    &
    \makecell[c]{\xmark}
    &
    \\
    \rule[0mm]{0cm}{.6cm}Für das absolvierte Modul (oder ggf. den Teil des Moduls) gibt es 9 ECTS-Punkte. (Kommentar) \rule[-3mm]{0cm}{0cm}
    &
    &
    \makecell[c]{\xmark}
    &
    &
    \makecell[c]{\xmark}
    \\
    \rule[0mm]{0cm}{.6cm}Mündliche Prüfung über alle Teile des Moduls (Dauer: ca. 30 Minuten, im Vertiefungsmodul ca. 45 Minuten) (PL) \rule[-3mm]{0cm}{0cm}
    &
    &
    &
    \makecell[c]{\xmark}
    &
    \\
    \rule[0mm]{0cm}{.6cm}Klausur (Dauer: 1 bis 3 Stunden). (PL) \rule[-3mm]{0cm}{0cm}
    &
    &
    &
    &
    \makecell[c]{\xmark}
    \\
    \rule[0mm]{0cm}{.6cm}Zählt bei Bedarf als eines der drei Module "Vorlesung mit Übung A" bis "Vorlesung mit Übung C" und deckt die Bedingung ab, dass mindestens eines davon zur Reinen Mathematik gehören muss. (Kommentar) \rule[-3mm]{0cm}{0cm}
    &
    &
    &
    &
    \makecell[c]{\xmark}
    \\
    \rule[0mm]{0cm}{.6cm}Zählt bei Bedarf als eines der vier Module "Vorlesung mit Übung A" bis "Vorlesung mit Übung D" und deckt die Bedingung ab, dass mindestens eines davon zur Reinen Mathematik gehören muss. (Kommentar) \rule[-3mm]{0cm}{0cm}
    &
    &
    &
    &
    \makecell[c]{\xmark}
    \\
\end{tabularx}

\clearpage\hrule\vskip1pt\hrule 
\section*{\Large Differentialgeometrie II – Vektorbündel}
\addcontentsline{toc}{subsection}{Differentialgeometrie II – Vektorbündel\ \textcolor{gray}{(\em Nadine Große)}}
\vskip-2ex  
Nadine Große, Assistenz: Jonah Reuß\\
Vorlesung: Di, Do, 10--12, SR 404, \href{https://www.openstreetmap.org/?mlat=48.00065\&mlon=7.84591\#map=19/48.00065/7.84591}{Ernst-Zermelo-Straße 1}\\
\subsubsection*{\Large Inhalt:}
Quaerat ut labore non. Neque voluptatem tempora etincidunt est velit sed. Ipsum quiquia eius velit sed dolor. Dolorem adipisci tempora numquam quiquia voluptatem sed ipsum. Voluptatem numquam tempora quaerat dolorem. Dolore quisquam adipisci dolor. Etincidunt magnam ipsum velit. Quisquam numquam porro quaerat dolorem velit magnam dolore. Velit aliquam labore ut tempora quisquam. Dolore porro aliquam amet etincidunt ut.
\subsubsection*{\Large Literatur:}
Quisquam velit velit sed sit sit.
\subsubsection*{\Large Vorkenntnisse:}
Man erhält so leicht, dass $x_{1/2} = \frac{-b \pm \sqrt{b^2 - 4ac}}{2a}$
\subsubsection*{\Large Verwendbarkeit, Studien- und Prüfungsleistungen:}
\begin{tabularx}{\textwidth}{ p{.5\textwidth}
    }
    \\[2ex] \hline 
\end{tabularx}

\clearpage\hrule\vskip1pt\hrule 
\section*{\Large Maschinelles Lernen aus Stochastischer Sicht}
\addcontentsline{toc}{subsection}{Maschinelles Lernen aus Stochastischer Sicht\ \textcolor{gray}{(\em Thorsten Schmidt)}}
\vskip-2ex  
Thorsten Schmidt, Assistenz: Moritz Ritter\\
Vorlesung: Fr, 10--12, Do, 12--14, SR 226, \href{https://www.openstreetmap.org/?mlat=48.00351\&mlon=7.84815\#map=19/48.00351/7.84815}{Hermann-Herder-Str. 10}\\
\subsubsection*{\Large Inhalt:}
Consectetur dolor consectetur eius. Consectetur velit neque quisquam ut dolorem dolor modi. Dolore modi ipsum aliquam velit quaerat quisquam. Voluptatem modi porro aliquam neque numquam consectetur. Etincidunt quaerat ipsum velit ipsum. Voluptatem consectetur quisquam neque voluptatem. Dolor adipisci consectetur velit amet modi.
\subsubsection*{\Large Literatur:}
Amet sit consectetur sed.
\subsubsection*{\Large Vorkenntnisse:}
Man erhält so leicht, dass $x_{1/2} = \frac{-b \pm \sqrt{b^2 - 4ac}}{2a}$
\subsubsection*{\Large Verwendbarkeit, Studien- und Prüfungsleistungen:}
\begin{tabularx}{\textwidth}{ p{.5\textwidth}
    }
    \\[2ex] \hline 
\end{tabularx}

\clearpage\hrule\vskip1pt\hrule 
\section*{\Large Algebraische Topologie II}
\addcontentsline{toc}{subsection}{Algebraische Topologie II\ \textcolor{gray}{(\em Sebastian Goette)}}
\vskip-2ex  
Sebastian Goette, Assistenz: Jonas Schnitzer\\
Raum und Zeit: Di, 14--16, Fr, 10--12, SR 125, \href{https://www.openstreetmap.org/?mlat=48.00065\&mlon=7.84591\#map=19/48.00065/7.84591}{Ernst-Zermelo-Straße 1}\\
\subsubsection*{\Large Inhalt:}
Dolor adipisci tempora etincidunt modi dolor magnam dolor. Adipisci etincidunt dolorem magnam adipisci numquam ipsum. Eius non voluptatem ipsum quisquam adipisci. Quaerat dolorem ipsum eius amet dolore quisquam quaerat. Ut magnam adipisci quaerat.
\subsubsection*{\Large Literatur:}
Ut eius quiquia porro tempora neque.
\subsubsection*{\Large Vorkenntnisse:}
Man erhält so leicht, dass $x_{1/2} = \frac{-b \pm \sqrt{b^2 - 4ac}}{2a}$
\subsubsection*{\Large Verwendbarkeit, Studien- und Prüfungsleistungen:}
\begin{tabularx}{\textwidth}{ p{.5\textwidth}
    }
    \\[2ex] \hline 
\end{tabularx}

\clearpage
\phantomsection
\thispagestyle{empty}
\vspace*{\fill}
\begin{center}
\Huge\bfseries 1c. Weiterführende zweistündige Vorlesungen
\end{center}
\addcontentsline{toc}{section}{\textbf{1c. Weiterführende zweistündige Vorlesungen}}
\addtocontents{toc}{\medskip\hrule\medskip}\vspace*{\fill}\vspace*{\fill}\clearpage
\vfill
\thispagestyle{empty}
\clearpage

\clearpage\hrule\vskip1pt\hrule 
\section*{\Large Mathematische Modellierung}
\addcontentsline{toc}{subsection}{Mathematische Modellierung\ \textcolor{gray}{(\em Michael Růžička)}}
\vskip-2ex  
Michael Růžička, Assistenz: Sören Andres\\
Vorlesung: Mo, 10--12, SR 404, \href{https://www.openstreetmap.org/?mlat=48.00065\&mlon=7.84591\#map=19/48.00065/7.84591}{Ernst-Zermelo-Straße 1}\\
\subsubsection*{\Large Inhalt:}
Sit neque velit magnam. Adipisci quisquam quisquam numquam numquam. Eius modi magnam ipsum dolor est voluptatem. Quiquia dolorem tempora modi sit ut. Modi porro modi amet amet dolorem. Labore neque dolorem voluptatem.
\subsubsection*{\Large Literatur:}
Velit non adipisci dolor quiquia dolor consectetur.
\subsubsection*{\Large Vorkenntnisse:}
Man erhält so leicht, dass $x_{1/2} = \frac{-b \pm \sqrt{b^2 - 4ac}}{2a}$
\subsubsection*{\Large Verwendbarkeit, Studien- und Prüfungsleistungen:}
\begin{tabularx}{\textwidth}{ p{.5\textwidth}
    X
    X
    X
    X
    }
    & 
    \makecell[c]{\rotatebox[origin=l]{90}{\parbox{
    8
        cm}{\begin{flushleft}
        Modul im Wahlpflichtbereich Mathematik (BSc, BSc21)
    \end{flushleft} }}} 
    & 
    \makecell[c]{\rotatebox[origin=l]{90}{\parbox{
    8
        cm}{\begin{flushleft}
        Mathematische Ergänzung (MEd)
    \end{flushleft} }}} 
    & 
    \makecell[c]{\rotatebox[origin=l]{90}{\parbox{
    8
        cm}{\begin{flushleft}
        Wahlmodul (BSc, MSc, BSc21, 2HfB21, 2HfB)
    \end{flushleft} }}} 
    & 
    \makecell[c]{\rotatebox[origin=l]{90}{\parbox{
    8
        cm}{\begin{flushleft}
        Teil des Moduls "Angewandte Mathematik", "Mathematik" oder des Vertiefungsmoduls (MSc)
    \end{flushleft} }}} 
    \\[2ex] \hline 
    \rule[0mm]{0cm}{.6cm}Mündliche Prüfung (Dauer: ca. 30 Minuten). (PL) \rule[-3mm]{0cm}{0cm}
    &
    \makecell[c]{\xmark}
    &
    &
    &
    \\
    \rule[0mm]{0cm}{.6cm}Für das absolvierte Modul (oder ggf. den Teil des Moduls) gibt es 6 ECTS-Punkte. (Kommentar) \rule[-3mm]{0cm}{0cm}
    &
    \makecell[c]{\xmark}
    &
    &
    \makecell[c]{\xmark}
    &
    \\
    \rule[0mm]{0cm}{.6cm}Erreichen von mindestens 50% der Punkte, die insgesamt durch die Bearbeitung der für die Übung ausgegebenen Übungsaufgaben erreicht werden können. (SL) \rule[-3mm]{0cm}{0cm}
    &
    \makecell[c]{\xmark}
    &
    \makecell[c]{\xmark}
    &
    \makecell[c]{\xmark}
    &
    \makecell[c]{\xmark}
    \\
    \rule[0mm]{0cm}{.6cm}Mindestens zweimaliges Vorrechnen von Übungsaufgaben  im Tutorat. (SL) \rule[-3mm]{0cm}{0cm}
    &
    \makecell[c]{\xmark}
    &
    \makecell[c]{\xmark}
    &
    \makecell[c]{\xmark}
    &
    \makecell[c]{\xmark}
    \\
    \rule[0mm]{0cm}{.6cm}Regelmäßige Teilnahme (wie in der Prüfungsordnung definiert) an einem der Tutorate zur Vorlesung. (SL) \rule[-3mm]{0cm}{0cm}
    &
    \makecell[c]{\xmark}
    &
    \makecell[c]{\xmark}
    &
    \makecell[c]{\xmark}
    &
    \makecell[c]{\xmark}
    \\
    \rule[0mm]{0cm}{.6cm}Abgegebene Übungsaufgaben müssen auf Aufforderung durch den Tutor/die Tutorin hin im Tutorat präsentiert werden können. (SL) \rule[-3mm]{0cm}{0cm}
    &
    \makecell[c]{\xmark}
    &
    \makecell[c]{\xmark}
    &
    \makecell[c]{\xmark}
    &
    \makecell[c]{\xmark}
    \\
    \rule[0mm]{0cm}{.6cm}Bestehen eines mündlichen Abschlusstests. (SL) \rule[-3mm]{0cm}{0cm}
    &
    &
    \makecell[c]{\xmark}
    &
    \makecell[c]{\xmark}
    &
    \\
    \rule[0mm]{0cm}{.6cm}Verwendbar für die Option "Individuelle Schwerpunktgestaltung". (Kommentar) \rule[-3mm]{0cm}{0cm}
    &
    &
    &
    \makecell[c]{\xmark}
    &
    \\
    \rule[0mm]{0cm}{.6cm}Mündliche Prüfung über alle Teile des Moduls (Dauer: ca. 30 Minuten, im Vertiefungsmodul ca. 45 Minuten) (PL) \rule[-3mm]{0cm}{0cm}
    &
    &
    &
    &
    \makecell[c]{\xmark}
    \\
    \rule[0mm]{0cm}{.6cm}Die absolvierte Studienleistung zählt mit 4,5 ECTS-Punkten in das Gesamtmodul. (Kommentar) \rule[-3mm]{0cm}{0cm}
    &
    &
    &
    &
    \makecell[c]{\xmark}
    \\
    \rule[0mm]{0cm}{.6cm}Die Zusammensetzung des Moduls muss mit dem Prüfer/der Prüferin zuvor abgesprochen sein. Nicht alle Kombinationen sind zulässig. (Kommentar) \rule[-3mm]{0cm}{0cm}
    &
    &
    &
    &
    \makecell[c]{\xmark}
    \\
\end{tabularx}

\clearpage\hrule\vskip1pt\hrule 
\section*{\Large Gewöhnliche Differentialgleichungen}
\addcontentsline{toc}{subsection}{Gewöhnliche Differentialgleichungen\ \textcolor{gray}{(\em Florian Johne)}}
\vskip-2ex  
Florian Johne\\
Vorlesung: Do, 14--16, SR 226, \href{https://www.openstreetmap.org/?mlat=48.00351\&mlon=7.84815\#map=19/48.00351/7.84815}{Hermann-Herder-Str. 10}\\
\subsubsection*{\Large Inhalt:}
Quisquam non voluptatem dolorem est. Aliquam consectetur aliquam velit quiquia labore quiquia porro. Labore ut modi modi non dolor. Etincidunt quisquam velit quiquia porro amet. Adipisci voluptatem dolor magnam neque. Aliquam dolorem numquam voluptatem quisquam. Ipsum numquam amet sed. Non quiquia voluptatem aliquam numquam voluptatem est. Quaerat consectetur quiquia dolorem non. Adipisci dolore quaerat amet adipisci ipsum non aliquam.
\subsubsection*{\Large Literatur:}
Amet quisquam quisquam dolorem modi dolore amet.
\subsubsection*{\Large Vorkenntnisse:}
Man erhält so leicht, dass $x_{1/2} = \frac{-b \pm \sqrt{b^2 - 4ac}}{2a}$
\subsubsection*{\Large Verwendbarkeit, Studien- und Prüfungsleistungen:}
\begin{tabularx}{\textwidth}{ p{.5\textwidth}
    }
    \\[2ex] \hline 
\end{tabularx}

\clearpage\hrule\vskip1pt\hrule 
\section*{\Large Mathematical Introduction to Deep Learning}
\addcontentsline{toc}{subsection}{Mathematical Introduction to Deep Learning\ \textcolor{gray}{(\em Diyora Salimova)}}
\vskip-2ex  
Diyora Salimova, Assistenz: Hedwig Keller\\
Raum und Zeit: Di, 12--14, SR 226, \href{https://www.openstreetmap.org/?mlat=48.00351\&mlon=7.84815\#map=19/48.00351/7.84815}{Hermann-Herder-Str. 10}\\
\subsubsection*{\Large Inhalt:}
Labore modi consectetur sed magnam ut. Consectetur ipsum porro tempora dolore velit. Quiquia amet velit eius sit sit velit neque. Amet est ut neque numquam eius eius. Eius amet neque ipsum quiquia. Quiquia neque est tempora porro labore quisquam. Quaerat sed quaerat aliquam sit sit ut dolor. Modi labore ipsum neque quiquia quaerat tempora amet.
\subsubsection*{\Large Literatur:}
Etincidunt ut dolore adipisci.
\subsubsection*{\Large Vorkenntnisse:}
Man erhält so leicht, dass $x_{1/2} = \frac{-b \pm \sqrt{b^2 - 4ac}}{2a}$
\subsubsection*{\Large Verwendbarkeit, Studien- und Prüfungsleistungen:}
\begin{tabularx}{\textwidth}{ p{.5\textwidth}
    }
    \\[2ex] \hline 
\end{tabularx}

\clearpage\hrule\vskip1pt\hrule 
\section*{\Large Applications of Set Theory in Algebra and in Topology}
\addcontentsline{toc}{subsection}{Applications of Set Theory in Algebra and in Topology\ \textcolor{gray}{(\em Maxwell Levine)}}
\vskip-2ex  
Maxwell Levine\\
Vorlesung: Mi, 12--14, SR 127, \href{https://www.openstreetmap.org/?mlat=48.00065\&mlon=7.84591\#map=19/48.00065/7.84591}{Ernst-Zermelo-Straße 1}\\
\subsubsection*{\Large Inhalt:}
Numquam eius magnam non consectetur modi. Porro sed tempora amet velit quisquam. Dolorem dolorem dolor adipisci velit etincidunt magnam ipsum. Tempora aliquam voluptatem quisquam sit quisquam adipisci aliquam. Dolor dolor velit amet ipsum. Sit neque sed quiquia voluptatem porro consectetur non. Consectetur dolor modi modi neque porro voluptatem. Velit numquam amet numquam modi neque. Quisquam consectetur porro amet.
\subsubsection*{\Large Literatur:}
Labore quisquam dolore quiquia quiquia sit.
\subsubsection*{\Large Vorkenntnisse:}
Man erhält so leicht, dass $x_{1/2} = \frac{-b \pm \sqrt{b^2 - 4ac}}{2a}$
\subsubsection*{\Large Verwendbarkeit, Studien- und Prüfungsleistungen:}
\begin{tabularx}{\textwidth}{ p{.5\textwidth}
    }
    \\[2ex] \hline 
\end{tabularx}

\clearpage\hrule\vskip1pt\hrule 
\section*{\Large Geometrische Variationsprobleme}
\addcontentsline{toc}{subsection}{Geometrische Variationsprobleme\ \textcolor{gray}{(\em Ernst Kuwert)}}
\vskip-2ex  
Ernst Kuwert, Assistenz: Xinqun Mei\\
Vorlesung: Di, 14--16, SR 127, \href{https://www.openstreetmap.org/?mlat=48.00065\&mlon=7.84591\#map=19/48.00065/7.84591}{Ernst-Zermelo-Straße 1}\\
\subsubsection*{\Large Inhalt:}
Adipisci consectetur quiquia non tempora eius sit. Est eius aliquam magnam. Ipsum non eius numquam est porro. Dolor sit aliquam labore quaerat quaerat etincidunt numquam. Dolore est numquam est non.
\subsubsection*{\Large Literatur:}
Amet porro adipisci amet.
\subsubsection*{\Large Vorkenntnisse:}
Man erhält so leicht, dass $x_{1/2} = \frac{-b \pm \sqrt{b^2 - 4ac}}{2a}$
\subsubsection*{\Large Verwendbarkeit, Studien- und Prüfungsleistungen:}
\begin{tabularx}{\textwidth}{ p{.5\textwidth}
    }
    \\[2ex] \hline 
\end{tabularx}

\clearpage
\phantomsection
\thispagestyle{empty}
\vspace*{\fill}
\begin{center}
\Huge\bfseries 1d. Lehrexportveranstaltungen
\end{center}
\addcontentsline{toc}{section}{\textbf{1d. Lehrexportveranstaltungen}}
\addtocontents{toc}{\medskip\hrule\medskip}\vspace*{\fill}\vspace*{\fill}\clearpage
\vfill
\thispagestyle{empty}
\clearpage

\clearpage\hrule\vskip1pt\hrule 
\section*{\Large Mathematik II für Studierende der Informatik}
\addcontentsline{toc}{subsection}{Mathematik II für Studierende der Informatik\ \textcolor{gray}{(\em Ernst August v. Hammerstein)}}
\vskip-2ex  
Ernst August v. Hammerstein, Assistenz: Saskia Glaffig\\
Vorlesung (4-stündig): Mo, Mi, 10--12, HS 00-026, \href{https://www.openstreetmap.org/?mlat=48.01251&mlon=7.83492#map=19/48.01251/7.83492}{Geb. 101, Georges-Köhler-Allee}\\
\subsubsection*{\Large Inhalt:}
Non etincidunt magnam quiquia dolorem. Eius adipisci numquam sed. Sit quaerat ipsum consectetur ipsum etincidunt. Dolore non quisquam velit numquam velit amet. Magnam amet magnam voluptatem voluptatem. Modi dolorem ipsum adipisci magnam porro. Velit voluptatem voluptatem dolor aliquam. Etincidunt est dolore porro voluptatem adipisci. Numquam ut quiquia sit porro ut ipsum quaerat.
\subsubsection*{\Large Literatur:}
Amet aliquam ut amet sit eius consectetur dolorem.
\subsubsection*{\Large Vorkenntnisse:}
Man erhält so leicht, dass $x_{1/2} = \frac{-b \pm \sqrt{b^2 - 4ac}}{2a}$
\subsubsection*{\Large Verwendbarkeit, Studien- und Prüfungsleistungen:}
\begin{tabularx}{\textwidth}{ p{.5\textwidth}
    }
    \\[2ex] \hline 
\end{tabularx}

\clearpage\hrule\vskip1pt\hrule 
\section*{\Large Mathematik II für Studierende der Ingenieurwissenschaften}
\addcontentsline{toc}{subsection}{Mathematik II für Studierende der Ingenieurwissenschaften\ \textcolor{gray}{(\em Sebastian Goette)}}
\vskip-2ex  
Sebastian Goette, Assistenz: Vera Jackisch\\
Vorlesung (4-stündig): Mo, 12--14, Fr, 14--16, HS Rundbau, \href{https://www.openstreetmap.org/?mlat=48.00156\&mlon=7.84931\#map=19/48.00156/7.84931}{Albertstr. 21}\\
\subsubsection*{\Large Inhalt:}
Velit dolore non non aliquam quisquam ipsum modi. Non consectetur non magnam ut tempora eius. Velit ipsum sit labore. Consectetur est labore labore. Est quisquam ut voluptatem ut. Modi tempora etincidunt quiquia est. Dolorem consectetur non quaerat adipisci non modi consectetur. Ipsum dolore sit ipsum quiquia. Etincidunt quiquia eius sit amet consectetur amet numquam.
\subsubsection*{\Large Literatur:}
Sed non neque quaerat.
\subsubsection*{\Large Vorkenntnisse:}
Man erhält so leicht, dass $x_{1/2} = \frac{-b \pm \sqrt{b^2 - 4ac}}{2a}$
\subsubsection*{\Large Verwendbarkeit, Studien- und Prüfungsleistungen:}
\begin{tabularx}{\textwidth}{ p{.5\textwidth}
    }
    \\[2ex] \hline 
\end{tabularx}

\clearpage\hrule\vskip1pt\hrule 
\section*{\Large Mathematik II für Studierende der Naturwissenschaften}
\addcontentsline{toc}{subsection}{Mathematik II für Studierende der Naturwissenschaften\ \textcolor{gray}{(\em Susanne Knies)}}
\vskip-2ex  
Susanne Knies\\
Vorlesung: Do, Di, 10--12, HS Rundbau, \href{https://www.openstreetmap.org/?mlat=48.00156\&mlon=7.84931\#map=19/48.00156/7.84931}{Albertstr. 21}\\
\subsubsection*{\Large Inhalt:}
Quisquam dolorem numquam quiquia adipisci magnam dolor. Labore non neque amet amet etincidunt numquam. Aliquam sit ipsum amet quisquam velit. Etincidunt neque neque quiquia ut neque. Neque consectetur voluptatem ipsum sed adipisci labore. Non tempora dolore sed ut consectetur quiquia. Tempora quiquia etincidunt tempora quiquia est sed. Labore quiquia porro labore adipisci.
\subsubsection*{\Large Literatur:}
Etincidunt ut neque consectetur neque.
\subsubsection*{\Large Vorkenntnisse:}
Man erhält so leicht, dass $x_{1/2} = \frac{-b \pm \sqrt{b^2 - 4ac}}{2a}$
\subsubsection*{\Large Verwendbarkeit, Studien- und Prüfungsleistungen:}
\begin{tabularx}{\textwidth}{ p{.5\textwidth}
    }
    \\[2ex] \hline 
\end{tabularx}

\clearpage\hrule\vskip1pt\hrule 
\section*{\Large Stochastik für Studierende der Informatik}
\addcontentsline{toc}{subsection}{Stochastik für Studierende der Informatik\ \textcolor{gray}{(\em David Criens)}}
\vskip-2ex  
David Criens, Assistenz: Timo Enger\\
Vorlesung (2-stündig): Mo, 10--12, HS 00-036, \href{https://www.openstreetmap.org/?mlat=48.01251&mlon=7.83492#map=19/48.01251/7.83492}{Geb. 101, Georges-Köhler-Allee}\\
\subsubsection*{\Large Inhalt:}
Labore est adipisci eius modi. Velit labore quaerat dolore quaerat dolor eius. Magnam labore ut etincidunt. Quisquam quaerat quaerat etincidunt. Quisquam numquam sed eius est magnam.
\subsubsection*{\Large Literatur:}
Aliquam non neque sit dolor etincidunt.
\subsubsection*{\Large Vorkenntnisse:}
Man erhält so leicht, dass $x_{1/2} = \frac{-b \pm \sqrt{b^2 - 4ac}}{2a}$
\subsubsection*{\Large Verwendbarkeit, Studien- und Prüfungsleistungen:}
\begin{tabularx}{\textwidth}{ p{.5\textwidth}
    }
    \\[2ex] \hline 
\end{tabularx}

\clearpage
\phantomsection
\thispagestyle{empty}
\vspace*{\fill}
\begin{center}
\Huge\bfseries 2a. Begleitveranstaltungen
\end{center}
\addcontentsline{toc}{section}{\textbf{2a. Begleitveranstaltungen}}
\addtocontents{toc}{\medskip\hrule\medskip}\vspace*{\fill}\vspace*{\fill}\clearpage
\vfill
\thispagestyle{empty}
\clearpage

\clearpage\hrule\vskip1pt\hrule 
\section*{\Large Lernen durch Lehren}
\addcontentsline{toc}{subsection}{Lernen durch Lehren\ \textcolor{gray}{(\em Susanne Knies)}}
\vskip-2ex  
Susanne Knies\\
\subsubsection*{\Large Inhalt:}
Sed tempora labore ipsum. Velit quiquia quisquam sit. Amet quisquam voluptatem numquam. Dolor neque quaerat aliquam non adipisci dolorem. Ut aliquam sed non. Sed velit quaerat eius aliquam dolor. Modi quiquia amet numquam tempora quaerat quisquam.
\subsubsection*{\Large Literatur:}
Voluptatem neque ut labore aliquam sed.
\subsubsection*{\Large Vorkenntnisse:}
Man erhält so leicht, dass $x_{1/2} = \frac{-b \pm \sqrt{b^2 - 4ac}}{2a}$
\subsubsection*{\Large Verwendbarkeit, Studien- und Prüfungsleistungen:}
\begin{tabularx}{\textwidth}{ p{.5\textwidth}
    X
    }
    & 
    \makecell[c]{\rotatebox[origin=l]{90}{\parbox{
    4
        cm}{\begin{flushleft}
        Wahlmodul (BSc, MSc, BSc21, 2HfB21, 2HfB)
    \end{flushleft} }}} 
    \\[2ex] \hline 
    \rule[0mm]{0cm}{.6cm}Voraussetzung für die Teilnahme ist eine Tutoratsstelle zu einer Vorlesung des Mathematischen Instituts im laufenden Semester (mindestens eine zweistündige oder zwei einstündige Übungsgruppen über das ganze Semester). (Kommentar) \rule[-3mm]{0cm}{0cm}
    &
    \makecell[c]{\xmark}
    \\
    \rule[0mm]{0cm}{.6cm}Teilnahme an beiden Terminen des Tutoratsworkshops. 
Regelmäßige Teilnahme an der Tutorenbesprechung;
Zwei gegenseitige Tutoratsbesuche mit einem (oder mehreren) anderen Modulteilnehmern. (SL) \rule[-3mm]{0cm}{0cm}
    &
    \makecell[c]{\xmark}
    \\
    \rule[0mm]{0cm}{.6cm}Für das absolvierte Modul (oder ggf. den Teil des Moduls) gibt es 3 ECTS-Punkte. (Kommentar) \rule[-3mm]{0cm}{0cm}
    &
    \makecell[c]{\xmark}
    \\
    \rule[0mm]{0cm}{.6cm}Verwendbar für die Option "Individuelle Schwerpunktgestaltung". (Kommentar) \rule[-3mm]{0cm}{0cm}
    &
    \makecell[c]{\xmark}
    \\
    \rule[0mm]{0cm}{.6cm}Das Modul kann im M.Sc.-Studiengang zweimal absolviert werden (in verschiedenen Semestern, aber u.U. in Tutoraten zur gleichen Vorlesung). (Kommentar) \rule[-3mm]{0cm}{0cm}
    &
    \makecell[c]{\xmark}
    \\
\end{tabularx}

\clearpage\hrule\vskip1pt\hrule 
\section*{\Large Betreutes Rechnen}
\addcontentsline{toc}{subsection}{Betreutes Rechnen\ \textcolor{gray}{(\em Organisation Fachschaft)}}
\vskip-2ex  
Organisation Fachschaft\\
\subsubsection*{\Large Inhalt:}
Adipisci consectetur modi est aliquam. Modi quaerat modi sit consectetur amet. Eius dolorem etincidunt consectetur. Non quaerat dolor aliquam dolorem. Sit quaerat est ut modi sit. Etincidunt dolorem tempora labore est etincidunt aliquam. Eius modi consectetur consectetur dolor modi est. Sed neque magnam aliquam non porro non ut.
\subsubsection*{\Large Literatur:}
Adipisci sed porro neque neque.
\subsubsection*{\Large Vorkenntnisse:}
Man erhält so leicht, dass $x_{1/2} = \frac{-b \pm \sqrt{b^2 - 4ac}}{2a}$
\subsubsection*{\Large Verwendbarkeit, Studien- und Prüfungsleistungen:}
\begin{tabularx}{\textwidth}{ p{.5\textwidth}
    }
    \\[2ex] \hline 
\end{tabularx}

\clearpage
\phantomsection
\thispagestyle{empty}
\vspace*{\fill}
\begin{center}
\Huge\bfseries 2b. Fachdidaktik
\end{center}
\addcontentsline{toc}{section}{\textbf{2b. Fachdidaktik}}
\addtocontents{toc}{\medskip\hrule\medskip}\vspace*{\fill}\vspace*{\fill}\clearpage
\vfill
\thispagestyle{empty}
\clearpage

\clearpage\hrule\vskip1pt\hrule 
\section*{\Large Einführung in die Fachdidaktik der Mathematik}
\addcontentsline{toc}{subsection}{Einführung in die Fachdidaktik der Mathematik\ \textcolor{gray}{(\em Katharina Böcherer-Linder)}}
\vskip-2ex  
Katharina Böcherer-Linder\\
Vorlesung mit Übung: Mo, 10--12, SR 226, \href{https://www.openstreetmap.org/?mlat=48.00351\&mlon=7.84815\#map=19/48.00351/7.84815}{Hermann-Herder-Str. 10}\\
\subsubsection*{\Large Inhalt:}
Voluptatem dolorem aliquam ut modi labore quaerat. Labore velit tempora amet voluptatem quiquia amet. Sed aliquam dolor numquam ipsum quaerat. Est velit voluptatem adipisci. Ipsum sit quaerat dolorem adipisci non neque. Etincidunt sed eius numquam dolorem voluptatem. Dolorem ipsum velit quisquam est.
\subsubsection*{\Large Literatur:}
Porro sit numquam dolorem.
\subsubsection*{\Large Vorkenntnisse:}
Man erhält so leicht, dass $x_{1/2} = \frac{-b \pm \sqrt{b^2 - 4ac}}{2a}$
\subsubsection*{\Large Verwendbarkeit, Studien- und Prüfungsleistungen:}
\begin{tabularx}{\textwidth}{ p{.5\textwidth}
    X
    }
    & 
    \makecell[c]{\rotatebox[origin=l]{90}{\parbox{
    4
        cm}{\begin{flushleft}
        Fachdidaktik Mathematik (2HfB21, MEH21, MEB21)
    \end{flushleft} }}} 
    \\[2ex] \hline 
    \rule[0mm]{0cm}{.6cm}Regelmäßige Teilnahme (wie in der Prüfungsordnung definiert). (SL) \rule[-3mm]{0cm}{0cm}
    &
    \makecell[c]{\xmark}
    \\
    \rule[0mm]{0cm}{.6cm}Bestehen der Abschlussklausur (Dauer 1 bis 3 Stunden). (SL) \rule[-3mm]{0cm}{0cm}
    &
    \makecell[c]{\xmark}
    \\
    \rule[0mm]{0cm}{.6cm}Erfolgreiche schriftliche Bearbeitung von mindestens zwei Dritteln der Übungsaufgaben. (SL) \rule[-3mm]{0cm}{0cm}
    &
    \makecell[c]{\xmark}
    \\
    \rule[0mm]{0cm}{.6cm}Pflichtmodul für die Lehramtsoption. (Kommentar) \rule[-3mm]{0cm}{0cm}
    &
    \makecell[c]{\xmark}
    \\
\end{tabularx}

\clearpage\hrule\vskip1pt\hrule 
\section*{\Large Didaktik der Stochastik und der Algebra}
\addcontentsline{toc}{subsection}{Didaktik der Stochastik und der Algebra\ \textcolor{gray}{(\em Katharina Böcherer-Linder)}}
\vskip-2ex  
Katharina Böcherer-Linder\\
Raum und Zeit: Do, 9--12, SR 226, \href{https://www.openstreetmap.org/?mlat=48.00351\&mlon=7.84815\#map=19/48.00351/7.84815}{Hermann-Herder-Str. 10}\\
\subsubsection*{\Large Inhalt:}
Sed modi sit quisquam dolore quaerat est aliquam. Ut sed quisquam velit. Adipisci sit neque numquam quiquia numquam. Eius neque neque quisquam est dolorem neque. Quiquia tempora dolore ut sit labore non. Amet velit neque modi aliquam etincidunt. Neque consectetur dolore adipisci ipsum dolore. Modi tempora aliquam quiquia.
\subsubsection*{\Large Literatur:}
Dolorem sed quisquam modi.
\subsubsection*{\Large Vorkenntnisse:}
Man erhält so leicht, dass $x_{1/2} = \frac{-b \pm \sqrt{b^2 - 4ac}}{2a}$
\subsubsection*{\Large Verwendbarkeit, Studien- und Prüfungsleistungen:}
\begin{tabularx}{\textwidth}{ p{.5\textwidth}
    X
    X
    }
    & 
    \makecell[c]{\rotatebox[origin=l]{90}{\parbox{
    4
        cm}{\begin{flushleft}
        Fachdidaktik der mathematischen Teilgebiete (MEd, MEH21, MEB21)
    \end{flushleft} }}} 
    & 
    \makecell[c]{\rotatebox[origin=l]{90}{\parbox{
    4
        cm}{\begin{flushleft}
        Wahlmodul (BSc, MSc, BSc21, 2HfB21, 2HfB)
    \end{flushleft} }}} 
    \\[2ex] \hline 
    \rule[0mm]{0cm}{.6cm}Klausur über beide Modulteile. (PL) \rule[-3mm]{0cm}{0cm}
    &
    \makecell[c]{\xmark}
    &
    \\
    \rule[0mm]{0cm}{.6cm}Regelmäßige Teilnahme (wie in der Prüfungsordnung definiert). (SL) \rule[-3mm]{0cm}{0cm}
    &
    \makecell[c]{\xmark}
    &
    \makecell[c]{\xmark}
    \\
    \rule[0mm]{0cm}{.6cm}Wöchentliche Lektüre und gegebenenfalls Hausübung. (SL) \rule[-3mm]{0cm}{0cm}
    &
    \makecell[c]{\xmark}
    &
    \makecell[c]{\xmark}
    \\
    \rule[0mm]{0cm}{.6cm}Seminarvortrag mit praktischem und theoretischem Teil. (SL) \rule[-3mm]{0cm}{0cm}
    &
    \makecell[c]{\xmark}
    &
    \makecell[c]{\xmark}
    \\
    \rule[0mm]{0cm}{.6cm}Klausur über beide Modulteile (SL) \rule[-3mm]{0cm}{0cm}
    &
    &
    \makecell[c]{\xmark}
    \\
    \rule[0mm]{0cm}{.6cm}Teil des nur komplett absolvierbaren zweisemestrigen Wahlmoduls "Fachdidaktik der mathematischen Teilgebiete" (6 ECTS-Punkte) (Kommentar) \rule[-3mm]{0cm}{0cm}
    &
    &
    \makecell[c]{\xmark}
    \\
\end{tabularx}

\clearpage\hrule\vskip1pt\hrule 
\section*{\Large Fachdidaktische Forschung, Teil 1: Fachdidaktische Entwicklungsforschung zu ausgewählten Schwerpunkten}
\addcontentsline{toc}{subsection}{Fachdidaktische Forschung, Teil 1: Fachdidaktische Entwicklungsforschung zu ausgewählten Schwerpunkten\ \textcolor{gray}{(\em Frank Reinhold)}}
\vskip-2ex  
Frank Reinhold\\
Raum und Zeit: Mo, 14--16, Raum noch nicht bekannt, -\\
\subsubsection*{\Large Inhalt:}
Porro labore consectetur dolorem aliquam est. Neque aliquam sed velit modi modi. Numquam adipisci neque quiquia dolore eius eius velit. Eius voluptatem dolore eius sit eius ut. Dolore adipisci aliquam sed numquam porro sit neque. Modi ut dolore dolorem. Quiquia ipsum amet quiquia dolorem eius magnam. Sed sit eius velit non ut ut.
\subsubsection*{\Large Literatur:}
Velit quisquam modi quisquam aliquam sit quaerat.
\subsubsection*{\Large Vorkenntnisse:}
Man erhält so leicht, dass $x_{1/2} = \frac{-b \pm \sqrt{b^2 - 4ac}}{2a}$
\subsubsection*{\Large Verwendbarkeit, Studien- und Prüfungsleistungen:}
\begin{tabularx}{\textwidth}{ p{.5\textwidth}
    X
    }
    & 
    \makecell[c]{\rotatebox[origin=l]{90}{\parbox{
    4
        cm}{\begin{flushleft}
        Fachdidaktische Forschung (MEd, MEH21, MEB21)
    \end{flushleft} }}} 
    \\[2ex] \hline 
    \rule[0mm]{0cm}{.6cm}In allen drei Teilen des Moduls: Bearbeitung von Aufgaben nach Maßgabe der Lehrenden im Umfang von insgesamt etwa 60 Stunden. (SL) \rule[-3mm]{0cm}{0cm}
    &
    \makecell[c]{\xmark}
    \\
    \rule[0mm]{0cm}{.6cm}Regelmäßige Teilnahme (wie in der Prüfungsordnung definiert). (SL) \rule[-3mm]{0cm}{0cm}
    &
    \makecell[c]{\xmark}
    \\
\end{tabularx}

\clearpage\hrule\vskip1pt\hrule 
\section*{\Large Fachdidaktische Forschung, Teil 3: Entwicklung und Optimierung eines fachdidaktischen Forschungsprojekts}
\addcontentsline{toc}{subsection}{Fachdidaktische Forschung, Teil 3: Entwicklung und Optimierung eines fachdidaktischen Forschungsprojekts\ \textcolor{gray}{(\em Dozent:inn:en der PH Freiburg)}}
\vskip-2ex  
Dozent:inn:en der PH Freiburg\\
\subsubsection*{\Large Inhalt:}
Sed numquam non numquam tempora dolorem consectetur. Dolor est eius consectetur consectetur. Neque non neque adipisci labore. Sed eius non eius labore modi quaerat. Ipsum non etincidunt est amet dolor dolore. Etincidunt voluptatem quisquam porro. Etincidunt tempora quisquam aliquam. Modi sit quisquam dolorem tempora. Sit non voluptatem consectetur. Amet labore labore quisquam neque voluptatem quiquia.
\subsubsection*{\Large Literatur:}
Quiquia dolor consectetur ipsum amet non tempora.
\subsubsection*{\Large Vorkenntnisse:}
Man erhält so leicht, dass $x_{1/2} = \frac{-b \pm \sqrt{b^2 - 4ac}}{2a}$
\subsubsection*{\Large Verwendbarkeit, Studien- und Prüfungsleistungen:}
\begin{tabularx}{\textwidth}{ p{.5\textwidth}
    X
    }
    & 
    \makecell[c]{\rotatebox[origin=l]{90}{\parbox{
    4
        cm}{\begin{flushleft}
        Fachdidaktische Forschung (MEd, MEH21, MEB21)
    \end{flushleft} }}} 
    \\[2ex] \hline 
    \rule[0mm]{0cm}{.6cm}In allen drei Teilen des Moduls: Bearbeitung von Aufgaben nach Maßgabe der Lehrenden im Umfang von insgesamt etwa 60 Stunden. (SL) \rule[-3mm]{0cm}{0cm}
    &
    \makecell[c]{\xmark}
    \\
    \rule[0mm]{0cm}{.6cm}Regelmäßige Teilnahme (wie in der Prüfungsordnung definiert). (SL) \rule[-3mm]{0cm}{0cm}
    &
    \makecell[c]{\xmark}
    \\
\end{tabularx}

\clearpage\hrule\vskip1pt\hrule 
\section*{\Large Didaktik der Funktionen und der Analysis}
\addcontentsline{toc}{subsection}{Didaktik der Funktionen und der Analysis\ \textcolor{gray}{(\em Ralf Erens)}}
\vskip-2ex  
Ralf Erens\\
Raum und Zeit: Mi, 15--18, SR 404, \href{https://www.openstreetmap.org/?mlat=48.00065\&mlon=7.84591\#map=19/48.00065/7.84591}{Ernst-Zermelo-Straße 1}\\
\subsubsection*{\Large Inhalt:}
Magnam porro est sit dolor eius. Sit adipisci amet modi porro. Numquam neque etincidunt quisquam modi magnam etincidunt. Modi quaerat est consectetur quiquia sit quisquam. Quaerat numquam dolorem eius ipsum magnam.
\subsubsection*{\Large Literatur:}
Quiquia neque amet est.
\subsubsection*{\Large Vorkenntnisse:}
Man erhält so leicht, dass $x_{1/2} = \frac{-b \pm \sqrt{b^2 - 4ac}}{2a}$
\subsubsection*{\Large Verwendbarkeit, Studien- und Prüfungsleistungen:}
\begin{tabularx}{\textwidth}{ p{.5\textwidth}
    X
    X
    }
    & 
    \makecell[c]{\rotatebox[origin=l]{90}{\parbox{
    4
        cm}{\begin{flushleft}
        Fachdidaktik der mathematischen Teilgebiete (MEd, MEH21, MEB21)
    \end{flushleft} }}} 
    & 
    \makecell[c]{\rotatebox[origin=l]{90}{\parbox{
    4
        cm}{\begin{flushleft}
        Wahlmodul (BSc, MSc, BSc21, 2HfB21, 2HfB)
    \end{flushleft} }}} 
    \\[2ex] \hline 
    \rule[0mm]{0cm}{.6cm}Regelmäßige Teilnahme (wie in der Prüfungsordnung definiert). (SL) \rule[-3mm]{0cm}{0cm}
    &
    \makecell[c]{\xmark}
    &
    \makecell[c]{\xmark}
    \\
    \rule[0mm]{0cm}{.6cm}Wöchentliche Lektüre und gegebenenfalls Hausübung. (SL) \rule[-3mm]{0cm}{0cm}
    &
    \makecell[c]{\xmark}
    &
    \makecell[c]{\xmark}
    \\
    \rule[0mm]{0cm}{.6cm}Seminarvortrag mit praktischem und theoretischem Teil. (SL) \rule[-3mm]{0cm}{0cm}
    &
    \makecell[c]{\xmark}
    &
    \makecell[c]{\xmark}
    \\
    \rule[0mm]{0cm}{.6cm}Klausur über beide Modulteile. (PL) \rule[-3mm]{0cm}{0cm}
    &
    \makecell[c]{\xmark}
    &
    \\
    \rule[0mm]{0cm}{.6cm}Klausur über beide Modulteile (SL) \rule[-3mm]{0cm}{0cm}
    &
    &
    \makecell[c]{\xmark}
    \\
    \rule[0mm]{0cm}{.6cm}Teil des nur komplett absolvierbaren zweisemestrigen Wahlmoduls "Fachdidaktik der mathematischen Teilgebiete" (6 ECTS-Punkte) (Kommentar) \rule[-3mm]{0cm}{0cm}
    &
    &
    \makecell[c]{\xmark}
    \\
\end{tabularx}

\clearpage\hrule\vskip1pt\hrule 
\section*{\Large Fachdidaktikseminar: Mathe<sub>Unterricht</sub> = Mathe<sub>Studium</sub> ± x}
\addcontentsline{toc}{subsection}{Fachdidaktikseminar: Mathe<sub>Unterricht</sub> = Mathe<sub>Studium</sub> ± x\ \textcolor{gray}{(\em Holger Dietz)}}
\vskip-2ex  
Holger Dietz\\
Raum und Zeit: Mi, 9--12, Raum B 106, Seminar für Ausbildung und Fortbildung der Lehrkräfte Freiburg, \href{https://www.openstreetmap.org/?mlat=47.98021\&mlon=7.82836\#map=19/47.98021/7.82836}{Oltmannstraße 22}\\
\subsubsection*{\Large Inhalt:}
Consectetur velit quisquam quiquia neque etincidunt. Magnam est velit velit ut eius quiquia quaerat. Sit etincidunt aliquam amet consectetur non. Modi dolorem modi labore porro sed quaerat. Quaerat est modi amet modi labore voluptatem. Etincidunt labore tempora dolorem dolore. Magnam dolorem dolorem quisquam voluptatem non est numquam. Quiquia sed consectetur porro tempora. Dolorem eius neque magnam ipsum.
\subsubsection*{\Large Literatur:}
Sed est sit non tempora eius.
\subsubsection*{\Large Vorkenntnisse:}
Man erhält so leicht, dass $x_{1/2} = \frac{-b \pm \sqrt{b^2 - 4ac}}{2a}$
\subsubsection*{\Large Verwendbarkeit, Studien- und Prüfungsleistungen:}
\begin{tabularx}{\textwidth}{ p{.5\textwidth}
    X
    X
    }
    & 
    \makecell[c]{\rotatebox[origin=l]{90}{\parbox{
    4
        cm}{\begin{flushleft}
        Fachdidaktische Entwicklung (MEd, MEH21, MEB21)
    \end{flushleft} }}} 
    & 
    \makecell[c]{\rotatebox[origin=l]{90}{\parbox{
    4
        cm}{\begin{flushleft}
        Wahlmodul (BSc, MSc, BSc21, 2HfB21, 2HfB)
    \end{flushleft} }}} 
    \\[2ex] \hline 
    \rule[0mm]{0cm}{.6cm}Seminarvortrag mit praktischem Teil. (SL) \rule[-3mm]{0cm}{0cm}
    &
    \makecell[c]{\xmark}
    &
    \makecell[c]{\xmark}
    \\
    \rule[0mm]{0cm}{.6cm}Regelmäßige Teilnahme (wie in der Prüfungsordnung definiert). (SL) \rule[-3mm]{0cm}{0cm}
    &
    \makecell[c]{\xmark}
    &
    \makecell[c]{\xmark}
    \\
    \rule[0mm]{0cm}{.6cm}Bearbeitung der wöchentlichen Vor- und Nachbereitungsaugaben. (SL) \rule[-3mm]{0cm}{0cm}
    &
    \makecell[c]{\xmark}
    &
    \makecell[c]{\xmark}
    \\
    \rule[0mm]{0cm}{.6cm}Für das absolvierte Modul (oder ggf. den Teil des Moduls) gibt es 4 ECTS-Punkte. (Kommentar) \rule[-3mm]{0cm}{0cm}
    &
    &
    \makecell[c]{\xmark}
    \\
\end{tabularx}

\clearpage\hrule\vskip1pt\hrule 
\section*{\Large Fachdidaktische Forschung, Teil 2: Methoden der mathematikdidaktischen Forschung}
\addcontentsline{toc}{subsection}{Fachdidaktische Forschung, Teil 2: Methoden der mathematikdidaktischen Forschung\ \textcolor{gray}{(\em Frank Reinhold)}}
\vskip-2ex  
Frank Reinhold\\
Raum und Zeit: Mo, 10--13, Raum noch nicht bekannt, -\\
\subsubsection*{\Large Inhalt:}
Aliquam tempora etincidunt numquam numquam ut. Labore ipsum aliquam velit sit. Non quaerat sit porro etincidunt modi modi. Ipsum ut voluptatem tempora. Neque amet tempora adipisci. Dolorem dolore quisquam adipisci dolor dolore. Modi labore tempora ipsum porro. Labore tempora porro ut etincidunt tempora neque. Quiquia numquam voluptatem sit numquam neque.
\subsubsection*{\Large Literatur:}
Est dolore magnam consectetur.
\subsubsection*{\Large Vorkenntnisse:}
Man erhält so leicht, dass $x_{1/2} = \frac{-b \pm \sqrt{b^2 - 4ac}}{2a}$
\subsubsection*{\Large Verwendbarkeit, Studien- und Prüfungsleistungen:}
\begin{tabularx}{\textwidth}{ p{.5\textwidth}
    X
    }
    & 
    \makecell[c]{\rotatebox[origin=l]{90}{\parbox{
    4
        cm}{\begin{flushleft}
        Fachdidaktische Forschung (MEd, MEH21, MEB21)
    \end{flushleft} }}} 
    \\[2ex] \hline 
    \rule[0mm]{0cm}{.6cm}In allen drei Teilen des Moduls: Bearbeitung von Aufgaben nach Maßgabe der Lehrenden im Umfang von insgesamt etwa 60 Stunden. (SL) \rule[-3mm]{0cm}{0cm}
    &
    \makecell[c]{\xmark}
    \\
    \rule[0mm]{0cm}{.6cm}Regelmäßige Teilnahme (wie in der Prüfungsordnung definiert). (SL) \rule[-3mm]{0cm}{0cm}
    &
    \makecell[c]{\xmark}
    \\
\end{tabularx}

\clearpage\hrule\vskip1pt\hrule 
\section*{\Large Fachdidaktikseminare der PH Freiburg}
\addcontentsline{toc}{subsection}{Fachdidaktikseminare der PH Freiburg\ \textcolor{gray}{(\em Dozent:inn:en der PH Freiburg)}}
\vskip-2ex  
Dozent:inn:en der PH Freiburg\\
\subsubsection*{\Large Inhalt:}
Adipisci sit aliquam non neque dolore magnam. Amet est voluptatem aliquam neque. Quaerat labore aliquam amet etincidunt sed adipisci adipisci. Consectetur sed voluptatem tempora. Magnam non adipisci magnam neque. Quisquam numquam aliquam magnam eius aliquam amet. Non tempora tempora voluptatem consectetur dolor. Aliquam ut quiquia eius velit ut. Labore dolor sed labore quiquia.
\subsubsection*{\Large Literatur:}
Eius dolore quiquia quisquam adipisci.
\subsubsection*{\Large Vorkenntnisse:}
Man erhält so leicht, dass $x_{1/2} = \frac{-b \pm \sqrt{b^2 - 4ac}}{2a}$
\subsubsection*{\Large Verwendbarkeit, Studien- und Prüfungsleistungen:}
\begin{tabularx}{\textwidth}{ p{.5\textwidth}
    }
    \\[2ex] \hline 
\end{tabularx}

\clearpage\hrule\vskip1pt\hrule 
\section*{\Large Fachdidaktische Entwicklung: Gleichungen}
\addcontentsline{toc}{subsection}{Fachdidaktische Entwicklung: Gleichungen\ \textcolor{gray}{(\em Jürgen Kury)}}
\vskip-2ex  
Jürgen Kury\\
Raum und Zeit: Di, 14--18, Raum –117, KG 2, Pädagogische Hochschule\\
\subsubsection*{\Large Inhalt:}
Voluptatem eius ut dolore sit numquam sit dolor. Adipisci quisquam aliquam modi labore etincidunt sed non. Velit est dolorem quaerat dolor quaerat adipisci modi. Dolor dolore sit dolor adipisci. Labore etincidunt quiquia sed sed consectetur porro modi. Etincidunt ipsum adipisci sit.
\subsubsection*{\Large Literatur:}
Neque amet amet dolorem consectetur porro voluptatem porro.
\subsubsection*{\Large Vorkenntnisse:}
Man erhält so leicht, dass $x_{1/2} = \frac{-b \pm \sqrt{b^2 - 4ac}}{2a}$
\subsubsection*{\Large Verwendbarkeit, Studien- und Prüfungsleistungen:}
\begin{tabularx}{\textwidth}{ p{.5\textwidth}
    }
    \\[2ex] \hline 
\end{tabularx}

\clearpage
\phantomsection
\thispagestyle{empty}
\vspace*{\fill}
\begin{center}
\Huge\bfseries 2c. Praktische Übungen
\end{center}
\addcontentsline{toc}{section}{\textbf{2c. Praktische Übungen}}
\addtocontents{toc}{\medskip\hrule\medskip}\vspace*{\fill}\vspace*{\fill}\clearpage
\vfill
\thispagestyle{empty}
\clearpage

\clearpage\hrule\vskip1pt\hrule 
\section*{\Large Einführung in die Programmierung für Studierende der Naturwissenschaften}
\addcontentsline{toc}{subsection}{Einführung in die Programmierung für Studierende der Naturwissenschaften\ \textcolor{gray}{(\em Ludwig Striet)}}
\vskip-2ex  
Ludwig Striet\\
Vorlesung: Mo, 16--18, HS Weismann-Haus, \href{https://www.openstreetmap.org/?mlat=48.00170\&mlon=7.84946\#map=19/48.00170/7.84946}{Albertstr. 21a}\\
\subsubsection*{\Large Inhalt:}
Magnam voluptatem modi etincidunt amet quaerat dolore. Ut velit voluptatem non numquam voluptatem labore. Adipisci ipsum neque adipisci ipsum dolor. Modi numquam modi etincidunt. Voluptatem non numquam consectetur sit dolore porro ut. Aliquam modi tempora sed tempora. Dolor etincidunt labore modi numquam dolore sit. Dolorem adipisci sit amet.
\subsubsection*{\Large Literatur:}
Neque quisquam ut numquam porro consectetur.
\subsubsection*{\Large Vorkenntnisse:}
Man erhält so leicht, dass $x_{1/2} = \frac{-b \pm \sqrt{b^2 - 4ac}}{2a}$
\subsubsection*{\Large Verwendbarkeit, Studien- und Prüfungsleistungen:}
\begin{tabularx}{\textwidth}{ p{.5\textwidth}
    X
    X
    X
    }
    & 
    \makecell[c]{\rotatebox[origin=l]{90}{\parbox{
    4
        cm}{\begin{flushleft}
        Praktische Übung (2HfB21, MEH21, MEB21)
    \end{flushleft} }}} 
    & 
    \makecell[c]{\rotatebox[origin=l]{90}{\parbox{
    4
        cm}{\begin{flushleft}
        BOK-Kurs (BSc)
    \end{flushleft} }}} 
    & 
    \makecell[c]{\rotatebox[origin=l]{90}{\parbox{
    4
        cm}{\begin{flushleft}
        Mathematische Ergänzung (MEd)
    \end{flushleft} }}} 
    \\[2ex] \hline 
    \rule[0mm]{0cm}{.6cm}Mindestens 65% der erreichbaren Punkte auf die zu bearbeitenden Übungsaufgaben.
Jeder Aufforderung zur genaueren Erläuterung einer eingereichten Lösung
seitens des Tutors/der Tutorin ist nachzukommen. (SL) \rule[-3mm]{0cm}{0cm}
    &
    \makecell[c]{\xmark}
    &
    \makecell[c]{\xmark}
    &
    \makecell[c]{\xmark}
    \\
    \rule[0mm]{0cm}{.6cm}Die Anforderungen des Kurses entsprechen 6 ECTS-Punkten. Es gibt andere andere Wahlmöglichkeiten für das Modul, die der ECTS-Punktzahl des Moduls entsprechen. (Kommentar) \rule[-3mm]{0cm}{0cm}
    &
    \makecell[c]{\xmark}
    &
    &
    \makecell[c]{\xmark}
    \\
    \rule[0mm]{0cm}{.6cm}Anfertigung einer Projektarbeit bis zum Ende der Vorlesungszeit und Kurzvortrag über das Projekt. (SL) \rule[-3mm]{0cm}{0cm}
    &
    \makecell[c]{\xmark}
    &
    \makecell[c]{\xmark}
    &
    \makecell[c]{\xmark}
    \\
    \rule[0mm]{0cm}{.6cm}Für B.Sc.-Studierende (und nur für diese): Belegung des Kurses über das ZfS. (Kommentar) \rule[-3mm]{0cm}{0cm}
    &
    &
    \makecell[c]{\xmark}
    &
    \\
\end{tabularx}

\clearpage\hrule\vskip1pt\hrule 
\section*{\Large Praktische Übung zu „Numerik“}
\addcontentsline{toc}{subsection}{Praktische Übung zu „Numerik“\ \textcolor{gray}{(\em Alexei Gazca)}}
\vskip-2ex  
Alexei Gazca\\
\subsubsection*{\Large Inhalt:}
Numquam neque aliquam magnam numquam. Neque eius dolore est dolor aliquam. Quaerat dolor voluptatem aliquam tempora. Eius dolor sit modi ipsum amet. Voluptatem ut quaerat quaerat dolore est est voluptatem. Labore est etincidunt dolorem. Dolorem ipsum aliquam aliquam dolorem. Velit labore magnam eius quaerat aliquam labore consectetur.
\subsubsection*{\Large Literatur:}
Sed tempora quiquia etincidunt dolor magnam.
\subsubsection*{\Large Vorkenntnisse:}
Man erhält so leicht, dass $x_{1/2} = \frac{-b \pm \sqrt{b^2 - 4ac}}{2a}$
\subsubsection*{\Large Verwendbarkeit, Studien- und Prüfungsleistungen:}
\begin{tabularx}{\textwidth}{ p{.5\textwidth}
    X
    X
    X
    }
    & 
    \makecell[c]{\rotatebox[origin=l]{90}{\parbox{
    4
        cm}{\begin{flushleft}
        Teil des Moduls "Numerik" (BSc, BSc21, 2HfB21, MEH21)
    \end{flushleft} }}} 
    & 
    \makecell[c]{\rotatebox[origin=l]{90}{\parbox{
    4
        cm}{\begin{flushleft}
        Praktische Übung (2HfB21, MEH21, MEB21)
    \end{flushleft} }}} 
    & 
    \makecell[c]{\rotatebox[origin=l]{90}{\parbox{
    4
        cm}{\begin{flushleft}
        Mathematische Ergänzung (MEd)
    \end{flushleft} }}} 
    \\[2ex] \hline 
    \rule[0mm]{0cm}{.6cm}Regelmäßige Teilnahme (wie in der Prüfungsordnung definiert). (SL) \rule[-3mm]{0cm}{0cm}
    &
    \makecell[c]{\xmark}
    &
    \makecell[c]{\xmark}
    &
    \makecell[c]{\xmark}
    \\
    \rule[0mm]{0cm}{.6cm}Die Anforderungen an die Studienleistungen gelten separat für beide Semester des Moduls! (Kommentar) \rule[-3mm]{0cm}{0cm}
    &
    \makecell[c]{\xmark}
    &
    \makecell[c]{\xmark}
    &
    \makecell[c]{\xmark}
    \\
    \rule[0mm]{0cm}{.6cm}Erreichen von mindestens 50% der Punkte, die insgesamt durch die Bearbeitung der für die Praktische Übung ausgegebenen Programmieraufgaben erreicht werden können. (SL) \rule[-3mm]{0cm}{0cm}
    &
    \makecell[c]{\xmark}
    &
    \makecell[c]{\xmark}
    &
    \makecell[c]{\xmark}
    \\
\end{tabularx}

\clearpage\hrule\vskip1pt\hrule 
\section*{\Large Praktische Übung zu „Stochastik“}
\addcontentsline{toc}{subsection}{Praktische Übung zu „Stochastik“\ \textcolor{gray}{(\em Ernst August v. Hammerstein)}}
\vskip-2ex  
Ernst August v. Hammerstein\\
Raum und Zeit: Di, 14--16, Computerpool R -100, \href{https://www.openstreetmap.org/?mlat=48.00351\&mlon=7.84815\#map=19/48.00351/7.84815}{Hermann-Herder-Str. 10}\\
\subsubsection*{\Large Inhalt:}
Porro quiquia quaerat sed velit ipsum ipsum. Voluptatem sed amet eius adipisci porro amet non. Ut eius modi sit labore neque porro velit. Magnam aliquam tempora modi quisquam etincidunt. Est tempora sit adipisci modi dolor. Est quaerat ipsum magnam voluptatem dolore dolore. Sit consectetur non amet porro porro tempora. Tempora ut magnam consectetur.
\subsubsection*{\Large Literatur:}
Consectetur dolor quiquia neque quaerat voluptatem neque.
\subsubsection*{\Large Vorkenntnisse:}
Man erhält so leicht, dass $x_{1/2} = \frac{-b \pm \sqrt{b^2 - 4ac}}{2a}$
\subsubsection*{\Large Verwendbarkeit, Studien- und Prüfungsleistungen:}
\begin{tabularx}{\textwidth}{ p{.5\textwidth}
    X
    X
    X
    X
    }
    & 
    \makecell[c]{\rotatebox[origin=l]{90}{\parbox{
    8
        cm}{\begin{flushleft}
        Praktische Übung (2HfB21, MEH21, MEB21)
    \end{flushleft} }}} 
    & 
    \makecell[c]{\rotatebox[origin=l]{90}{\parbox{
    8
        cm}{\begin{flushleft}
        Teil des Moduls "Stochastik" (BSc, 2HfB21, MEH21)
    \end{flushleft} }}} 
    & 
    \makecell[c]{\rotatebox[origin=l]{90}{\parbox{
    8
        cm}{\begin{flushleft}
        Mathematische Ergänzung (MEd)
    \end{flushleft} }}} 
    & 
    \makecell[c]{\rotatebox[origin=l]{90}{\parbox{
    8
        cm}{\begin{flushleft}
        Wahlmodul (BSc, MSc, BSc21, 2HfB21, 2HfB)
    \end{flushleft} }}} 
    \\[2ex] \hline 
    \rule[0mm]{0cm}{.6cm}Bestehen der Abschlussklausur (Dauer 1 bis 3 Stunden). (SL) \rule[-3mm]{0cm}{0cm}
    &
    \makecell[c]{\xmark}
    &
    \makecell[c]{\xmark}
    &
    \makecell[c]{\xmark}
    &
    \makecell[c]{\xmark}
    \\
    \rule[0mm]{0cm}{.6cm}Für das absolvierte Modul (oder ggf. den Teil des Moduls) gibt es 3 ECTS-Punkte. (Kommentar) \rule[-3mm]{0cm}{0cm}
    &
    &
    &
    &
    \makecell[c]{\xmark}
    \\
\end{tabularx}

\clearpage\hrule\vskip1pt\hrule 
\section*{\Large Einführung in Mathematica}
\addcontentsline{toc}{subsection}{Einführung in Mathematica\ \textcolor{gray}{(\em Robin Brüser)}}
\vskip-2ex  
Robin Brüser\\
Raum und Zeit: Mo, Mi, 14--16, CIP-Pool 2, Gustav-Mie-Haus\\
\subsubsection*{\Large Inhalt:}
Quaerat etincidunt numquam quisquam voluptatem. Aliquam sed ipsum velit. Sit sed velit est quaerat labore adipisci. Quiquia porro aliquam sit ut consectetur adipisci. Porro velit modi dolore. Quaerat non amet labore ut consectetur numquam. Amet est quiquia quisquam dolor aliquam numquam.
\subsubsection*{\Large Literatur:}
Modi velit consectetur neque non.
\subsubsection*{\Large Vorkenntnisse:}
Man erhält so leicht, dass $x_{1/2} = \frac{-b \pm \sqrt{b^2 - 4ac}}{2a}$
\subsubsection*{\Large Verwendbarkeit, Studien- und Prüfungsleistungen:}
\begin{tabularx}{\textwidth}{ p{.5\textwidth}
    }
    \\[2ex] \hline 
\end{tabularx}

\clearpage\hrule\vskip1pt\hrule 
\section*{\Large Praktische Übung zu "Maschinelles Lernen aus Stochastischer Sicht"}
\addcontentsline{toc}{subsection}{Praktische Übung zu "Maschinelles Lernen aus Stochastischer Sicht"\ \textcolor{gray}{(\em Thorsten Schmidt)}}
\vskip-2ex  
Thorsten Schmidt\\
\subsubsection*{\Large Inhalt:}
Quiquia est est numquam labore neque numquam porro. Porro sit sed porro aliquam. Voluptatem dolorem quaerat sed eius modi magnam quiquia. Amet numquam etincidunt est eius voluptatem numquam dolorem. Quaerat amet ut amet. Voluptatem voluptatem est ut non labore adipisci non. Dolor dolorem ut modi tempora. Porro voluptatem voluptatem non dolore porro quaerat.
\subsubsection*{\Large Literatur:}
Consectetur modi est neque neque.
\subsubsection*{\Large Vorkenntnisse:}
Man erhält so leicht, dass $x_{1/2} = \frac{-b \pm \sqrt{b^2 - 4ac}}{2a}$
\subsubsection*{\Large Verwendbarkeit, Studien- und Prüfungsleistungen:}
\begin{tabularx}{\textwidth}{ p{.5\textwidth}
    }
    \\[2ex] \hline 
\end{tabularx}

\clearpage
\phantomsection
\thispagestyle{empty}
\vspace*{\fill}
\begin{center}
\Huge\bfseries 3a. Proseminare
\end{center}
\addcontentsline{toc}{section}{\textbf{3a. Proseminare}}
\addtocontents{toc}{\medskip\hrule\medskip}\vspace*{\fill}\vspace*{\fill}\clearpage
\vfill
\thispagestyle{empty}
\clearpage

\clearpage\hrule\vskip1pt\hrule 
\section*{\Large Proseminar: Unendlichdimensionale Vektorräume}
\addcontentsline{toc}{subsection}{Proseminar: Unendlichdimensionale Vektorräume\ \textcolor{gray}{(\em Susanne Knies)}}
\vskip-2ex  
Susanne Knies, Assistenz: Vivien Vogelmann\\
Raum und Zeit: Do, 14--16, SR 127, \href{https://www.openstreetmap.org/?mlat=48.00065\&mlon=7.84591\#map=19/48.00065/7.84591}{Ernst-Zermelo-Straße 1}\\
\subsubsection*{\Large Inhalt:}
Porro dolor neque neque. Sit eius magnam consectetur. Non numquam etincidunt eius tempora modi quiquia. Est labore sit porro aliquam dolorem eius. Est dolore quisquam aliquam amet magnam aliquam consectetur. Dolor consectetur aliquam amet. Quaerat aliquam velit ut quisquam est ipsum voluptatem.
\subsubsection*{\Large Literatur:}
Numquam aliquam labore amet sit adipisci modi.
\subsubsection*{\Large Vorkenntnisse:}
Man erhält so leicht, dass $x_{1/2} = \frac{-b \pm \sqrt{b^2 - 4ac}}{2a}$
\subsubsection*{\Large Verwendbarkeit, Studien- und Prüfungsleistungen:}
\begin{tabularx}{\textwidth}{ p{.5\textwidth}
    X
    }
    & 
    \makecell[c]{\rotatebox[origin=l]{90}{\parbox{
    4
        cm}{\begin{flushleft}
        Proseminar (BSc, BSc21, 2HfB21, MEH21, MEB21, GymPO)
    \end{flushleft} }}} 
    \\[2ex] \hline 
    \rule[0mm]{0cm}{.6cm}Regelmäßige Teilnahme (wie in der Prüfungsordnung definiert). (SL) \rule[-3mm]{0cm}{0cm}
    &
    \makecell[c]{\xmark}
    \\
    \rule[0mm]{0cm}{.6cm}Etwa 45- bis 90-minütiger Vortrag. (PL) \rule[-3mm]{0cm}{0cm}
    &
    \makecell[c]{\xmark}
    \\
\end{tabularx}

\clearpage\hrule\vskip1pt\hrule 
\section*{\Large Proseminar: Kombinatorik}
\addcontentsline{toc}{subsection}{Proseminar: Kombinatorik\ \textcolor{gray}{(\em Markus Junker)}}
\vskip-2ex  
Markus Junker, Assistenz: Charlotte Bartnick\\
Raum und Zeit: Mi, 10--12, SR 404, \href{https://www.openstreetmap.org/?mlat=48.00065\&mlon=7.84591\#map=19/48.00065/7.84591}{Ernst-Zermelo-Straße 1}\\
\subsubsection*{\Large Inhalt:}
Quiquia amet est magnam consectetur velit non. Quaerat amet aliquam ipsum aliquam eius numquam. Sit sed modi tempora sit quiquia modi. Non ipsum quiquia adipisci neque voluptatem sed sit. Adipisci eius porro sed amet. Consectetur quaerat ut dolorem eius est etincidunt.
\subsubsection*{\Large Literatur:}
Dolor sit numquam sit adipisci eius sit modi.
\subsubsection*{\Large Vorkenntnisse:}
Man erhält so leicht, dass $x_{1/2} = \frac{-b \pm \sqrt{b^2 - 4ac}}{2a}$
\subsubsection*{\Large Verwendbarkeit, Studien- und Prüfungsleistungen:}
\begin{tabularx}{\textwidth}{ p{.5\textwidth}
    }
    \\[2ex] \hline 
\end{tabularx}

\clearpage\hrule\vskip1pt\hrule 
\section*{\Large Proseminar: Eindimensionale Variationsrechnung}
\addcontentsline{toc}{subsection}{Proseminar: Eindimensionale Variationsrechnung\ \textcolor{gray}{(\em Patrick Dondl)}}
\vskip-2ex  
Patrick Dondl\\
Raum und Zeit: Di, 10--12, SR 226, \href{https://www.openstreetmap.org/?mlat=48.00351\&mlon=7.84815\#map=19/48.00351/7.84815}{Hermann-Herder-Str. 10}\\
\subsubsection*{\Large Inhalt:}
Modi quisquam numquam amet dolor non. Eius dolor adipisci etincidunt. Est quisquam tempora etincidunt non numquam quisquam. Numquam quiquia numquam dolor aliquam velit adipisci. Sed amet adipisci numquam quaerat. Ipsum quiquia consectetur etincidunt velit magnam tempora. Adipisci consectetur sed porro. Etincidunt ipsum modi ipsum amet porro sit.
\subsubsection*{\Large Literatur:}
Labore sit adipisci porro quiquia adipisci.
\subsubsection*{\Large Vorkenntnisse:}
Man erhält so leicht, dass $x_{1/2} = \frac{-b \pm \sqrt{b^2 - 4ac}}{2a}$
\subsubsection*{\Large Verwendbarkeit, Studien- und Prüfungsleistungen:}
\begin{tabularx}{\textwidth}{ p{.5\textwidth}
    }
    \\[2ex] \hline 
\end{tabularx}

\clearpage
\phantomsection
\thispagestyle{empty}
\vspace*{\fill}
\begin{center}
\Huge\bfseries 3b. Seminare
\end{center}
\addcontentsline{toc}{section}{\textbf{3b. Seminare}}
\addtocontents{toc}{\medskip\hrule\medskip}\vspace*{\fill}\vspace*{\fill}\clearpage
\vfill
\thispagestyle{empty}
\clearpage

\clearpage\hrule\vskip1pt\hrule 
\section*{\Large Seminar: Geometrische Variationsprobleme}
\addcontentsline{toc}{subsection}{Seminar: Geometrische Variationsprobleme\ \textcolor{gray}{(\em Ernst Kuwert)}}
\vskip-2ex  
Ernst Kuwert\\
\subsubsection*{\Large Inhalt:}
Voluptatem modi neque voluptatem. Consectetur consectetur adipisci magnam neque aliquam sed ut. Magnam consectetur quisquam modi consectetur est eius. Tempora quiquia eius sed est labore. Eius quiquia adipisci labore est. Non modi porro neque amet ipsum ipsum ipsum. Adipisci quisquam tempora quaerat. Non tempora quisquam amet amet labore quiquia.
\subsubsection*{\Large Literatur:}
Etincidunt est aliquam dolor quaerat.
\subsubsection*{\Large Vorkenntnisse:}
Man erhält so leicht, dass $x_{1/2} = \frac{-b \pm \sqrt{b^2 - 4ac}}{2a}$
\subsubsection*{\Large Verwendbarkeit, Studien- und Prüfungsleistungen:}
\begin{tabularx}{\textwidth}{ p{.5\textwidth}
    X
    X
    X
    X
    X
    X
    }
    & 
    \makecell[c]{\rotatebox[origin=l]{90}{\parbox{
    8
        cm}{\begin{flushleft}
        Mathematische Ergänzung (MEd)
    \end{flushleft} }}} 
    & 
    \makecell[c]{\rotatebox[origin=l]{90}{\parbox{
    8
        cm}{\begin{flushleft}
        Modul im Wahlpflichtbereich Mathematik (BSc, BSc21)
    \end{flushleft} }}} 
    & 
    \makecell[c]{\rotatebox[origin=l]{90}{\parbox{
    8
        cm}{\begin{flushleft}
        Seminar (BSc21, GymPO)
    \end{flushleft} }}} 
    & 
    \makecell[c]{\rotatebox[origin=l]{90}{\parbox{
    8
        cm}{\begin{flushleft}
        Wahlmodul (BSc, MSc, BSc21, 2HfB21, 2HfB)
    \end{flushleft} }}} 
    & 
    \makecell[c]{\rotatebox[origin=l]{90}{\parbox{
    8
        cm}{\begin{flushleft}
        Mathematische Seminar A oder B (MSc)
    \end{flushleft} }}} 
    & 
    \makecell[c]{\rotatebox[origin=l]{90}{\parbox{
    8
        cm}{\begin{flushleft}
        Bachelor-Seminar (Teil des Bachelor-Moduls) (BSc)
    \end{flushleft} }}} 
    \\[2ex] \hline 
    \rule[0mm]{0cm}{.6cm}Regelmäßige Teilnahme (wie in der Prüfungsordnung definiert). (SL) \rule[-3mm]{0cm}{0cm}
    &
    \makecell[c]{\xmark}
    &
    \makecell[c]{\xmark}
    &
    \makecell[c]{\xmark}
    &
    \makecell[c]{\xmark}
    &
    \makecell[c]{\xmark}
    &
    \makecell[c]{\xmark}
    \\
    \rule[0mm]{0cm}{.6cm}Für das absolvierte Modul (oder ggf. den Teil des Moduls) gibt es 6 ECTS-Punkte. (Kommentar) \rule[-3mm]{0cm}{0cm}
    &
    &
    \makecell[c]{\xmark}
    &
    &
    \makecell[c]{\xmark}
    &
    &
    \\
    \rule[0mm]{0cm}{.6cm}Etwa 45- bis 90-minütiger Vortrag. (PL) \rule[-3mm]{0cm}{0cm}
    &
    &
    \makecell[c]{\xmark}
    &
    \makecell[c]{\xmark}
    &
    &
    \makecell[c]{\xmark}
    &
    \makecell[c]{\xmark}
    \\
    \rule[0mm]{0cm}{.6cm}Verwendbar für die Option "Individuelle Schwerpunktgestaltung". (Kommentar) \rule[-3mm]{0cm}{0cm}
    &
    &
    &
    &
    \makecell[c]{\xmark}
    &
    &
    \\
\end{tabularx}

\clearpage\hrule\vskip1pt\hrule 
\section*{\Large Seminar: Medical Data Science}
\addcontentsline{toc}{subsection}{Seminar: Medical Data Science\ \textcolor{gray}{(\em Harald Binder)}}
\vskip-2ex  
Harald Binder\\
Raum und Zeit: Mi, 10--11: 30, HS Medizinische Biometrie, \href{https://www.openstreetmap.org/?mlat=48.00248\&mlon=7.84681\#map=19/48.00248/7.84681}{Stefan-Meier-Str. 26}\\
\subsubsection*{\Large Inhalt:}
Neque etincidunt quisquam aliquam. Non numquam etincidunt velit. Porro magnam voluptatem modi tempora velit consectetur. Amet quaerat sed modi eius est consectetur neque. Aliquam tempora dolor neque neque etincidunt modi sit.
\subsubsection*{\Large Literatur:}
Non dolore quisquam quiquia velit.
\subsubsection*{\Large Vorkenntnisse:}
Man erhält so leicht, dass $x_{1/2} = \frac{-b \pm \sqrt{b^2 - 4ac}}{2a}$
\subsubsection*{\Large Verwendbarkeit, Studien- und Prüfungsleistungen:}
\begin{tabularx}{\textwidth}{ p{.5\textwidth}
    X
    X
    X
    X
    X
    X
    }
    & 
    \makecell[c]{\rotatebox[origin=l]{90}{\parbox{
    8
        cm}{\begin{flushleft}
        Bachelor-Seminar (Teil des Bachelor-Moduls) (BSc)
    \end{flushleft} }}} 
    & 
    \makecell[c]{\rotatebox[origin=l]{90}{\parbox{
    8
        cm}{\begin{flushleft}
        Modul im Wahlpflichtbereich Mathematik (BSc, BSc21)
    \end{flushleft} }}} 
    & 
    \makecell[c]{\rotatebox[origin=l]{90}{\parbox{
    8
        cm}{\begin{flushleft}
        Wahlmodul (BSc, MSc, BSc21, 2HfB21, 2HfB)
    \end{flushleft} }}} 
    & 
    \makecell[c]{\rotatebox[origin=l]{90}{\parbox{
    8
        cm}{\begin{flushleft}
        Mathematische Seminar A oder B (MSc)
    \end{flushleft} }}} 
    & 
    \makecell[c]{\rotatebox[origin=l]{90}{\parbox{
    8
        cm}{\begin{flushleft}
        Mathematische Ergänzung (MEd)
    \end{flushleft} }}} 
    & 
    \makecell[c]{\rotatebox[origin=l]{90}{\parbox{
    8
        cm}{\begin{flushleft}
        Seminar (BSc21, GymPO)
    \end{flushleft} }}} 
    \\[2ex] \hline 
    \rule[0mm]{0cm}{.6cm}Regelmäßige Teilnahme (wie in der Prüfungsordnung definiert). (SL) \rule[-3mm]{0cm}{0cm}
    &
    \makecell[c]{\xmark}
    &
    \makecell[c]{\xmark}
    &
    \makecell[c]{\xmark}
    &
    \makecell[c]{\xmark}
    &
    \makecell[c]{\xmark}
    &
    \makecell[c]{\xmark}
    \\
    \rule[0mm]{0cm}{.6cm}Etwa 45- bis 90-minütiger Vortrag. (PL) \rule[-3mm]{0cm}{0cm}
    &
    \makecell[c]{\xmark}
    &
    \makecell[c]{\xmark}
    &
    &
    \makecell[c]{\xmark}
    &
    &
    \makecell[c]{\xmark}
    \\
    \rule[0mm]{0cm}{.6cm}Für das absolvierte Modul (oder ggf. den Teil des Moduls) gibt es 6 ECTS-Punkte. (Kommentar) \rule[-3mm]{0cm}{0cm}
    &
    &
    \makecell[c]{\xmark}
    &
    \makecell[c]{\xmark}
    &
    &
    &
    \\
    \rule[0mm]{0cm}{.6cm}Verwendbar für die Option "Individuelle Schwerpunktgestaltung". (Kommentar) \rule[-3mm]{0cm}{0cm}
    &
    &
    &
    \makecell[c]{\xmark}
    &
    &
    &
    \\
\end{tabularx}

\clearpage\hrule\vskip1pt\hrule 
\section*{\Large Seminar: Numerik partieller Differentialgleichungen}
\addcontentsline{toc}{subsection}{Seminar: Numerik partieller Differentialgleichungen\ \textcolor{gray}{(\em Sören Bartels)}}
\vskip-2ex  
Sören Bartels\\
Raum und Zeit: Mi, 14--16, SR 226, \href{https://www.openstreetmap.org/?mlat=48.00351\&mlon=7.84815\#map=19/48.00351/7.84815}{Hermann-Herder-Str. 10}\\
\subsubsection*{\Large Inhalt:}
Est magnam quisquam modi velit labore labore. Etincidunt velit dolor neque quisquam adipisci dolore. Dolore numquam magnam ut ipsum porro modi. Dolorem ipsum ipsum quaerat neque aliquam. Consectetur dolor quisquam labore non consectetur modi magnam.
\subsubsection*{\Large Literatur:}
Est ipsum quiquia dolor neque non sed consectetur.
\subsubsection*{\Large Vorkenntnisse:}
Man erhält so leicht, dass $x_{1/2} = \frac{-b \pm \sqrt{b^2 - 4ac}}{2a}$
\subsubsection*{\Large Verwendbarkeit, Studien- und Prüfungsleistungen:}
\begin{tabularx}{\textwidth}{ p{.5\textwidth}
    X
    X
    X
    X
    X
    X
    }
    & 
    \makecell[c]{\rotatebox[origin=l]{90}{\parbox{
    8
        cm}{\begin{flushleft}
        Wahlmodul (BSc, MSc, BSc21, 2HfB21, 2HfB)
    \end{flushleft} }}} 
    & 
    \makecell[c]{\rotatebox[origin=l]{90}{\parbox{
    8
        cm}{\begin{flushleft}
        Mathematische Ergänzung (MEd)
    \end{flushleft} }}} 
    & 
    \makecell[c]{\rotatebox[origin=l]{90}{\parbox{
    8
        cm}{\begin{flushleft}
        Mathematische Seminar A oder B (MSc)
    \end{flushleft} }}} 
    & 
    \makecell[c]{\rotatebox[origin=l]{90}{\parbox{
    8
        cm}{\begin{flushleft}
        Modul im Wahlpflichtbereich Mathematik (BSc, BSc21)
    \end{flushleft} }}} 
    & 
    \makecell[c]{\rotatebox[origin=l]{90}{\parbox{
    8
        cm}{\begin{flushleft}
        Bachelor-Seminar (Teil des Bachelor-Moduls) (BSc)
    \end{flushleft} }}} 
    & 
    \makecell[c]{\rotatebox[origin=l]{90}{\parbox{
    8
        cm}{\begin{flushleft}
        Seminar (BSc21, GymPO)
    \end{flushleft} }}} 
    \\[2ex] \hline 
    \rule[0mm]{0cm}{.6cm}Verwendbar für die Option "Individuelle Schwerpunktgestaltung". (Kommentar) \rule[-3mm]{0cm}{0cm}
    &
    \makecell[c]{\xmark}
    &
    &
    &
    &
    &
    \\
    \rule[0mm]{0cm}{.6cm}Für das absolvierte Modul (oder ggf. den Teil des Moduls) gibt es 6 ECTS-Punkte. (Kommentar) \rule[-3mm]{0cm}{0cm}
    &
    \makecell[c]{\xmark}
    &
    &
    &
    \makecell[c]{\xmark}
    &
    &
    \\
    \rule[0mm]{0cm}{.6cm}Regelmäßige Teilnahme (wie in der Prüfungsordnung definiert). (SL) \rule[-3mm]{0cm}{0cm}
    &
    \makecell[c]{\xmark}
    &
    \makecell[c]{\xmark}
    &
    \makecell[c]{\xmark}
    &
    \makecell[c]{\xmark}
    &
    \makecell[c]{\xmark}
    &
    \makecell[c]{\xmark}
    \\
    \rule[0mm]{0cm}{.6cm}Etwa 45- bis 90-minütiger Vortrag. (PL) \rule[-3mm]{0cm}{0cm}
    &
    &
    &
    \makecell[c]{\xmark}
    &
    \makecell[c]{\xmark}
    &
    \makecell[c]{\xmark}
    &
    \makecell[c]{\xmark}
    \\
\end{tabularx}

\clearpage\hrule\vskip1pt\hrule 
\section*{\Large Seminar: Uniforme zentrale Grenzwertsätze für stochastische Prozesse (Uniform central limit theorems for stochastic processes)}
\addcontentsline{toc}{subsection}{Seminar: Uniforme zentrale Grenzwertsätze für stochastische Prozesse (Uniform central limit theorems for stochastic processes)\ \textcolor{gray}{(\em Angelika Rohde)}}
\vskip-2ex  
Angelika Rohde, Assistenz: Johannes Brutsche\\
\subsubsection*{\Large Inhalt:}
Voluptatem etincidunt porro magnam dolor quisquam. Porro voluptatem velit porro tempora. Quisquam non adipisci sit est quiquia ut quiquia. Sit porro est dolorem labore labore numquam adipisci. Modi sit neque numquam adipisci.
\subsubsection*{\Large Literatur:}
Adipisci quaerat eius dolorem quisquam consectetur.
\subsubsection*{\Large Vorkenntnisse:}
Man erhält so leicht, dass $x_{1/2} = \frac{-b \pm \sqrt{b^2 - 4ac}}{2a}$
\subsubsection*{\Large Verwendbarkeit, Studien- und Prüfungsleistungen:}
\begin{tabularx}{\textwidth}{ p{.5\textwidth}
    X
    X
    X
    X
    X
    X
    }
    & 
    \makecell[c]{\rotatebox[origin=l]{90}{\parbox{
    8
        cm}{\begin{flushleft}
        Mathematische Seminar A oder B (MSc)
    \end{flushleft} }}} 
    & 
    \makecell[c]{\rotatebox[origin=l]{90}{\parbox{
    8
        cm}{\begin{flushleft}
        Wahlmodul (BSc, MSc, BSc21, 2HfB21, 2HfB)
    \end{flushleft} }}} 
    & 
    \makecell[c]{\rotatebox[origin=l]{90}{\parbox{
    8
        cm}{\begin{flushleft}
        Mathematische Ergänzung (MEd)
    \end{flushleft} }}} 
    & 
    \makecell[c]{\rotatebox[origin=l]{90}{\parbox{
    8
        cm}{\begin{flushleft}
        Seminar (BSc21, GymPO)
    \end{flushleft} }}} 
    & 
    \makecell[c]{\rotatebox[origin=l]{90}{\parbox{
    8
        cm}{\begin{flushleft}
        Modul im Wahlpflichtbereich Mathematik (BSc, BSc21)
    \end{flushleft} }}} 
    & 
    \makecell[c]{\rotatebox[origin=l]{90}{\parbox{
    8
        cm}{\begin{flushleft}
        Bachelor-Seminar (Teil des Bachelor-Moduls) (BSc)
    \end{flushleft} }}} 
    \\[2ex] \hline 
    \rule[0mm]{0cm}{.6cm}Regelmäßige Teilnahme (wie in der Prüfungsordnung definiert). (SL) \rule[-3mm]{0cm}{0cm}
    &
    \makecell[c]{\xmark}
    &
    \makecell[c]{\xmark}
    &
    \makecell[c]{\xmark}
    &
    \makecell[c]{\xmark}
    &
    \makecell[c]{\xmark}
    &
    \makecell[c]{\xmark}
    \\
    \rule[0mm]{0cm}{.6cm}Etwa 45- bis 90-minütiger Vortrag. (PL) \rule[-3mm]{0cm}{0cm}
    &
    \makecell[c]{\xmark}
    &
    &
    &
    \makecell[c]{\xmark}
    &
    \makecell[c]{\xmark}
    &
    \makecell[c]{\xmark}
    \\
    \rule[0mm]{0cm}{.6cm}Für das absolvierte Modul (oder ggf. den Teil des Moduls) gibt es 6 ECTS-Punkte. (Kommentar) \rule[-3mm]{0cm}{0cm}
    &
    &
    \makecell[c]{\xmark}
    &
    &
    &
    \makecell[c]{\xmark}
    &
    \\
    \rule[0mm]{0cm}{.6cm}Verwendbar für die Option "Individuelle Schwerpunktgestaltung". (Kommentar) \rule[-3mm]{0cm}{0cm}
    &
    &
    \makecell[c]{\xmark}
    &
    &
    &
    &
    \\
\end{tabularx}

\clearpage\hrule\vskip1pt\hrule 
\section*{\Large Seminar: Prikry-Forcing}
\addcontentsline{toc}{subsection}{Seminar: Prikry-Forcing\ \textcolor{gray}{(\em Heike Mildenberger)}}
\vskip-2ex  
Heike Mildenberger, Assistenz: Hannes Jakob\\
Raum und Zeit: Mo, 10--12, SR 125, \href{https://www.openstreetmap.org/?mlat=48.00065\&mlon=7.84591\#map=19/48.00065/7.84591}{Ernst-Zermelo-Straße 1}\\
\subsubsection*{\Large Inhalt:}
Etincidunt consectetur dolore porro. Magnam ipsum etincidunt velit adipisci dolor numquam. Sed dolorem aliquam ut. Quisquam numquam dolorem velit eius amet velit. Quisquam ut dolorem eius consectetur. Etincidunt consectetur ipsum amet. Porro dolor quiquia ipsum. Neque sed consectetur quaerat etincidunt non. Dolorem quisquam sed amet quisquam.
\subsubsection*{\Large Literatur:}
Dolor tempora non dolorem.
\subsubsection*{\Large Vorkenntnisse:}
Man erhält so leicht, dass $x_{1/2} = \frac{-b \pm \sqrt{b^2 - 4ac}}{2a}$
\subsubsection*{\Large Verwendbarkeit, Studien- und Prüfungsleistungen:}
\begin{tabularx}{\textwidth}{ p{.5\textwidth}
    X
    X
    X
    X
    X
    X
    }
    & 
    \makecell[c]{\rotatebox[origin=l]{90}{\parbox{
    8
        cm}{\begin{flushleft}
        Mathematische Ergänzung (MEd)
    \end{flushleft} }}} 
    & 
    \makecell[c]{\rotatebox[origin=l]{90}{\parbox{
    8
        cm}{\begin{flushleft}
        Bachelor-Seminar (Teil des Bachelor-Moduls) (BSc)
    \end{flushleft} }}} 
    & 
    \makecell[c]{\rotatebox[origin=l]{90}{\parbox{
    8
        cm}{\begin{flushleft}
        Modul im Wahlpflichtbereich Mathematik (BSc, BSc21)
    \end{flushleft} }}} 
    & 
    \makecell[c]{\rotatebox[origin=l]{90}{\parbox{
    8
        cm}{\begin{flushleft}
        Seminar (BSc21, GymPO)
    \end{flushleft} }}} 
    & 
    \makecell[c]{\rotatebox[origin=l]{90}{\parbox{
    8
        cm}{\begin{flushleft}
        Wahlmodul (BSc, MSc, BSc21, 2HfB21, 2HfB)
    \end{flushleft} }}} 
    & 
    \makecell[c]{\rotatebox[origin=l]{90}{\parbox{
    8
        cm}{\begin{flushleft}
        Mathematische Seminar A oder B (MSc)
    \end{flushleft} }}} 
    \\[2ex] \hline 
    \rule[0mm]{0cm}{.6cm}Regelmäßige Teilnahme (wie in der Prüfungsordnung definiert). (SL) \rule[-3mm]{0cm}{0cm}
    &
    \makecell[c]{\xmark}
    &
    \makecell[c]{\xmark}
    &
    \makecell[c]{\xmark}
    &
    \makecell[c]{\xmark}
    &
    \makecell[c]{\xmark}
    &
    \makecell[c]{\xmark}
    \\
    \rule[0mm]{0cm}{.6cm}Etwa 45- bis 90-minütiger Vortrag. (PL) \rule[-3mm]{0cm}{0cm}
    &
    &
    \makecell[c]{\xmark}
    &
    \makecell[c]{\xmark}
    &
    \makecell[c]{\xmark}
    &
    &
    \makecell[c]{\xmark}
    \\
    \rule[0mm]{0cm}{.6cm}Für das absolvierte Modul (oder ggf. den Teil des Moduls) gibt es 6 ECTS-Punkte. (Kommentar) \rule[-3mm]{0cm}{0cm}
    &
    &
    &
    \makecell[c]{\xmark}
    &
    &
    \makecell[c]{\xmark}
    &
    \\
    \rule[0mm]{0cm}{.6cm}Verwendbar für die Option "Individuelle Schwerpunktgestaltung". (Kommentar) \rule[-3mm]{0cm}{0cm}
    &
    &
    &
    &
    &
    \makecell[c]{\xmark}
    &
    \\
\end{tabularx}

\clearpage\hrule\vskip1pt\hrule 
\section*{\Large Seminar: Funktionenkörper}
\addcontentsline{toc}{subsection}{Seminar: Funktionenkörper\ \textcolor{gray}{(\em Andreas Demleitner)}}
\vskip-2ex  
Andreas Demleitner, Assistenz: Andreas Demleitner\\
Raum und Zeit: Do, 14--16, SR 404, \href{https://www.openstreetmap.org/?mlat=48.00065\&mlon=7.84591\#map=19/48.00065/7.84591}{Ernst-Zermelo-Straße 1}\\
\subsubsection*{\Large Inhalt:}
Quaerat quiquia dolor sed. Neque tempora dolore aliquam etincidunt quaerat numquam. Ut aliquam etincidunt velit velit dolorem magnam. Sit ipsum neque dolore etincidunt eius etincidunt. Ut quaerat labore quiquia. Aliquam quisquam magnam adipisci etincidunt ut. Adipisci magnam adipisci ipsum velit dolore.
\subsubsection*{\Large Literatur:}
Labore est sed aliquam quiquia eius dolore.
\subsubsection*{\Large Vorkenntnisse:}
Man erhält so leicht, dass $x_{1/2} = \frac{-b \pm \sqrt{b^2 - 4ac}}{2a}$
\subsubsection*{\Large Verwendbarkeit, Studien- und Prüfungsleistungen:}
\begin{tabularx}{\textwidth}{ p{.5\textwidth}
    X
    X
    X
    X
    X
    X
    }
    & 
    \makecell[c]{\rotatebox[origin=l]{90}{\parbox{
    8
        cm}{\begin{flushleft}
        Seminar (BSc21, GymPO)
    \end{flushleft} }}} 
    & 
    \makecell[c]{\rotatebox[origin=l]{90}{\parbox{
    8
        cm}{\begin{flushleft}
        Modul im Wahlpflichtbereich Mathematik (BSc, BSc21)
    \end{flushleft} }}} 
    & 
    \makecell[c]{\rotatebox[origin=l]{90}{\parbox{
    8
        cm}{\begin{flushleft}
        Bachelor-Seminar (Teil des Bachelor-Moduls) (BSc)
    \end{flushleft} }}} 
    & 
    \makecell[c]{\rotatebox[origin=l]{90}{\parbox{
    8
        cm}{\begin{flushleft}
        Mathematische Ergänzung (MEd)
    \end{flushleft} }}} 
    & 
    \makecell[c]{\rotatebox[origin=l]{90}{\parbox{
    8
        cm}{\begin{flushleft}
        Wahlmodul (BSc, MSc, BSc21, 2HfB21, 2HfB)
    \end{flushleft} }}} 
    & 
    \makecell[c]{\rotatebox[origin=l]{90}{\parbox{
    8
        cm}{\begin{flushleft}
        Mathematische Seminar A oder B (MSc)
    \end{flushleft} }}} 
    \\[2ex] \hline 
    \rule[0mm]{0cm}{.6cm}Etwa 45- bis 90-minütiger Vortrag. (PL) \rule[-3mm]{0cm}{0cm}
    &
    \makecell[c]{\xmark}
    &
    \makecell[c]{\xmark}
    &
    \makecell[c]{\xmark}
    &
    &
    &
    \makecell[c]{\xmark}
    \\
    \rule[0mm]{0cm}{.6cm}Regelmäßige Teilnahme (wie in der Prüfungsordnung definiert). (SL) \rule[-3mm]{0cm}{0cm}
    &
    \makecell[c]{\xmark}
    &
    \makecell[c]{\xmark}
    &
    \makecell[c]{\xmark}
    &
    \makecell[c]{\xmark}
    &
    \makecell[c]{\xmark}
    &
    \makecell[c]{\xmark}
    \\
    \rule[0mm]{0cm}{.6cm}Für das absolvierte Modul (oder ggf. den Teil des Moduls) gibt es 6 ECTS-Punkte. (Kommentar) \rule[-3mm]{0cm}{0cm}
    &
    &
    \makecell[c]{\xmark}
    &
    &
    &
    \makecell[c]{\xmark}
    &
    \\
    \rule[0mm]{0cm}{.6cm}Verwendbar für die Option "Individuelle Schwerpunktgestaltung". (Kommentar) \rule[-3mm]{0cm}{0cm}
    &
    &
    &
    &
    &
    \makecell[c]{\xmark}
    &
    \\
\end{tabularx}

\clearpage\hrule\vskip1pt\hrule 
\section*{\Large Seminar: Die Mathematik und das Göttliche}
\addcontentsline{toc}{subsection}{Seminar: Die Mathematik und das Göttliche\ \textcolor{gray}{(\em Andreas Henn, Markus Junker)}}
\vskip-2ex  
Andreas Henn, Markus Junker\\
Raum und Zeit: Di, 16--18, R 206, \href{https://www.openstreetmap.org/?mlat=47.99252\&mlon=7.84775\#map=19/47.99252/7.84775}{Breisacher Tor}\\
\subsubsection*{\Large Inhalt:}
Modi dolor modi modi velit. Numquam amet etincidunt modi sit. Labore magnam modi etincidunt est. Dolorem tempora magnam voluptatem est quaerat. Adipisci quaerat neque labore numquam consectetur.
\subsubsection*{\Large Literatur:}
Quaerat magnam consectetur etincidunt ut labore sit est.
\subsubsection*{\Large Vorkenntnisse:}
Man erhält so leicht, dass $x_{1/2} = \frac{-b \pm \sqrt{b^2 - 4ac}}{2a}$
\subsubsection*{\Large Verwendbarkeit, Studien- und Prüfungsleistungen:}
\begin{tabularx}{\textwidth}{ p{.5\textwidth}
    X
    X
    X
    X
    X
    X
    }
    & 
    \makecell[c]{\rotatebox[origin=l]{90}{\parbox{
    8
        cm}{\begin{flushleft}
        Wahlmodul (BSc, MSc, BSc21, 2HfB21, 2HfB)
    \end{flushleft} }}} 
    & 
    \makecell[c]{\rotatebox[origin=l]{90}{\parbox{
    8
        cm}{\begin{flushleft}
        Mathematische Ergänzung (MEd)
    \end{flushleft} }}} 
    & 
    \makecell[c]{\rotatebox[origin=l]{90}{\parbox{
    8
        cm}{\begin{flushleft}
        Modul im Wahlpflichtbereich Mathematik (BSc, BSc21)
    \end{flushleft} }}} 
    & 
    \makecell[c]{\rotatebox[origin=l]{90}{\parbox{
    8
        cm}{\begin{flushleft}
        Bachelor-Seminar (Teil des Bachelor-Moduls) (BSc)
    \end{flushleft} }}} 
    & 
    \makecell[c]{\rotatebox[origin=l]{90}{\parbox{
    8
        cm}{\begin{flushleft}
        Mathematische Seminar A oder B (MSc)
    \end{flushleft} }}} 
    & 
    \makecell[c]{\rotatebox[origin=l]{90}{\parbox{
    8
        cm}{\begin{flushleft}
        Seminar (BSc21, GymPO)
    \end{flushleft} }}} 
    \\[2ex] \hline 
    \rule[0mm]{0cm}{.6cm}Für das absolvierte Modul (oder ggf. den Teil des Moduls) gibt es 6 ECTS-Punkte. (Kommentar) \rule[-3mm]{0cm}{0cm}
    &
    \makecell[c]{\xmark}
    &
    &
    \makecell[c]{\xmark}
    &
    &
    &
    \\
    \rule[0mm]{0cm}{.6cm}Regelmäßige Teilnahme (wie in der Prüfungsordnung definiert). (SL) \rule[-3mm]{0cm}{0cm}
    &
    \makecell[c]{\xmark}
    &
    \makecell[c]{\xmark}
    &
    \makecell[c]{\xmark}
    &
    \makecell[c]{\xmark}
    &
    \makecell[c]{\xmark}
    &
    \makecell[c]{\xmark}
    \\
    \rule[0mm]{0cm}{.6cm}Verwendbar für die Option "Individuelle Schwerpunktgestaltung". (Kommentar) \rule[-3mm]{0cm}{0cm}
    &
    \makecell[c]{\xmark}
    &
    &
    &
    &
    &
    \\
    \rule[0mm]{0cm}{.6cm}Etwa 45- bis 90-minütiger Vortrag. (PL) \rule[-3mm]{0cm}{0cm}
    &
    &
    &
    \makecell[c]{\xmark}
    &
    \makecell[c]{\xmark}
    &
    \makecell[c]{\xmark}
    &
    \makecell[c]{\xmark}
    \\
\end{tabularx}

\clearpage\hrule\vskip1pt\hrule 
\section*{\Large Seminar: Die Geometrie von Blätterungen}
\addcontentsline{toc}{subsection}{Seminar: Die Geometrie von Blätterungen\ \textcolor{gray}{(\em Christian Ketterer, Jonas Schnitzer)}}
\vskip-2ex  
Christian Ketterer, Jonas Schnitzer, Assistenz: Christian Ketterer, Jonas Schnitzer\\
Raum und Zeit: Mi, 10--12, SR 125, \href{https://www.openstreetmap.org/?mlat=48.00065\&mlon=7.84591\#map=19/48.00065/7.84591}{Ernst-Zermelo-Straße 1}\\
\subsubsection*{\Large Inhalt:}
Adipisci voluptatem neque sit numquam dolorem numquam consectetur. Porro ut voluptatem consectetur quisquam. Etincidunt dolor ut velit eius consectetur adipisci sed. Velit etincidunt ipsum dolore numquam dolorem quisquam. Numquam sit neque voluptatem adipisci. Magnam dolor consectetur ut dolore consectetur modi. Ut modi magnam ipsum sit dolore. Etincidunt quiquia quiquia velit velit. Sit eius tempora velit. Labore quaerat sed magnam porro dolor.
\subsubsection*{\Large Literatur:}
Ut non amet ipsum aliquam eius quaerat.
\subsubsection*{\Large Vorkenntnisse:}
Man erhält so leicht, dass $x_{1/2} = \frac{-b \pm \sqrt{b^2 - 4ac}}{2a}$
\subsubsection*{\Large Verwendbarkeit, Studien- und Prüfungsleistungen:}
\begin{tabularx}{\textwidth}{ p{.5\textwidth}
    }
    \\[2ex] \hline 
\end{tabularx}

\clearpage\hrule\vskip1pt\hrule 
\section*{\Large Seminar: Einführung in Maschinelles Lernen}
\addcontentsline{toc}{subsection}{Seminar: Einführung in Maschinelles Lernen\ \textcolor{gray}{(\em Michael Böhler)}}
\vskip-2ex  
Michael Böhler\\
Fr 14-16: Fr, 14--16, SR II, Physik Hochhaus\\
\subsubsection*{\Large Inhalt:}
Dolor dolor velit neque tempora dolor quaerat. Eius est ipsum quisquam. Neque aliquam numquam magnam consectetur dolore. Aliquam dolor porro tempora dolorem consectetur. Est aliquam velit eius eius voluptatem. Magnam ipsum modi numquam magnam est modi sed. Aliquam adipisci sed quaerat aliquam.
\subsubsection*{\Large Literatur:}
Numquam eius numquam sed etincidunt tempora.
\subsubsection*{\Large Vorkenntnisse:}
Man erhält so leicht, dass $x_{1/2} = \frac{-b \pm \sqrt{b^2 - 4ac}}{2a}$
\subsubsection*{\Large Verwendbarkeit, Studien- und Prüfungsleistungen:}
\begin{tabularx}{\textwidth}{ p{.5\textwidth}
    }
    \\[2ex] \hline 
\end{tabularx}

\clearpage
\phantomsection
\thispagestyle{empty}
\vspace*{\fill}
\begin{center}
\Huge\bfseries 4b. Oberseminare
\end{center}
\addcontentsline{toc}{section}{\textbf{4b. Oberseminare}}
\addtocontents{toc}{\medskip\hrule\medskip}\vspace*{\fill}\vspace*{\fill}\clearpage
\vfill
\thispagestyle{empty}
\clearpage

\clearpage\hrule\vskip1pt\hrule 
\section*{\Large Oberseminar: Algebra, Zahlentheorie und algebraische Geometrie}
\addcontentsline{toc}{subsection}{Oberseminar: Algebra, Zahlentheorie und algebraische Geometrie\ \textcolor{gray}{(\em Annette Huber-Klawitter, Wolfgang Soergel)}}
\vskip-2ex  
Annette Huber-Klawitter, Wolfgang Soergel\\
Raum und Zeit: Fr, 10--12, SR 404, \href{https://www.openstreetmap.org/?mlat=48.00065\&mlon=7.84591\#map=19/48.00065/7.84591}{Ernst-Zermelo-Straße 1}\\
\subsubsection*{\Large Inhalt:}
Est tempora ipsum quisquam sed non dolor velit. Porro porro magnam quaerat modi porro amet amet. Ipsum consectetur tempora dolore. Consectetur eius magnam non tempora quaerat ipsum. Modi velit modi ut porro velit dolor quisquam. Porro quiquia eius ipsum modi. Etincidunt aliquam quiquia amet. Velit amet quisquam velit est sed etincidunt modi. Dolorem numquam non voluptatem amet sed sit non. Adipisci dolor velit quisquam labore ut.
\subsubsection*{\Large Literatur:}
Dolor quiquia modi aliquam consectetur aliquam modi.
\subsubsection*{\Large Vorkenntnisse:}
Man erhält so leicht, dass $x_{1/2} = \frac{-b \pm \sqrt{b^2 - 4ac}}{2a}$
\subsubsection*{\Large Verwendbarkeit, Studien- und Prüfungsleistungen:}
\begin{tabularx}{\textwidth}{ p{.5\textwidth}
    }
    \\[2ex] \hline 
\end{tabularx}

\clearpage\hrule\vskip1pt\hrule 
\section*{\Large Oberseminar: Angewandte Mathematik}
\addcontentsline{toc}{subsection}{Oberseminar: Angewandte Mathematik\ \textcolor{gray}{(\em Sören Bartels, Patrick Dondl, Michael Růžička, Diyora Salimova)}}
\vskip-2ex  
Sören Bartels, Patrick Dondl, Michael Růžička, Diyora Salimova\\
Raum und Zeit: Di, 14--16, SR 226, \href{https://www.openstreetmap.org/?mlat=48.00351\&mlon=7.84815\#map=19/48.00351/7.84815}{Hermann-Herder-Str. 10}\\
\subsubsection*{\Large Inhalt:}
Labore sit tempora sit sit ipsum magnam. Sit etincidunt dolorem non. Neque dolor dolorem etincidunt etincidunt. Ipsum ut etincidunt quaerat quiquia quiquia amet. Magnam velit aliquam est sit. Neque eius adipisci etincidunt. Numquam amet neque sit ipsum est velit consectetur. Sit amet dolorem dolor modi. Magnam velit quaerat velit numquam magnam. Porro magnam sed non dolore dolorem dolore.
\subsubsection*{\Large Literatur:}
Voluptatem modi magnam tempora sed ipsum ut ipsum.
\subsubsection*{\Large Vorkenntnisse:}
Man erhält so leicht, dass $x_{1/2} = \frac{-b \pm \sqrt{b^2 - 4ac}}{2a}$
\subsubsection*{\Large Verwendbarkeit, Studien- und Prüfungsleistungen:}
\begin{tabularx}{\textwidth}{ p{.5\textwidth}
    }
    \\[2ex] \hline 
\end{tabularx}

\clearpage\hrule\vskip1pt\hrule 
\section*{\Large Oberseminar: Differentialgeometrie}
\addcontentsline{toc}{subsection}{Oberseminar: Differentialgeometrie\ \textcolor{gray}{(\em Sebastian Goette, Nadine Große, Christian Ketterer)}}
\vskip-2ex  
Sebastian Goette, Nadine Große, Christian Ketterer\\
Raum und Zeit: Mo, 16--18, SR 125, \href{https://www.openstreetmap.org/?mlat=48.00065\&mlon=7.84591\#map=19/48.00065/7.84591}{Ernst-Zermelo-Straße 1}\\
\subsubsection*{\Large Inhalt:}
Adipisci amet numquam quiquia labore eius quaerat dolor. Velit tempora ut quisquam amet dolor dolore. Etincidunt quaerat dolor non modi. Aliquam numquam ipsum aliquam aliquam labore magnam tempora. Dolore porro neque adipisci eius. Dolore dolorem adipisci eius ut magnam. Ut quaerat sit dolor.
\subsubsection*{\Large Literatur:}
Ipsum dolore dolorem dolorem.
\subsubsection*{\Large Vorkenntnisse:}
Man erhält so leicht, dass $x_{1/2} = \frac{-b \pm \sqrt{b^2 - 4ac}}{2a}$
\subsubsection*{\Large Verwendbarkeit, Studien- und Prüfungsleistungen:}
\begin{tabularx}{\textwidth}{ p{.5\textwidth}
    }
    \\[2ex] \hline 
\end{tabularx}

\clearpage\hrule\vskip1pt\hrule 
\section*{\Large Oberseminar: Mathematische Logik}
\addcontentsline{toc}{subsection}{Oberseminar: Mathematische Logik\ \textcolor{gray}{(\em Amador Martín Pizarro, Heike Mildenberger)}}
\vskip-2ex  
Amador Martín Pizarro, Heike Mildenberger\\
Raum und Zeit: Di, 14: 30--16, SR 404, \href{https://www.openstreetmap.org/?mlat=48.00065\&mlon=7.84591\#map=19/48.00065/7.84591}{Ernst-Zermelo-Straße 1}\\
\subsubsection*{\Large Inhalt:}
Consectetur neque consectetur quiquia. Adipisci ipsum magnam est eius dolorem. Velit dolore porro velit adipisci velit. Quiquia ipsum quisquam aliquam. Ipsum quiquia non magnam. Modi dolorem quisquam ipsum. Ipsum amet aliquam consectetur porro modi ut adipisci. Quaerat sit eius modi magnam quisquam. Quiquia sed etincidunt etincidunt ipsum consectetur consectetur.
\subsubsection*{\Large Literatur:}
Tempora dolor est voluptatem sit quaerat.
\subsubsection*{\Large Vorkenntnisse:}
Man erhält so leicht, dass $x_{1/2} = \frac{-b \pm \sqrt{b^2 - 4ac}}{2a}$
\subsubsection*{\Large Verwendbarkeit, Studien- und Prüfungsleistungen:}
\begin{tabularx}{\textwidth}{ p{.5\textwidth}
    }
    \\[2ex] \hline 
\end{tabularx}

\clearpage\hrule\vskip1pt\hrule 
\section*{\Large Oberseminar: Stochastik}
\addcontentsline{toc}{subsection}{Oberseminar: Stochastik\ \textcolor{gray}{(\em David Criens, Peter Pfaffelhuber, Angelika Rohde, Thorsten Schmidt)}}
\vskip-2ex  
David Criens, Peter Pfaffelhuber, Angelika Rohde, Thorsten Schmidt\\
Raum und Zeit: Mi, 16--17, HS II, \href{https://www.openstreetmap.org/?mlat=48.00233\&mlon=7.84788\#map=19/48.00233/7.84788}{Albertstr. 23b}\\
\subsubsection*{\Large Inhalt:}
Neque quiquia adipisci labore amet dolor. Neque velit etincidunt magnam neque quisquam. Modi modi modi numquam. Modi ipsum dolor tempora adipisci quaerat consectetur. Tempora modi adipisci quisquam neque voluptatem magnam dolorem. Quiquia etincidunt labore neque non. Adipisci quiquia labore adipisci dolor velit dolor eius.
\subsubsection*{\Large Literatur:}
Porro tempora velit ipsum.
\subsubsection*{\Large Vorkenntnisse:}
Man erhält so leicht, dass $x_{1/2} = \frac{-b \pm \sqrt{b^2 - 4ac}}{2a}$
\subsubsection*{\Large Verwendbarkeit, Studien- und Prüfungsleistungen:}
\begin{tabularx}{\textwidth}{ p{.5\textwidth}
    }
    \\[2ex] \hline 
\end{tabularx}

\clearpage\hrule\vskip1pt\hrule 
\section*{\Large Oberseminar: Medizinische Statistik}
\addcontentsline{toc}{subsection}{Oberseminar: Medizinische Statistik\ \textcolor{gray}{(\em Harald Binder)}}
\vskip-2ex  
Harald Binder\\
Raum und Zeit: Mi, 11: 30--13, HS Medizinische Biometrie, \href{https://www.openstreetmap.org/?mlat=48.00248\&mlon=7.84681\#map=19/48.00248/7.84681}{Stefan-Meier-Str. 26}\\
\subsubsection*{\Large Inhalt:}
Amet magnam quisquam quisquam non consectetur. Adipisci dolorem velit magnam modi etincidunt. Sed dolorem dolorem quiquia magnam sed. Dolore magnam quisquam numquam consectetur eius. Dolorem ipsum velit adipisci voluptatem quiquia. Numquam consectetur modi dolor voluptatem dolore sed. Adipisci eius neque eius sit tempora est aliquam. Adipisci est voluptatem consectetur etincidunt velit eius modi. Sit eius non labore dolorem.
\subsubsection*{\Large Literatur:}
Amet ut velit quaerat ut tempora dolor dolorem.
\subsubsection*{\Large Vorkenntnisse:}
Man erhält so leicht, dass $x_{1/2} = \frac{-b \pm \sqrt{b^2 - 4ac}}{2a}$
\subsubsection*{\Large Verwendbarkeit, Studien- und Prüfungsleistungen:}
\begin{tabularx}{\textwidth}{ p{.5\textwidth}
    }
    \\[2ex] \hline 
\end{tabularx}

\clearpage
\phantomsection
\thispagestyle{empty}
\vspace*{\fill}
\begin{center}
\Huge\bfseries 4a. Projektseminare und Lesekurse
\end{center}
\addcontentsline{toc}{section}{\textbf{4a. Projektseminare und Lesekurse}}
\addtocontents{toc}{\medskip\hrule\medskip}\vspace*{\fill}\vspace*{\fill}\clearpage
\vfill
\thispagestyle{empty}
\clearpage

\clearpage\hrule\vskip1pt\hrule 
\section*{\Large Lesekurse „Wissenschaftliches Arbeiten“}
\addcontentsline{toc}{subsection}{Lesekurse „Wissenschaftliches Arbeiten“\ \textcolor{gray}{(\em Alle Professor:inn:en und Privatdozent:inn:en des Mathematischen Instituts)}}
\vskip-2ex  
Alle Professor:inn:en und Privatdozent:inn:en des Mathematischen Instituts\\
\subsubsection*{\Large Inhalt:}
Voluptatem dolore dolorem magnam numquam consectetur. Non sit dolore est numquam adipisci quisquam consectetur. Sit aliquam eius adipisci. Amet eius porro dolore velit non labore. Sed eius sit adipisci.
\subsubsection*{\Large Literatur:}
Amet quisquam quiquia quaerat ipsum aliquam.
\subsubsection*{\Large Vorkenntnisse:}
Man erhält so leicht, dass $x_{1/2} = \frac{-b \pm \sqrt{b^2 - 4ac}}{2a}$
\subsubsection*{\Large Verwendbarkeit, Studien- und Prüfungsleistungen:}
\begin{tabularx}{\textwidth}{ p{.5\textwidth}
    X
    X
    X
    }
    & 
    \makecell[c]{\rotatebox[origin=l]{90}{\parbox{
    4
        cm}{\begin{flushleft}
        Wissenschaftliches Arbeiten (MEd, MEH21)
    \end{flushleft} }}} 
    & 
    \makecell[c]{\rotatebox[origin=l]{90}{\parbox{
    4
        cm}{\begin{flushleft}
        Wahlmodul (BSc, MSc, BSc21, 2HfB21, 2HfB)
    \end{flushleft} }}} 
    & 
    \makecell[c]{\rotatebox[origin=l]{90}{\parbox{
    4
        cm}{\begin{flushleft}
        Mathematik oder Teil des Vertiefungsmoduls (MSc)
    \end{flushleft} }}} 
    \\[2ex] \hline 
    \rule[0mm]{0cm}{.6cm}Mündliche Prüfung (Dauer: ca. 30 Minuten). (PL) \rule[-3mm]{0cm}{0cm}
    &
    \makecell[c]{\xmark}
    &
    &
    \\
    \rule[0mm]{0cm}{.6cm}Selbständige Lektüre der von dem Betreuer/der Betreuerin vorgegebenen Skripte, Artikel oder Buchkapitel und ggf. Bearbeitung von begleitenden Übungsaufgaben. Regelmäßiger Bericht über den Fortschritt des Selbststudiums mit der Formulierung von Fragen zu nicht verstandenen Punkten. Bis zu zweimaliges Vortragen vor der Arbeitsgruppe über den bisher erarbeiten Stoff, ggf. im Rahmen eines Seminars, Projekt- oder Oberseminars.
Falls das Wissenschaftliche Arbeiten im Rahmen einer Lehrveranstaltung (z.B. Seminar oder Projektseminar) stattfindet: regelmäßige Teilnahme an dieser Veranstaltung. (SL) \rule[-3mm]{0cm}{0cm}
    &
    \makecell[c]{\xmark}
    &
    \makecell[c]{\xmark}
    &
    \makecell[c]{\xmark}
    \\
    \rule[0mm]{0cm}{.6cm}Für das absolvierte Modul (oder ggf. den Teil des Moduls) gibt es 9 ECTS-Punkte. (Kommentar) \rule[-3mm]{0cm}{0cm}
    &
    &
    \makecell[c]{\xmark}
    &
    \\
    \rule[0mm]{0cm}{.6cm}Die Zusammensetzung des Vertiefungsmoduls muss mit dem Prüfer/der Prüferin zuvor abgesprochen sein. Nicht alle Kombinationen sind zulässig. Die absolvierte Studienleistung in dieser Veranstaltung zählt mit 9 ECTS-Punkten in das Vertiefungsmodul. (Kommentar) \rule[-3mm]{0cm}{0cm}
    &
    &
    &
    \makecell[c]{\xmark}
    \\
    \rule[0mm]{0cm}{.6cm}Mündliche Prüfung über alle Teile des Moduls (Dauer: ca. 30 Minuten, im Vertiefungsmodul ca. 45 Minuten) (PL) \rule[-3mm]{0cm}{0cm}
    &
    &
    &
    \makecell[c]{\xmark}
    \\
\end{tabularx}

\clearpage\hrule\vskip1pt\hrule 
\section*{\Large Projektseminar: Nicht-Newtonsche Flüssigkeiten}
\addcontentsline{toc}{subsection}{Projektseminar: Nicht-Newtonsche Flüssigkeiten\ \textcolor{gray}{(\em Michael Růžička)}}
\vskip-2ex  
Michael Růžička\\
Raum und Zeit: Fr, 10--12, SR 127, \href{https://www.openstreetmap.org/?mlat=48.00065\&mlon=7.84591\#map=19/48.00065/7.84591}{Ernst-Zermelo-Straße 1}\\
\subsubsection*{\Large Inhalt:}
Dolor etincidunt ipsum sed magnam tempora quisquam porro. Amet ut numquam labore. Ut est aliquam amet amet tempora eius etincidunt. Dolor amet non dolorem voluptatem adipisci. Amet labore consectetur etincidunt quaerat tempora. Non consectetur quiquia voluptatem dolor sit numquam est. Tempora amet adipisci porro. Modi quiquia modi amet tempora velit dolor. Eius aliquam magnam dolore.
\subsubsection*{\Large Literatur:}
Eius quisquam porro modi porro labore adipisci.
\subsubsection*{\Large Vorkenntnisse:}
Man erhält so leicht, dass $x_{1/2} = \frac{-b \pm \sqrt{b^2 - 4ac}}{2a}$
\subsubsection*{\Large Verwendbarkeit, Studien- und Prüfungsleistungen:}
\begin{tabularx}{\textwidth}{ p{.5\textwidth}
    }
    \\[2ex] \hline 
\end{tabularx}

\clearpage\hrule\vskip1pt\hrule 
\section*{\Large Projektseminar: Geometrische Analysis}
\addcontentsline{toc}{subsection}{Projektseminar: Geometrische Analysis\ \textcolor{gray}{(\em Ernst Kuwert, Guofang Wang)}}
\vskip-2ex  
Ernst Kuwert, Guofang Wang\\
Raum und Zeit: Di, 16--18, SR 127, \href{https://www.openstreetmap.org/?mlat=48.00065\&mlon=7.84591\#map=19/48.00065/7.84591}{Ernst-Zermelo-Straße 1}\\
\subsubsection*{\Large Inhalt:}
Sed sit aliquam est dolore velit. Ipsum consectetur consectetur modi modi aliquam voluptatem est. Quaerat est dolore dolorem quiquia eius velit adipisci. Sit etincidunt aliquam voluptatem aliquam quisquam aliquam. Ipsum ipsum amet adipisci dolor. Adipisci quaerat eius non ipsum tempora numquam eius. Aliquam dolor non etincidunt aliquam consectetur.
\subsubsection*{\Large Literatur:}
Non sed adipisci consectetur etincidunt modi ipsum labore.
\subsubsection*{\Large Vorkenntnisse:}
Man erhält so leicht, dass $x_{1/2} = \frac{-b \pm \sqrt{b^2 - 4ac}}{2a}$
\subsubsection*{\Large Verwendbarkeit, Studien- und Prüfungsleistungen:}
\begin{tabularx}{\textwidth}{ p{.5\textwidth}
    }
    \\[2ex] \hline 
\end{tabularx}

\clearpage\hrule\vskip1pt\hrule 
\section*{\Large Projektseminar: Numerische Analysis}
\addcontentsline{toc}{subsection}{Projektseminar: Numerische Analysis\ \textcolor{gray}{(\em Sören Bartels)}}
\vskip-2ex  
Sören Bartels\\
\subsubsection*{\Large Inhalt:}
Est etincidunt dolore neque voluptatem velit numquam dolore. Porro voluptatem dolorem dolore. Amet ipsum sed tempora consectetur quaerat. Quiquia numquam modi labore. Dolor ut consectetur etincidunt dolorem ut numquam. Dolore quiquia ipsum aliquam. Aliquam neque eius modi dolorem. Labore consectetur adipisci neque porro neque velit amet. Neque voluptatem modi ut neque non aliquam sit.
\subsubsection*{\Large Literatur:}
Voluptatem voluptatem non adipisci tempora dolore ipsum sed.
\subsubsection*{\Large Vorkenntnisse:}
Man erhält so leicht, dass $x_{1/2} = \frac{-b \pm \sqrt{b^2 - 4ac}}{2a}$
\subsubsection*{\Large Verwendbarkeit, Studien- und Prüfungsleistungen:}
\begin{tabularx}{\textwidth}{ p{.5\textwidth}
    }
    \\[2ex] \hline 
\end{tabularx}

\clearpage
\phantomsection
\thispagestyle{empty}
\vspace*{\fill}
\begin{center}
\Huge\bfseries 4c. Kolloquien und weitere Veranstaltungen
\end{center}
\addcontentsline{toc}{section}{\textbf{4c. Kolloquien und weitere Veranstaltungen}}
\addtocontents{toc}{\medskip\hrule\medskip}\vspace*{\fill}\vspace*{\fill}\clearpage
\vfill
\thispagestyle{empty}
\clearpage

\clearpage\hrule\vskip1pt\hrule 
\section*{\Large Seminar über Datenanalyse und Modellbildung}
\addcontentsline{toc}{subsection}{Seminar über Datenanalyse und Modellbildung\ \textcolor{gray}{(\em Harald Binder, Peter Pfaffelhuber, Angelika Rohde, Thorsten Schmidt, Jens Timmer)}}
\vskip-2ex  
Harald Binder, Peter Pfaffelhuber, Angelika Rohde, Thorsten Schmidt, Jens Timmer\\
Raum und Zeit: Fr, 12--13, SR 404, \href{https://www.openstreetmap.org/?mlat=48.00065\&mlon=7.84591\#map=19/48.00065/7.84591}{Ernst-Zermelo-Straße 1}\\
\subsubsection*{\Large Inhalt:}
Voluptatem sed amet porro numquam modi quisquam. Tempora consectetur eius velit aliquam adipisci quiquia est. Consectetur dolor non adipisci amet porro. Velit quisquam quiquia eius est aliquam. Consectetur non modi ut. Numquam velit ut quiquia. Est quiquia est aliquam.
\subsubsection*{\Large Literatur:}
Tempora neque aliquam non est.
\subsubsection*{\Large Vorkenntnisse:}
Man erhält so leicht, dass $x_{1/2} = \frac{-b \pm \sqrt{b^2 - 4ac}}{2a}$
\subsubsection*{\Large Verwendbarkeit, Studien- und Prüfungsleistungen:}
\begin{tabularx}{\textwidth}{ p{.5\textwidth}
    }
    \\[2ex] \hline 
\end{tabularx}

\clearpage\hrule\vskip1pt\hrule 
\section*{\Large Didaktisches Seminar}
\addcontentsline{toc}{subsection}{Didaktisches Seminar\ \textcolor{gray}{(\em Katharina Böcherer-Linder, Ernst Kuwert)}}
\vskip-2ex  
Katharina Böcherer-Linder, Ernst Kuwert\\
Raum und Zeit: Di, 18: 30--20, HS II, \href{https://www.openstreetmap.org/?mlat=48.00233\&mlon=7.84788\#map=19/48.00233/7.84788}{Albertstr. 23b}\\
\subsubsection*{\Large Inhalt:}
Porro dolorem non quaerat non. Sit tempora labore sit velit quisquam numquam est. Non sit non velit voluptatem est sit quisquam. Adipisci voluptatem dolore quisquam. Non adipisci ipsum etincidunt.
\subsubsection*{\Large Literatur:}
Quisquam dolor sed porro etincidunt.
\subsubsection*{\Large Vorkenntnisse:}
Man erhält so leicht, dass $x_{1/2} = \frac{-b \pm \sqrt{b^2 - 4ac}}{2a}$
\subsubsection*{\Large Verwendbarkeit, Studien- und Prüfungsleistungen:}
\begin{tabularx}{\textwidth}{ p{.5\textwidth}
    }
    \\[2ex] \hline 
\end{tabularx}

\clearpage\hrule\vskip1pt\hrule 
\section*{\Large Kolloquium der Mathematik}
\addcontentsline{toc}{subsection}{Kolloquium der Mathematik\ \textcolor{gray}{(\em Nadine Große, Amador Martín Pizarro)}}
\vskip-2ex  
Nadine Große, Amador Martín Pizarro\\
Raum und Zeit: Do, 15--16, HS II, \href{https://www.openstreetmap.org/?mlat=48.00233\&mlon=7.84788\#map=19/48.00233/7.84788}{Albertstr. 23b}\\
\subsubsection*{\Large Inhalt:}
Porro quisquam quiquia dolore adipisci sed neque. Etincidunt sit ipsum labore dolorem. Ut neque magnam dolore quiquia. Eius ipsum ut quisquam quaerat quiquia. Adipisci non consectetur labore porro. Magnam numquam ut numquam ut.
\subsubsection*{\Large Literatur:}
Est labore quisquam quiquia consectetur ipsum dolore.
\subsubsection*{\Large Vorkenntnisse:}
Man erhält so leicht, dass $x_{1/2} = \frac{-b \pm \sqrt{b^2 - 4ac}}{2a}$
\subsubsection*{\Large Verwendbarkeit, Studien- und Prüfungsleistungen:}
\begin{tabularx}{\textwidth}{ p{.5\textwidth}
    }
    \\[2ex] \hline 
\end{tabularx}

\clearpage\hrule\vskip1pt\hrule 
\section*{\Large Mathematisches Kolloquium für Studierende}
\addcontentsline{toc}{subsection}{Mathematisches Kolloquium für Studierende\ \textcolor{gray}{(\em Annette Huber-Klawitter, Markus Junker, Amador Martín Pizarro)}}
\vskip-2ex  
Annette Huber-Klawitter, Markus Junker, Amador Martín Pizarro\\
Raum und Zeit: Do, 14--15, HS II, \href{https://www.openstreetmap.org/?mlat=48.00233\&mlon=7.84788\#map=19/48.00233/7.84788}{Albertstr. 23b}\\
\subsubsection*{\Large Inhalt:}
Voluptatem non ipsum non. Etincidunt etincidunt tempora neque. Ipsum numquam modi neque dolorem etincidunt etincidunt modi. Non voluptatem dolorem quiquia aliquam sit ipsum. Adipisci adipisci quaerat labore. Sit consectetur non numquam eius.
\subsubsection*{\Large Literatur:}
Neque magnam quiquia eius.
\subsubsection*{\Large Vorkenntnisse:}
Man erhält so leicht, dass $x_{1/2} = \frac{-b \pm \sqrt{b^2 - 4ac}}{2a}$
\subsubsection*{\Large Verwendbarkeit, Studien- und Prüfungsleistungen:}
\begin{tabularx}{\textwidth}{ p{.5\textwidth}
    }
    \\[2ex] \hline 
\end{tabularx}

