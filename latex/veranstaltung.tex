
\clearpage\hrule\vskip1pt\hrule 
\section*{\Large Basics in Applied Mathematics}
\addcontentsline{toc}{subsection}{Basics in Applied Mathematics\ \textcolor{gray}{(\textit{Patrick Dondl, Angelika Rohde})}}
\vskip-2ex  
{\itshape Patrick Dondl, Angelika Rohde}, 
Assistenz: 
{\itshape N.N.}
\\
4-stündige Vorlesung mit 2-stündiger Übung und 2-stündiger Pratkischer Übung \\
Vorlesung: Di, Do 8-10, HS II, Albertstr. 23b\\ Praktische Übung: Termine werden noch festgelegt


\begin{tabularx}{\textwidth}{ p{.5\textwidth}
    X
    }
    & 
    \makecell[c]{\rotatebox[origin=l]{90}{\parbox{
    3
        cm}{\begin{flushleft}
        Basics in Applied Mathematics (MScData 2024)
    \end{flushleft} }}} 
    \\[2ex] \hline 
    \rule[0mm]{0cm}{.6cm}Erreichen von mindestens 50\% der Punkte, die insgesamt durch die Bearbeitung für die Übung ausgegebenen Übungsaufgaben erreicht werden können. (SL)\rule[-3mm]{0cm}{0cm}
    &
    \makecell[c]{\xmark}
    \\
    \rule[0mm]{0cm}{.6cm}Mindestens 1-maliges Vorrechnen von Übungsaufgaben im Tutorat. (SL)\rule[-3mm]{0cm}{0cm}
    &
    \makecell[c]{\xmark}
    \\
    \rule[0mm]{0cm}{.6cm}Klausur (ein- bis dreistündig). (SL)\rule[-3mm]{0cm}{0cm}
    &
    \makecell[c]{\xmark}
    \\
\end{tabularx}



\clearpage\hrule\vskip1pt\hrule 
\section*{\Large Introduction to Theory and Numerics of Partial Differential Equations}
\addcontentsline{toc}{subsection}{Introduction to Theory and Numerics of Partial Differential Equations\ \textcolor{gray}{(\textit{Diora Salimova})}}
\vskip-2ex  
{\itshape Diora Salimova}, 
Assistenz: 
{\itshape N.N.}
\\
4-stündige Vorlesung mit 2-stündiger Übung \\
Vorlesung: Di, Fr 12-14, SR 226, Hermann-Herder-Str. 10\\ Übung (2-stündig): Termin wird noch festgelegt


\begin{tabularx}{\textwidth}{ p{.5\textwidth}
    X
    X
    X
    X
    X
    X
    }
    & 
    \makecell[c]{\rotatebox[origin=l]{90}{\parbox{
    8
        cm}{\begin{flushleft}
        Modul Mathematik (MSc 2014)
    \end{flushleft} }}} 
    & 
    \makecell[c]{\rotatebox[origin=l]{90}{\parbox{
    8
        cm}{\begin{flushleft}
        Modul Angewandte Mathematik (MSc 2014)
    \end{flushleft} }}} 
    & 
    \makecell[c]{\rotatebox[origin=l]{90}{\parbox{
    8
        cm}{\begin{flushleft}
        Teile des Vertiefungsmoduls (MSc 2014)
    \end{flushleft} }}} 
    & 
    \makecell[c]{\rotatebox[origin=l]{90}{\parbox{
    8
        cm}{\begin{flushleft}
        Wahlpflichtmodul Mathematik (BSc 2021)
    \end{flushleft} }}} 
    & 
    \makecell[c]{\rotatebox[origin=l]{90}{\parbox{
    8
        cm}{\begin{flushleft}
        Advanced Lecture in Numerics/Stochastics (MScData 2024)
    \end{flushleft} }}} 
    & 
    \makecell[c]{\rotatebox[origin=l]{90}{\parbox{
    8
        cm}{\begin{flushleft}
        Electives in Data (MScData 2024)
    \end{flushleft} }}} 
    \\[2ex] \hline 
    \rule[0mm]{0cm}{.6cm}Mündliche Abschlussprüfung (Dauer ca. 30 Minuten) (PL)\rule[-3mm]{0cm}{0cm}
    &
    \makecell[c]{\xmark}
    &
    \makecell[c]{\xmark}
    &
    &
    &
    \makecell[c]{\xmark}
    &
    \\
    \rule[0mm]{0cm}{.6cm}Mündliche Abschlussprüfung über alle Teile des Moduls (Dauer ca. 45 Minuten) (PL)\rule[-3mm]{0cm}{0cm}
    &
    &
    &
    \makecell[c]{\xmark}
    &
    &
    &
    \\
    \rule[0mm]{0cm}{.6cm}Klausur (ein- bis dreistündig). (PL)\rule[-3mm]{0cm}{0cm}
    &
    &
    &
    &
    \makecell[c]{\xmark}
    &
    &
    \makecell[c]{\xmark}
    \\
    \rule[0mm]{0cm}{.6cm}Klausur (ein- bis dreistündig). (SL)\rule[-3mm]{0cm}{0cm}
    &
    \makecell[c]{\xmark}
    &
    \makecell[c]{\xmark}
    &
    \makecell[c]{\xmark}
    &
    &
    \makecell[c]{\xmark}
    &
    \\
    \rule[0mm]{0cm}{.6cm}Erreichen von mindestens 50\% der Punkte, die insgesamt durch die Bearbeitung für die Übung ausgegebenen Übungsaufgaben erreicht werden können. (SL)\rule[-3mm]{0cm}{0cm}
    &
    \makecell[c]{\xmark}
    &
    \makecell[c]{\xmark}
    &
    \makecell[c]{\xmark}
    &
    \makecell[c]{\xmark}
    &
    \makecell[c]{\xmark}
    &
    \makecell[c]{\xmark}
    \\
    \rule[0mm]{0cm}{.6cm}Mindestens 1-maliges Vorrechnen von Übungsaufgaben im Tutorat. (SL)\rule[-3mm]{0cm}{0cm}
    &
    \makecell[c]{\xmark}
    &
    \makecell[c]{\xmark}
    &
    \makecell[c]{\xmark}
    &
    \makecell[c]{\xmark}
    &
    \makecell[c]{\xmark}
    &
    \makecell[c]{\xmark}
    \\
\end{tabularx}



\clearpage\hrule\vskip1pt\hrule 
\section*{\Large Mathematical Statistics}
\addcontentsline{toc}{subsection}{Mathematical Statistics\ \textcolor{gray}{(\textit{E. A. von Hammerstein})}}
\vskip-2ex  
{\itshape E. A. von Hammerstein}, 
Assistenz: 
{\itshape N.N.}
\\
4-stündige Vorlesung mit 2-stündiger Übung \\
Vorlesung: Di, Do 14-16, HS Weismann-Haus, Albertstr. 21a\\ Übung (2-stündig): Termin wird noch festgelegt


\begin{tabularx}{\textwidth}{ p{.5\textwidth}
    X
    X
    X
    X
    X
    X
    }
    & 
    \makecell[c]{\rotatebox[origin=l]{90}{\parbox{
    8
        cm}{\begin{flushleft}
        Modul Mathematik (MSc 2014)
    \end{flushleft} }}} 
    & 
    \makecell[c]{\rotatebox[origin=l]{90}{\parbox{
    8
        cm}{\begin{flushleft}
        Modul Angewandte Mathematik (MSc 2014)
    \end{flushleft} }}} 
    & 
    \makecell[c]{\rotatebox[origin=l]{90}{\parbox{
    8
        cm}{\begin{flushleft}
        Teile des Vertiefungsmoduls (MSc 2014)
    \end{flushleft} }}} 
    & 
    \makecell[c]{\rotatebox[origin=l]{90}{\parbox{
    8
        cm}{\begin{flushleft}
        Wahlpflichtmodul Mathematik (BSc 2021)
    \end{flushleft} }}} 
    & 
    \makecell[c]{\rotatebox[origin=l]{90}{\parbox{
    8
        cm}{\begin{flushleft}
        Advanced Lecture in Numerics/Stochastics (MScData 2024)
    \end{flushleft} }}} 
    & 
    \makecell[c]{\rotatebox[origin=l]{90}{\parbox{
    8
        cm}{\begin{flushleft}
        Electives in Data (MScData 2024)
    \end{flushleft} }}} 
    \\[2ex] \hline 
    \rule[0mm]{0cm}{.6cm}Mündliche Abschlussprüfung (Dauer ca. 30 Minuten) (PL)\rule[-3mm]{0cm}{0cm}
    &
    \makecell[c]{\xmark}
    &
    \makecell[c]{\xmark}
    &
    &
    &
    \makecell[c]{\xmark}
    &
    \\
    \rule[0mm]{0cm}{.6cm}Mündliche Abschlussprüfung über alle Teile des Moduls (Dauer ca. 45 Minuten) (PL)\rule[-3mm]{0cm}{0cm}
    &
    &
    &
    \makecell[c]{\xmark}
    &
    &
    &
    \\
    \rule[0mm]{0cm}{.6cm}Erreichen von mindestens 50\% der Punkte, die insgesamt durch die Bearbeitung für die Übung ausgegebenen Übungsaufgaben erreicht werden können. (SL)\rule[-3mm]{0cm}{0cm}
    &
    \makecell[c]{\xmark}
    &
    \makecell[c]{\xmark}
    &
    \makecell[c]{\xmark}
    &
    \makecell[c]{\xmark}
    &
    \makecell[c]{\xmark}
    &
    \makecell[c]{\xmark}
    \\
    \rule[0mm]{0cm}{.6cm}Mindestens 1-maliges Vorrechnen von Übungsaufgaben im Tutorat. (SL)\rule[-3mm]{0cm}{0cm}
    &
    \makecell[c]{\xmark}
    &
    \makecell[c]{\xmark}
    &
    \makecell[c]{\xmark}
    &
    \makecell[c]{\xmark}
    &
    \makecell[c]{\xmark}
    &
    \makecell[c]{\xmark}
    \\
\end{tabularx}



\clearpage\hrule\vskip1pt\hrule 
\section*{\Large Measure Theory for Probabilists}
\addcontentsline{toc}{subsection}{Measure Theory for Probabilists\ \textcolor{gray}{(\textit{Peter Pfaffelhuber})}}
\vskip-2ex  
{\itshape Peter Pfaffelhuber}, 
Assistenz: 
{\itshape Samuel Adeosun}
\\
Online-Kurs mit 2-stündiger Übung \\
Vorlesung (2-stündig): asynchrone Videos\\ Übung (2-stündig): Termin wird noch festgelegt


\begin{tabularx}{\textwidth}{ p{.5\textwidth}
    X
    }
    & 
    \makecell[c]{\rotatebox[origin=l]{90}{\parbox{
    3
        cm}{\begin{flushleft}
        Electives (MScData 2024)
    \end{flushleft} }}} 
    \\[2ex] \hline 
    \rule[0mm]{0cm}{.6cm}Erreichen von mindestens 50\% der Punkte, die insgesamt durch die Bearbeitung für die Übung ausgegebenen Übungsaufgaben erreicht werden können. (SL)\rule[-3mm]{0cm}{0cm}
    &
    \makecell[c]{\xmark}
    \\
    \rule[0mm]{0cm}{.6cm}Mindestens 1-maliges Vorrechnen von Übungsaufgaben im Tutorat. (SL)\rule[-3mm]{0cm}{0cm}
    &
    \makecell[c]{\xmark}
    \\
\end{tabularx}



\clearpage\hrule\vskip1pt\hrule 
\section*{\Large Praktische Übung zu 'Introduction to Theory and Numerics of Partial Differential Equations'}
\addcontentsline{toc}{subsection}{Praktische Übung zu 'Introduction to Theory and Numerics of Partial Differential Equations'\ \textcolor{gray}{(\textit{Diora Salimova})}}
\vskip-2ex  
{\itshape Diora Salimova}, 
Assistenz: 
{\itshape N.N.}
\\
2-stündige Praktische Übung \\
2-stündig: Termin wird noch festgelegt


\begin{tabularx}{\textwidth}{ p{.5\textwidth}
    X
    X
    X
    }
    & 
    \makecell[c]{\rotatebox[origin=l]{90}{\parbox{
    3
        cm}{\begin{flushleft}
        Teil des Wahlmoduls (MSc 2014)
    \end{flushleft} }}} 
    & 
    \makecell[c]{\rotatebox[origin=l]{90}{\parbox{
    3
        cm}{\begin{flushleft}
        Teil des Wahlmoduls (BSc 2021)
    \end{flushleft} }}} 
    & 
    \makecell[c]{\rotatebox[origin=l]{90}{\parbox{
    3
        cm}{\begin{flushleft}
        Electives (MScData 2024)
    \end{flushleft} }}} 
    \\[2ex] \hline 
    \rule[0mm]{0cm}{.6cm}Regelmäßige Teilnahme an der Veranstaltung (wie in der Prüfungsordnung definiert). (SL)\rule[-3mm]{0cm}{0cm}
    &
    \makecell[c]{\xmark}
    &
    \makecell[c]{\xmark}
    &
    \makecell[c]{\xmark}
    \\
    \rule[0mm]{0cm}{.6cm}Erreichen von mindestens 50\% der Punkte, die insgesamt durch die Bearbeitung der für die Praktische Übung ausgegebenen Programmieraufgaben erreicht werden können. (SL)\rule[-3mm]{0cm}{0cm}
    &
    \makecell[c]{\xmark}
    &
    \makecell[c]{\xmark}
    &
    \makecell[c]{\xmark}
    \\
\end{tabularx}



\clearpage\hrule\vskip1pt\hrule 
\section*{\Large Medical Data Science}
\addcontentsline{toc}{subsection}{Medical Data Science\ \textcolor{gray}{(\textit{Harald Binder})}}
\vskip-2ex  
{\itshape Harald Binder}, 
Assistenz: 
{\itshape N.N.}
\\
Seminar \\
Mi 10-11:30, HS Medizinische Biometrie, Stefan-Meier-Str. 26


\begin{tabularx}{\textwidth}{ p{.5\textwidth}
    X
    X
    X
    }
    & 
    \makecell[c]{\rotatebox[origin=l]{90}{\parbox{
    3
        cm}{\begin{flushleft}
        Mathematisches Seminar (BSc 2021)
    \end{flushleft} }}} 
    & 
    \makecell[c]{\rotatebox[origin=l]{90}{\parbox{
    3
        cm}{\begin{flushleft}
        Seminar A/B (MSc 2014)
    \end{flushleft} }}} 
    & 
    \makecell[c]{\rotatebox[origin=l]{90}{\parbox{
    3
        cm}{\begin{flushleft}
        Seminar (MScData 2024)
    \end{flushleft} }}} 
    \\[2ex] \hline 
    \rule[0mm]{0cm}{.6cm}Vortrag (Dauer 45 bis 90 Minuten) (PL)\rule[-3mm]{0cm}{0cm}
    &
    \makecell[c]{\xmark}
    &
    \makecell[c]{\xmark}
    &
    \makecell[c]{\xmark}
    \\
    \rule[0mm]{0cm}{.6cm}Regelmäßige Teilnahme an der Veranstaltung (wie in der Prüfungsordnung definiert). (SL)\rule[-3mm]{0cm}{0cm}
    &
    \makecell[c]{\xmark}
    &
    \makecell[c]{\xmark}
    &
    \makecell[c]{\xmark}
    \\
\end{tabularx}





\newpage\section*{Abkürzungen}
\addcontentsline{toc}{section}{Abkürzungen}

\begin{tabular}{p{.2\textwidth}p{.8\textwidth}}
MScData (2014) & Master of Science, Prüfungsordnung 2014 \\
BSc (2021) & Bachelor of Science Mathematik, Prüfungsordnung 2021 \\
MScData (2024) & Master of Science Mathematics in Data and Technology, Prüfungsordnung 2024 \\
PL & Prüfungsleistung \\
SL & Studienleistung \\
\end{tabular}
